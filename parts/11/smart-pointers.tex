\section{Smart Pointers}

Memory allocation in \cpp{} is a tough subject, dynamic allocation
being the hardest part.

Before \cpp11, the programmer had to be very careful with the life
span of dynamically allocated memory in order to, first, be sure that
it is released at some point and, second, that no access is made to it
once it has been released, not even another release.

See how many problems could occur with this short snippet:

\begin{lstlisting}
struct foo
{
  foo(int* p)
    : m_p(p)
  {}

  ~foo()
  {
    delete p;
  }

private:
  int* m_p;
};

void bar()
{
  foo f1(new int);
  foo f2(f1);

  int* p1(new int);
}
\end{lstlisting}

There are two problems with this code. First, \code{foo} does not
define nor disable its copy constructor, so, by default, its
\code{m\_p} pointer will be copied to the new instance. Then, when the
original instance and its copy will be destroyed, both will call
\code{delete} on the same pointer. This is what will happen with
\code{f1} and its copy \code{f2}. In the best scenario the program
would crash here.

The second problem is \code{p1}. This pointer points to a dynamically
allocated \code{int} for which no \code{delete} is written. When the
pointer will go out of scope then there will be no way to release the
allocated memory.

%-------------------------------------------------------------------------------
\subsection{\code{std::unique\_ptr}}

\Cpp11 introduces a pointer wrapper named \code{std::unique\_ptr},
which has the merit of automatically calling \code{delete} on the
pointer upon destruction. Applied to the previous example, it would
solve one problem and force us to find a solution for the other:

\begin{lstlisting}
struct foo
{
  foo(std::unique_ptr<int> p)
    : m_p(std::move(p))
  {}

  // The default destructor does the job.

private:
  std::unique_ptr<int> m_p;
};

void bar()
{
  std::unique_ptr<int> p1(new int);

  foo f1(std::unique_ptr<int>(new int));
  foo f2(std::move(f1));
}
\end{lstlisting}

This program is undoubtedly safer than the previous one. The copy
constructor of \code{foo} is still not defined, but it is for sure
deleted since \code{std::unique\_ptr} has no copy constructor
neither. So, by default, we cannot share the resource between two
instances, which is great. The only solution here is to either
allocate a new int for \code{f2} or steal the one from {f1}. The
latter is implemented here.

Then, for the release of \code{p1}, it is automatically done when the
variable leaves the scope, so no memory is leaked.

\bigskip

One of the best features of \code{std::unique\_ptr} is the possibility
to use a custom deleter to release the pointer. This makes this smart
pointer a tool of choice when using C-like resources.

Check for example the use case of libavcodec's
\code{AVFormatContext}. The format context is obtained via a call to
\code{avformat\_open\_input(AVFormatContext**, const char*,
  AVInputFormat*, AVDictionary**)} and must be released by a call to
\code{avformat\_close\_input(AVFormatContext**)}. With the help of
\code{std::unique\_ptr} this could be done as follows:

\begin{lstlisting}
namespace detail
{
  static void close_format_context(AVFormatContext* context);
}

void foo(const char* path)
{
  AVFormatContext* raw_context_pointer(nullptr);

  const int open_result =
    avformat_open_input
      (&raw_context_pointer, path, nullptr, nullptr);

  std::unique_ptr
  <
    AVFormatContext,
    decltype(&detail::close_format_context)
  >
  context_pointer
  (raw_context_pointer, &detail::close_format_context);

  // …
}
\end{lstlisting}


With this approach, the format context will be released via a call to
\code{detail::close\_format\_context} as soon as
\code{context\_pointer} leaves the scope. Do we know if the memory
pointed by \code{raw\_context\_pointer}? We do not, and it does not
matter. What we have here is a simple robust way to attach a release
function to an acquired resource.

%-------------------------------------------------------------------------------
\subsection{\code{shared\_ptr}}

%-------------------------------------------------------------------------------
\subsection{\code{weak\_ptr}}

