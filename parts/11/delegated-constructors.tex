%-------------------------------------------------------------------------------
\subsection{Delegated Constructors}

\problemtitle

Before \cpp11, constructors could not call other
constructors. Consider the example below:

\begin{lstlisting}
struct foo
{
  foo(bar* b, float f, int i)
    : m_bar(b),
      m_f(f),
      m_i(i)
  {}

  foo(bar* b, int i)
    // can't I call foo(b, 0, i) directly?
    : m_bar(b),
      m_f(0),
      m_i(i)
    {}

private:
  bar* const m_bar;
  const float m_f
  const int m_i;
};
\end{lstlisting}

If we want to share initialization code between multiple constructors,
then we have to put it in some separate member function called by the
constructor, like this:

\begin{lstlisting}
struct foo
{
  foo(bar* b, float f, int i)
  {
    @\emcode{init}@(b, f, i);
  }

  foo(bar* b, int i)
  {
    @\emcode{init}@(b, 0, i);
  }

private:
  void @\emcode{init}@(bar* b, float f, int i)
  {
    m_bar = b;
    m_f = f;
    m_i = i;
  }

  // We cannot make any of these members const anymore.
  bar* m_bar;
  float m_f;
  int m_i;
};
\end{lstlisting}

This approach was kind of error-prone. Stuff may happen before and
after the call to the \code{init()} function, and actually nothing
prevents it to be called at any point in the life of the
instance. Finally, this is incompatible with \code{const} members.

\solutiontitle

Starting from \cpp11, a constructor can call another constructor:

\begin{lstlisting}
struct foo
{
  foo(bar* b, float f, int i)
    : m_bar(b),
      m_f(f),
      m_i(i)
  {}

  foo(bar* b, int i)
    : @\emcode{foo}@(b, 0, i)
  {}

private:
  bar* const m_bar;
  float m_f;
  int m_i;
};
\end{lstlisting}

This solves all problems.
