\subsection{Compile-Time Assertions}

\problemtitle

Compile-time assertions, declared with a \code{static\_assert}, are a
way to verify a property at compile time.

For example, the code below declares a type with an array whose size,
with respect to the type's name, must be even:

\begin{lstlisting}
template<typename T, unsigned N>
struct even_sized_array
{
  typedef T type[N];
};
\end{lstlisting}

How can we ensure that \code{N} is always even? Before \cpp11 a
solution to that would have been to enter some template dance such
that passing an odd value would instantiate an incomplete type.

\lstinputlisting{examples/static-assert/even_sized_array-98.cpp}

Not only this is verbose, but these kind of template instantiation
errors are well known for producing kilometer-long unbearable error
messages.

\solutiontitle

Enters \code{static\_assert} in \cpp11. Just like the good old
\code{assert(condition)} would check that the given condition is true
at run time, \code{static\_assert(condition, message)} will check the
condition during the compilation.

\lstinputlisting{examples/static-assert/even_sized_array-11.cpp}

This is very less verbose than previously and it actually carries the
intent. The error message is also clearer, as it is just something
like ``static assertion failed: the message''.
