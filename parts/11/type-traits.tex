\subsection{Type Traits}

Let's have a look at this small function:

\begin{lstlisting}
template<typename T>
T mix(T a, T b, T r)
{
  return r * a + (1 - r) * b;
}
\end{lstlisting}

This is a quite basic function that just combine two values with a
weighted sum. Its implementation tells us that $r$ should probably be
in $[0, 1]$, but most importantly, the computation makes sense only if
T is a float-like type. If we want to prevent the compiler or the
programmer to call this function with an incompatible type, we can add
some SFINAE\footnote{Substitution Failure Is Not An Error, is a
  principle in template instantiations according to which the compiler
  must not emit an error if it fails to instantiate a
  template. Instead it should try another template candidate. This
  feature is often used and abused to provide multiple implementation
  behind the same function signature.}, for example by introducing a
type deduced from T that would not be defined if T is not a float-like
type. In \cpp98 the implementation could have been similar to:

\lstinputlistinghl{21}{examples/type-traits/mix-98.hpp}

Aside from the fact that \code{long double} is not handled, does it
work as expected?

\begin{lstlisting}
void test()
{
  // float and double are handled as expected.
  printf("%f\n", mix(1.f, 3.f, 0.5f));
  printf("%f\n", mix(1.d, 3.d, 0.5d));

  // Using integers triggers an error, we can say that it fails successfully!
  // printf("%d\n", mix(1, 3, 2));
}
\end{lstlisting}

Typing custom type like \code{only\_if\_float\_like} for every use
case is repetitive, so we would certainly end up splitting it into two
types, that could be used like
\code{only\_if<is\_float\_like<T>::value, T>::type}.

\bigskip

\marginheader{<type\_traits>}%
%
Lucky us, \cpp11 greatly simplify this work by introducing the
required types, in the form of \code{std::enable\_if} and
\code{std::is\_floating\_point\_type}. Using these types the whole
implementation is reduced to the following:

\lstinputlistinghl{4}{examples/type-traits/mix-11.hpp}

There are many other predicates and operations added in
\cppheader{type\_traits} in \cpp11, that can greatly help for
metaprogramming. I can only suggest to have a look at them on a nice
online reference.
