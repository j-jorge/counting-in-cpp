\subsection{Hash Tables}

There were two main associative containers in \cpp{} before \cpp11:
\code{std::vector} and the likes, to associate integer keys with
almost any value type, and \code{std::map}, to use any key type whose
values can be strictly ordered. Similar to the latter, \code{std::set}
is an ordered set of values. In practice, \code{std::map} is an
\code{std::set} where the value type is \code{std::pair}, with the key
as the first field (with a const modifier), and the value as the
second field.

These last two collections had several problems, the main one being
that they require a total order on their entries. However, it is not
exceptional to have a type for which \code{operator<()} is not defined
and where providing one would be weird.

\begin{lstlisting}
struct color
{
  uint8_t red;
  uint8_t green;
  uint8_t blue;

  // Does it make sense to say that a color is less than another color?
  bool operator<(const color& that) const
  {
    // Note that in @\cpp@11 we could have used std::tie() to create a
    // tuple from the fields @\aref{tuple}@, then used
    // std::tuple::operator<() to compare them lexicographically.

    if (red != that.red)
      return red < that.red;

    if (green != that.green)
      return green < that.green;

    return blue < that.blue;
  }
};
\end{lstlisting}

A way to avoid the weird operator is to declare the comparison
operator outside the type, as an independent function object. It has
the effect of not pretending that there is a meaningful order on the
given type. The type of the function object is then passed to
\code{std::set} or \code{std::map}; it becomes a property of the
container (i.e. the way its entries are ordered) instead of a property
of the data.

\begin{lstlisting}
struct color
{
  uint8_t red;
  uint8_t green;
  uint8_t blue;
};

// This is one way to order, amongst many. The advantage here is that
// there is no inherent ordering attached to color.
struct color_lexicographical_order
{
  bool operator()(const color& left, const color& right) const
  {
    if (left.red != right.red)
      return left.red < right.red;

    if (left.green != right.green)
      return left.green < right.green;

    return left.blue < right.blue;
  }
};

std::set<color, color_lexicographical_order> used_colors;
\end{lstlisting}

Another problem with \code{std::set} and \code{std::map} is that they
are implemented as a balanced tree, typically a red-black tree, where
the nodes are dynamically allocated. This is not very
performance-friendly since the data ends up scattered in memory, with
an additional cost of three pointers per entry. Additionally, look-up
is logarithmic.

\bigskip

\marginheader{<unordered\_map>,\\
<unordered\_set>}%
%
Since \cpp11, these containers are advantageously replaced by
\code{std::unordered\_set} and \code{std::unordered\_map}, which
provide the same service but with an implementation based on hash
tables. Now, look-up is more often constant-time than anything
else. The difficulty being to find a good hash function for the stored
type.

\begin{lstlisting}
struct color
{
  // Both std::unordered_set and std::unordered_map need a way to tell
  // if two instances are equal, in case of a hash collision.
  bool operator==(const color& that) const

  uint8_t red;
  uint8_t green;
  uint8_t blue;
};

namespace std
{
  // We can specialize std::hash for our own types. It is one of the
  // few symbols from the STL that we are allowed to specialize.
  template<>
  struct hash<color>
  {
    std::size_t operator()(const color& c) const
    {
      return
        ((std::size_t)c.red << 16)
        | ((std::size_t)c.green << 8)
        | (std::size_t)c.blue;
    }
  };
}

// No need to define the hash since we defined it via
// std::hash. Alternatively, we could have set the second template
// parameter std::unordered_set<color, some_hash_type>.
std::unordered_set<color> used_colors;
\end{lstlisting}

Moreover, a new way to insert elements in a set or a map has been
introduced, via the \code{emplace()} method. This method will create
the item in-place, without copies, while the previous \code{insert()}
approach would copy its argument into the container. It is available
both for the ordered and unordered variants:

\begin{lstlisting}
struct color
{
  color(uint8_t r, uint8_t g, uint8_t b);

  bool operator==(const color& that) const

  uint8_t red;
  uint8_t green;
  uint8_t blue;
};

std::unordered_set<color> used_colors;

// The arguments are forwarded to the constructor of color.
used_colors.emplace(218, 13, 13);
\end{lstlisting}

For the map version, since the entries are instances of
\code{std::pair}, we cannot pass all parameters in a single argument
list, as the compiler would not know where the constructor arguments
for \code{std::pair::first} would end and where the ones for
\code{std::pair::second} would start. Consequently, we must use tuples
\aref{tuple} and the piecewise constructor:

\begin{lstlisting}
std::unordered_map<color, unsigned> color_count;

// The std::piecewise_construct is here to select the corresponding
// constructor from std::pair. The triplet argument will be forwarded
// directly to the construction of the first member; The second
// singleton argument will be forwarded to the second member.
color_count.emplace
  (std::piecewise_construct,
   std::forward_as_tuple(218, 13, 13),
   std::forward_as_tuple(1));
\end{lstlisting}

The main problem with \code{std::unordered\_set} and
\code{std::unordered\_map} is that they perform quite
poorly. Unfortunately, due to the constraints imposed by the standard,
they cannot be implemented in another way.

\begin{guideline}
If you need an hash table in the internals of your application or
library, i.e. it won't be visible by anything outside, then you
probably want to switch to another implementation. Otherwise, if your
library or application exposes a hash table, you have to consider:

\begin{enumerate}
\item if it makes sense to pass the exposed table as-is to another
  third-party application, then use the standard containers,
\item if the exposed table is expected not to go anywhere, then use a
  better container.
\end{enumerate}

So far, I have been quite satisfied with the robin hood hashing from
Martin Ankerl, a.k.a martinus, at
\url{https://github.com/martinus/robin-hood-hashing}.
\end{guideline}
