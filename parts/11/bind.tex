\subsection{Argument Bindings}

The bindings example from \ref{sec:tuple} was quite complex, and it
does not even handle member functions. Since it is already using
\cpp11 features, it is left to the reader to consider how to implement
something equivalent with pre-\cpp11 features.

\marginheader{<utility>}%
%
However, a better implementation of a binding is available in \cpp11,
via \code{std::bind()}. This function creates a function object, of an
unspecified type, that will forward its arguments to the bound
function. This is something I would typically use in event handling.

\begin{lstlisting}
struct message_queue { /* … */ };

struct message_counter
{
  void count_message();
};

void connect_handlers(message_queue& queue)
{
  message_counter counter;

  // Here std::bind() creates a function object that calls
  // counter.process_message() when invoked.
  //
  // Note that the arguments of the function object are deduced from
  // the signature of message_dispatcher::count_message.
  queue.on_message(@\emcode{std::bind}@(&message_counter::process_message, &counter));
}
\end{lstlisting}

Interestingly, it is also possible to either force the value of an
argument, or to redirect an argument from the caller to a specific
argument of the called.

\begin{lstlisting}
struct message_queue { /* … */ };

void log_error(int queue_id, error e);

void connect_error_handler(message_queue& queue, int queue_id)
{
  // This one creates a function object accepting a single argument,
  // that calls log_error(queue_id, e), for a given argument e.
  //
  // Note that we must explicitly tell what to do with the argument
  // given by the caller here, via the placeholder.
  queue.on_message(@\emcode{std::bind}@(&log_error, queue_id, std::placeholders::_1));
}
\end{lstlisting}

In the above code, \code{std::placeholders::\_1} tells
\code{std::bind} to pass whatever is received as the first argument
(because the \_1) to the second argument of \code{log\_error} (second
because it is the second following the function when the binding is
created).

Note that the standard does not define how many placeholders must be
defined.
