\subsection{Iterator Successors and Predecessors}

\problemtitle

There are many kind of iterators: forward iterators, that can be
incremented one step at a time with \code{operator++}, bidirectional
iterators, that can additionally be decremented with
\code{operator--}, and random access iterators, that can be
incremented or decremented by any amount at once.

When writing a function taking an iterator whose type is templated,
incrementing an iterator by more than one unit cannot be done
directly, as it would fail for non-random iterators. Prior to \cpp11,
this was done with \code{std::advance()}.

\begin{lstlisting}
template<typename Iterator, typename F>
void every_n_items
(Iterator it, std::size_t count, std::size_t step, F f)
{
  for (; count > step; count -= step)
  {
    f(*it);
    std::advance(it, step);
  }
}
\end{lstlisting}

\code{std::advance()} accepts a negative distance, in which case it
will advance… hum… backwards. Note that the function modifies its
argument, so using it to get another iterator from a given one is
cumbersome:

\begin{lstlisting}
template<typename Iterator>
void inplace_adjacent_sums(Iterator first, const Iterator& last)
{
  if (first == last)
    return;

  // Two steps to get an iterator on the second element. Ideally we
  // would have wanted to limit its scope to the loop too.
  Iterator second = first;
  std::advance(second, 1);

  for (; second != last; )
  {
    *first += *second;
    first = second;
    std::advance(second, 1);
  }
}
\end{lstlisting}

\solutiontitle

\marginheader{<iterator>}%
%
\Cpp11 introduces the replacement functions \code{std::prev()} and
\code{std::next()}, which are mostly here to make things clear and
more convenient. The former is to be used to move the iterator
backwards, while the latter is used to move it forward. If no distance
is passed, then it defaults to one.

Note that they return a copy of the new iterator, instead of modifying
its argument.

\begin{lstlisting}
template<typename Iterator>
void inplace_adjacent_sums(Iterator first, const Iterator& last)
{
  if (first == last)
    return;

  // Here we can initialize the second iterator in the loop, reducing
  // its scope.
  for (Iterator second = @\emcode{std::next}@(first); second != last; )
  {
    *first += *second;
    first = second;
    std::advance(second, 1);
  }
}
\end{lstlisting}
