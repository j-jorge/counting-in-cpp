\subsection{Arrays}

How do we usually pass a raw array to a function in \cpp? By passing
both a pointer to the first element and the number of elements, that
is how.

And when we need to copy such array, or to return such array from a
function, it becomes quite tricky.

\begin{lstlisting}
void replace_zeros_copy(int* out, int* in, std::size_t size, int r)
{
  // Can I trust the caller that the target array is large enough to
  // store the result? Is it a valid use case to pass null pointers
  // here? So many questions :(

  std::transform
    (in, in + size, out,
     [r](int v) -> int
     {
       return (v == 0) ? r : v;
     });
}

// We certainly cannot properly return a raw array unless we use
// some dynamic allocation or other techniques with their own
// problems.
int* replace_zeros_copy(int* array, std::size_t size, int r) { /* … */ }
\end{lstlisting}

With \cpp11 came \code{std::array} a template type to be used as an
alternative to raw arrays, with two parameters: the type of the
elements and their count.

\begin{lstlisting}
template<std::size_t N>
void replace_zeros_copy
(@\emcode{std::array<int, N>}@& out, const @\emcode{std::array<int, N>}@& in, int r)
{
  // Forget about the null pointers problem and the size issues,
  // everything fits perfectly here.

  std::transform
    (in.begin(), in.end(), out.begin(),
     [r](int v) -> int
     {
       return (v == 0) ? r : v;
     });
}

// We can also directly return the array.
template<std::size_t N>
@\emcode{std::array<int, N>}@ replace_zeros_copy(const @\emcode{std::array<int, N>}@& array, int r)
{
  std::array<int, N> result;
  replace_zeros_copy(result, array, r);
  return result;
}
\end{lstlisting}

An \code{std::array} is as cheap as a raw array. It has no fancy
constructor or any subtleties, and I can think of only two downsides
to using it:
\begin{enumerate}
\item the cost of including an extra header \cite{compile-health}, not
  negligible in this case as \cppheader{array} pulls
  \cppheader{utility} and more,
\item the need to carry the size as a template parameter.
\end{enumerate}

\begin{guideline}
  Because of the downsides associated with the type and its header,
  using \code{std::array} should always be preceded by two
  considerations:

  \begin{enumerate}
  \item Will the header propagate to many files?
  \item Will I have to templatize everything?
  \item Should I implement iterator-based algorithms instead?
  \end{enumerate}
\end{guideline}

Also, note that \code{std::array} is not a fixed-capacity vector. All
its entries are default initialized as soon as the array is
constructed. So if it contains complex types, it means that the
default constructor of this type is called for each entry.
