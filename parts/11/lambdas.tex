\subsection{Lambdas}
\label{sec:lambda}

\problemtitle

How would we pass a custom comparator to \code{std::max\_element}
before \cpp11, say to compare strings by increasing size? Certainly by
writing a free or static function taking two strings as arguments and
returning the result of the comparison.

\begin{lstlisting}
// Comparator for strings by increasing size.
// Declared as a free function.
static bool string_size_less_equal
(const std::string& lhs, const std::string& rhs)
{
  return lhs.size() < rhs.size();
}

std::size_t
largest_string_size(const std::vector<std::string>& strings)
{
  // The comparator is outside the scope of this function.
  return
    std::max_element
      (strings.begin(), strings.end(), &string_size_less_equal)
    ->size();
}
\end{lstlisting}

Now what if we needed a parameterized comparator, for example to call
\code{std::find\_if} to search for a string of a specific size? The
free function would not allow to store the size, so we would
certainly write a functor object, i.e. a \code{struct} storing the
size parameter and defining an \code{operator()} receiving a string
and returning the result of the comparison.

\begin{lstlisting}
// Need a function object if the comparator has parameters.
struct string_size_equals
{
  std::size_t size;

  bool operator()(const std::string& string) const
  {
    return string.size() == size;
  }
};

bool has_string_of_size
(const std::vector<std::string>& strings, std::size_t s)
{
  string_size_equals comparator = {s};

  return
    std::find_if(strings.begin(), strings.end(), comparator)
      != strings.end();
}
\end{lstlisting}

\solutiontitle

Having the comparators as independent objects or functions is nice if
they are used in many places, but for a single use it is undoubtedly
too verbose. And confusing too. Wouldn't it be clearer if the
single-use comparator was declared next to where it is used?
Thankfully this is something we can do with lambdas, starting with
\cpp11:

\begin{lstlisting}
bool has_string_of_size
(const std::vector<std::string>& strings, std::size_t s)
{
  // Only three lines for the equivalent of the type
  // declaration, definition and the instantiation.
  // The third argument is a lambda.
  return std::find_if
    (strings.begin(), strings.end(),
     [=](const std::string& string) -> bool
     {
       return string.size() == s;
     })
    != strings.end();
}
\end{lstlisting}

\begin{guideline}
  Mind the next reader: keep your lambdas small.
\end{guideline}

%-------------------------------------------------------------------------------
\subsubsection{The Internals of Lambdas}
\label{lambdas-internals}

A declaration like

\begin{lstlisting}
[a, b, c]( /* arguments */ ) -> T { /* statements */ }
\end{lstlisting}

is equivalent to

\begin{lstlisting}
struct something
{
  T operator()( /* arguments */ ) const { /* statements */ }

  /* deduced type */ a;
  /* deduced type */ b;
  /* deduced type */ c;
};
\end{lstlisting}

Actually the compiler will create a unique type like this struct for
every lambda we write.

More conceptually, the parts of a lambda are the following:

\bigskip

\begin{center}
[{\it capture}]({\it arguments}) {\it specifier} -\textgreater
        {\it return\_type} \{ {\it body} \}
\end{center}

\bigskip

Where:

\begin{itemize}
\item {\it capture} is a list of variables from the parent scope that
  must be accessible inside the lambda. Use \code{[=]} to tell the
  compiler to automatically copy any variable used by the lambda, or
  \code{[\&]} to keep a reference to the corresponding variables from
  the parent scope. Variables can also be captured in a fine-grained
  way, e.g. \code{[=a, \&b]} to copy the value of \code{a} but store a
  reference to \code{b}.
\item {\it arguments} are the arguments of the function.
\item By default the variables captured by value cannot be assigned to
  in the body, unless we put the mutable keyword in {\it
    specifier}. In effect, it removes the \code{const} from
  \code{operator()}.
\item {\it return\_type} is the type of the value returned by this
  function, if any, or void otherwise. Contrary to any other
  function, the return type appear after the argument list instead of
  before.
\item {\it body} is the body of the function.
\end{itemize}

