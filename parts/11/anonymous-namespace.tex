\subsection{Anonymous Namespaces}

In the pre \cpp11{} era we could tell the compiler that a given global
variable or free function was only used in the current translation
unit by preceding its declaration with the keyword \code{static}:

\begin{lstlisting}
static int global_int;
static void some_function() { /* … */ }
\end{lstlisting}

Thanks to this keyword the same symbol (variable or function) could
appear in multiple translation units without problems.

However, this keyword is not applicable to type declarations. So what
happened if we needed a class definition at file scope but happened to
use the same name in different files, like with \code{struct system}
below?

\lstinputlisting[title=sound\_system.cpp]
                {examples/anonymous-namespace/sound_system.cpp}
\lstinputlisting[title=main.cpp]{examples/anonymous-namespace/main.cpp}

If we try to compile the these files for example with \code{g++
  sound\_system.cpp main.cpp} then run the program, we would expect to
have the following output:

\begin{lstlisting}{language=bash}
$ ./a.out
sound system
main system
\end{lstlisting}

Instead, we may have the following output:

\begin{lstlisting}{language=bash}
$ ./a.out
sound system
sound system
\end{lstlisting}

I say ``may'' because this program violates the one definition rule,
which states that a symbol cannot have more than one definition in the
program. Here, however, the \code{struct system} has two definitions,
one in each cpp file. We are thus entering the territory of undefined
behavior\footnote{An undefined behavior is the result of an ill-formed
program, where the programmer wrote something explicitly forbidden by
the standard. Typical examples of undefined behavior are: comparing
iterators on different containers, overflowing a signed integer,
creating a reference to nullptr, or defining two different types with
the same name. The compiler can and does consider that such situations
cannot occur, and will optimize the binary accordingly.}.

\bigskip

With the anonymous namespaces from \cpp11, we can totally solve this
issue. Any symbol defined in an anonymous namespace has internal
linkage; which means that it cannot conflict with anything defined
outside the current compilation unit. Its usage is as we would expect:

\begin{lstlisting}
// No identifier following the keyword.
namespace
{
  // Everything declared here is available in the current
  // compilation unit.
}
\end{lstlisting}

Concretely, when the code above is modified as follows, the program
is well defined and the output matches the expectations.

\lstinputlisting[
  title=sound\_system-anonymous.cpp,
  emph=namespace
]{%
  examples/anonymous-namespace/sound_system-anonymous.cpp%
}

\lstinputlisting[
  title=main-anonymous.cpp,
  emph=namespace
]{%
  examples/anonymous-namespace/main-anonymous.cpp%
}
