\subsection{Using parent functions}

\problemtitle

Here is an actual problem I encountered when developing mobile games
in \cpp{}. When targeting Android devices, the application has to do
some calls from the \cpp{} part to the Java part. These calls are done
via the Java Native Interface (JNI), which is accessed via an object
of type \code{JNIEnv}.

Calling a Java method from Java requires the identifier of the class,
of type \code{jclass}, and the identifier of the method, of type
\code{jmethodID}. When all of them are known, we can call for example
a static method returning an integer as follows:

\begin{lstlisting}
JNIenv* env = /* … */;
jclass the_class = /* … */;
jmethodID the_method = /* … */;

const jint result =
  env->CallStaticIntMethod(the_class, the_method, /* arguments */);
\end{lstlisting}

I certainly did not want to write this stuff everywhere and was hoping
for something more like:

\begin{lstlisting}
// This variable represents the function to call.
static_method<jint> method = /* … */;

// Then we call it like a function.
const jint result = method(/* arguments */);
\end{lstlisting}

Since all method calls need the same parameters (i.e. the JNI
environment, the class, and the method), I put them in a base class
from which specialization for the returned type would be
created\footnote{This is simplified for the example. Check
  \url{https://github.com/IsCoolEntertainment/iscool-core/} for the
  actual code.}:

\begin{lstlisting}
class static_method_base
{
public:
  static_method_base
  ( JNIEnv* env, jclass class_id, jmethodID method_id );

protected:
  JNIEnv* m_env;
  jclass m_class;
  jmethodID m_method;
};
\end{lstlisting}

Then I would inherit from this class to create a function object for
each supported return type. Here for a static method returning an
integer:

\begin{lstlisting}
template< typename R >
class static_method;

template<>
class static_method<jint>:
  public static_method_base
{
public:
  // This constructor seems useless, doesn't it?
  static_method_base
  (JNIEnv* env, jclass class_id, jmethodID method_id)
    : static_method_base(env, class_id, method_id)
  {}

  template<typename... Arg>
  jint operator()(Arg&&... args) const;
};
\end{lstlisting}

At this point we have something weird: the only reason we have defined
a constructor in \code{static\_method\textless{}jint\textgreater} is
to be pass its argument to the parent class. Wouldn't it be better if
we could just say ``reuse the constructor from the parent''?

\solutiontitle

This is possible thanks to the \code{using} keyword from \cpp11:

\begin{lstlisting}
template< typename R >
class static_method;

template<>
class static_method<jint>:
  public static_method_base
{
public:
  // This is concise and precise.
  @\emcode{using static\_method\_base::static\_method\_base;}@

  template<typename... Arg>
  jint operator()(Arg&&... args) const;
};
\end{lstlisting}

When used like this, the \code{using} keyword imports the function
into the current class. It is also a way to solve the problem of
parent functions hidden by overloads in the child class:

\begin{lstlisting}
struct base
{
  void foo();
};

struct derived: base
{
  // This declaration hides base::foo().
  void foo(int);
};

void bar()
{
  derived d;
  // This fails, derived::foo requires an int argument.
  d.foo();
}
\end{lstlisting}

With the \code{using} keyword:

\begin{lstlisting}
struct base
{
  void foo();
};

struct derived: base
{
  @\emcode{using base::foo;}@
  void foo(int);
};

void bar()
{
  derived d;
  // This is ok, foo is a member function in derived.
  d.foo();
}
\end{lstlisting}
