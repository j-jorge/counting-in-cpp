%-------------------------------------------------------------------------------
\label{move}

Copies! Copies everywhere! And dynamic allocations too!

So was the world before \cpp11. In these old days, if we had to pass
a large object to a function that needed its own instance of it, then
we would have certainly created a copy of it.

\begin{lstlisting}
struct catalog
{
  catalog(const std::vector<item>& entries)
    // Here we have one allocaction, for the storage,
    // plus copies of the vector elements.
    : m_entries(entries)
  {}

private:
  std::vector<item> m_entries;
};

void create_catalog(int n)
{
  // One allocation of n items, plus their initialization.
  std::vector<item> entries(n);

  // ...

  catalog c(entries);
  // I don't need the entries anymore, but they are still here.
}
\end{lstlisting}

\marginheader{<utility>}%
%
\Cpp11 introduces the move semantics and the \code{std::move()}
function. In practice it means far fewer copies, as the instances'
data is transfered from one place to another.

\begin{lstlisting}
struct catalog
{
  // Note the pass-by-value for the argument.
  catalog(std::vector<item> entries)
    // Explicit transfer into m_entries, no allocation.
    : m_entries(@\emcode{std::move}@(entries))
  {}

private:
  std::vector<item> m_entries;
};

void create_catalog(int n)
{
  // The single allocation in this program.
  std::vector<item> entries(n);

  // ...

  // I don't need the entries anymore, so I transfer them.
  catalog c(@\emcode{std::move}@(entries));
}
\end{lstlisting}

In the example above, there is only one allocation, for the vector of
entries in \code{create\_catalog()}, and zero copies. The ownership of
the allocated space is actually transfered from one vector to another
every time \code{std::move()} is used, until it ends up
in \code{m\_entries}.

%-------------------------------------------------------------------------------
\subsection{The Move Constructor and Assignment Operator}
\label{sec:move-constructor}

In order to make it work, a new type of reference was added to
the language; and classes can use this reference in constructors and
assignment operators to implement the actual transfer of data from one
instance to another. So to mark an argument as moveable, use \&\&:

\begin{lstlisting}
struct foo
{
  foo(foo@\emcode{\&\&}@ that)
    : m_resource(that.m_resource)
  {
    that.m_resource = nullptr;
  }

  foo& operator=(foo@\emcode{\&\&}@ that)
  {
    std::swap(m_resource, that.m_resource);
  }
};

// All of these call the constructor defined above.
foo f(foo());
foo f(construct_a_foo());
foo f(std::move(g));

// All of these call the assignment operator defined above.
f = foo();
f = construct_a_foo();
f = std::move(g);
\end{lstlisting}

The parameter of such functions is considered a temporary on the call
site. They are actually not restricted to constructors or assignment
operators, and can be used in any function.

The semantics of such arguments must be interpreted as ``I {\em may}
steal your data''. The {\em may} is important here, as there is no
guarantee on the call site that the instance going through an
\code{std::move} is actually moved.

%-------------------------------------------------------------------------------
\subsection{\code{std::move}, the Subtleties}

Naming quality: \faStar\faStarO\faStarO\faStarO\faStarO. Would have
put zero stars if I could.

As a great example of ``naming things is hard'', \code{std::move} does
not move anything. It actually just marks the value as transferable,
by casting it to an rvalue-reference. See this actual implementation
exctracted from GCC 9:

\begin{lstlisting}
template<typename _Tp>
constexpr typename std::remove_reference<_Tp>::type&&
move(_Tp&& __t) noexcept
{
  return static_cast
  <
    typename std::remove_reference<_Tp>::type&&
  >(__t);
}
\end{lstlisting}

\begin{guideline}
  As a rule of thumb of its usage, consider two use cases:

  \begin{enumerate}
  \item If callee \underline{always} takes ownership of the value:
    prefer arguments passed by value.
  \item If callee \underline{may} take ownership of the value: use an
    rvalue reference.
  \end{enumerate}
\end{guideline}

\begin{lstlisting}
// "I need my own bar."
void foo_copy(bar b) { /* ... */ }

// "I may take ownership of b...
//  ...Unless I don't."
void foo_rvalue(bar&& b) { /* ... */ }

void baz()
{
  bar b1;
  // Valid, b1 is unchanged in baz.
  foo_copy(b1);

  // Valid, we can forget about b1 now.
  foo_copy(std::move(b1));

  bar b2;
  // Invalid, b2 is not an rvalue
  // foo_rvalue(b2);

  // Valid, but is b2 actually moved?
  foo_rvalue(std::move(b2));
}
\end{lstlisting}

%-------------------------------------------------------------------------------
\subsection{\code{std::forward} and Universal References}

When \code{\&\&} is used in association with a type to be deduced,
like in the function below:

\begin{lstlisting}
template<typename T>
void foo(T&& arg)
{
  /* … */
}
\end{lstlisting}

Then it refers neither to an rvalue-reference nor a reference. In this
context, it is called a \emph{universal reference}. Depending on how
the function is called then \code{T\&\&} will be either deduced to an
rvalue-reference, if the argument is an rvalue, or else the references
are collapsed and \code{T\&\&} becomes an lvalue-reference, just like
\code{T\&}.

\begin{lstlisting}
void bar()
{
  std::string s;

  // calls foo(std::string&)
  foo(s);

  // calls foo(std::string&&)
  foo(s + "abc");
}
\end{lstlisting}

Now how one should pass to another function an argument received as a
universal reference? If it is deduced as an rvalue-reference, then
\code{std::move()} should be used, otherwise it should be passed as
is.

\marginheader{<utility>}%
%
Fortunately \cpp11 provides \code{std::forward()} to do the check for
us:

\begin{lstlisting}
template<typename T>
void foo(T&& arg)
{
  // Pass as an rvalue-reference or by address, whichever fits.
  foobar(std::forward<T>(arg));
}
\end{lstlisting}
