\subsection{Functions}

Storing functions in a variable in \cpp{} used to be a pain. For
example, how would we implement \code{callable} such that the code
below works?

\begin{lstlisting}
void foo();
void bar(int arg);

struct some_object
{
  void some_method();
};

void test();
{
  std::vector<callable> scheduled;

  scheduled.push_back(&foo);
  // bar with arg = 5.
  scheduled.push_back(std::bind(&bar, 5));

  some_object c;
  scheduled.push_back(std::bind(&some_object::some_method, &c));

  // All stored functions can be called as if they were void().
  scheduled[0]();
  scheduled[1]();
  scheduled[2]();
}
\end{lstlisting}

Having a working type for all these use cases is a huge task, and
before \cpp11 our only hope were Boost.Function, some other
libraries\footnote{Search for FastDelegate, the Impossibly Fast C++
  Delegates, More Fasterest Delegates, and Ultimate Fast C++ Delegates
  II'. Some of them may not exist.}, or and homemade solution.

A binding like presented in Section~\ref{tuple} is a partial answer to
that but first, it does not even handle member functions, and two, it
is already \cpp11.

\bigskip

\marginheader{<utility>}%
%
The \code{std::function} type introduced in \cpp11, in combination
with \code{std::bind} \aref{bind}, is exactly what we
need to fix our previous example. The former is an object that represents a
function, and that can be called to invoke this function. It can be a
function object or a free function, everything works.

\begin{lstlisting}
void foo();
void bar(int arg);

struct some_object
{
  void some_method();
};

void test();
{
  std::vector<@\emcode{std::function<void()>}@> scheduled;

  scheduled.push_back(&foo);
  // bar with arg = 5.
  scheduled.push_back(std::bind(&bar, 5));

  some_object c;
  scheduled.push_back(std::bind(&some_object::some_method, &c));

  // All stored functions can be called as if they were void().
  scheduled[0]();
  scheduled[1]();
  scheduled[2]();
}
\end{lstlisting}

As far as can tell there is only one downside to \code{std::function},
it is that every call begins with a test checking if a function is
set. If we care about performance, it can be an issue\footnote{This
  test is used to throw an \code{std::bad\_function\_call} if the
  invocation is done on an empty instance. If you wonder why
  \code{std::function} does not reference by default a function that
  throws the exception, such that no test is done and the exception is
  still thrown when an empty function is invoked, know that I wonder
  too. If you have insight about it, I would love to know.}.
