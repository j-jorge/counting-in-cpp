In the previous sections we saw many nice features from \cpp11. There
are actually many more! As I did not use all of them, here come a
short description for the missing ones.

\subsection{Regular Expressions}

\marginheader{\cppheader{regex}}%
%
Regular expressions are now first class citizen in the STL. We can
finally easily write an HTML parser in a few lines
of \cpp{}\footnote{Reference for the
joke: \url{https://stackoverflow.com/a/1732454/1171783}. If you don't
follow this link, just know that one cannot correctly parse random
HTML with a regular expression.}!

Interestingly, the format for the expressions can be of almost any
existing syntax: ECMAScript, grep, awk…

\subsection{Explicit Conversion Operators}

Conversion operators can now be declared as explicit, to avoid
unexpected conversions:

\begin{lstlisting}
// This struct wraps an 8 bits non-negative integer value.
struct custom_uint8 { /* … */ };

// This struct wraps a 64 bits signed integer value.
struct custom_int64 { /* … */ };

// This struct wraps a 32 bits integer value.
struct custom_int32
{
  // This operator is explicit, as information may be lost when
  // truncating to 8 bits. Thus we don't want it to happen silently.
  @\emcode{explicit}@ operator custom_uint8() const
  {
    return custom_uint8(m_value);
  }

  // This operator is implicit, as all signed 32 bits values can be
  // represented in a signed 64 bits.
  operator custom_int64() const
  {
    return custom_int64(m_value);
  }

private:
  int32_t m_value;
};

void foo()
{
  custom_uint8 i8;
  custom_int32 i32;
  custom_int64 i64;

  // This is ok thanks to the implicit conversion operator.
  i64 = i32;

  // This won't work since there is an implicit conversion.
  i8 = i32;

  // Here the conversion is explicit, so it will work.
  i8 = static_cast<custom_uint8>(i32);
}
\end{lstlisting}

\subsection{Unrestricted Unions}

Unions can now contain non-POD members. I do not know where it is
useful but it can be done.

\subsection{String Literals}

New string literals have been introduced to represent UTF-8, UTF-16,
UTF-32 strings, as well as raw strings:

\begin{lstlisting}
const char* utf_8 = @\emcode{u8}@"I'm UTF-8.";
const char* utf_16 = @\emcode{u}@"I'm UTF-16.";
const char* utf_32 = @\emcode{U}@"I'm UTF-32.";

const char* raw = @\emcode{R"\_(}@This text is stored as is,
line breaks included,
  spaces included.
It is thus a four-lines text.@\emcode{)\_}@";
\end{lstlisting}

%-------------------------------------------------------------------------------
\subsection{User Defined Literals}

Have you ever mixed some units like in the declartion of \code{m} below?

\begin{lstlisting}
struct mass
{
  mass(float kg)
    : kilograms(kg)
  {}

  float kilograms;
};

void foo()
{
  // A mass of 2000 grams.
  mass m(2000);

  // Oops, mass() takes the value in kilograms :(
}
\end{lstlisting}

If you have already made this mistake, or just if you use this kind of
code, you may be happy to learn that you can now use custom suffixes
for literals, which may help to explicit the unit.

\begin{lstlisting}
// We define the 'g' suffix for float numbers, to represent grams.
mass @\emcode{operator"" \_g}@(float v)
{
  return mass(v / 1000);
}

void foo()
{
  // A mass of 2000 grams.
  mass m = 2000@\emcode{\_g}@;

  // m.kilograms is equal to 2, everything is fine.
}
\end{lstlisting}

%-------------------------------------------------------------------------------
\subsection{Very Long Integers}

Sometimes an \code{int} is a bit short to store large values. Then we
switch to \code{long int}, but sometimes it is still not enough. Now
with \cpp11 we can also switch to \code{long long int}\footnote{Also
available as \code{long int long}, or \code{int long long},
or \code{signed long int long}, or \code {long int signed long}, etc.}.

This type is guaranteed to be made of at least 64 bits,
while \code{int} and \code{long int} are made of at least 16 and 32
bits respectively.

Actually, I doubt I will ever use it, because when I need to store a
value in at least $n$ bits, I usually choose \code{int$n$\_t} type
from \cppheader{cstdint}.

%-------------------------------------------------------------------------------
\subsection{Size of Members}

Computing the size of a struct member in the old days of \cpp{}
required to use an instance of the struct, like in:

\begin{lstlisting}
struct foo
{
  int m;
};

int main()
{
  // There is no way to apply sizeof directly to m, so we "create"
  // an instance just to get the member. Fortunately the expression
  // on which sizeof is applied is not evaluated.
  printf("%ld\n", sizeof(foo().m));

  return 0;
}
\end{lstlisting}

This is doable for simple structs but clearly becomes difficult if
the constructor needs arguments.

In \cpp11 there is now a syntax to get the size of a member:

\begin{lstlisting}
int main()
{
  printf("%ld\n", sizeof(@\emcode{foo::m}@));
  return 0;
}
\end{lstlisting}

%-------------------------------------------------------------------------------
\subsection{Type Alignment}

If you are the kind of person who write allocators or who use
placement new in a buffer, then you will be happy to learn
about \code{alignas(T)}, which sets a type alignment to the
value \code{T}, if it is an integer, or to match the alignment of the
type \code{T} otherwise. Similarly, \code{alignof(T)} is the
alignement of the type \code{T}.

\begin{lstlisting}
char buffer[1024];
int consumed = 0;

template<typename T>
T* allocate()
{
  const int shift = consumed % @\emcode{alignof(T)}@;
  const int offset = (shift == 0) ? 0 : (@\emcode{alignof(T)}@ - shift);

  T* const result = new (buffer + consumed + offset) T();
  consumed += offset + sizeof(T);

  return result;
}
\end{lstlisting}

%-------------------------------------------------------------------------------
\subsection{Attributes}

Maybe have you already seen the \code{\_\_attribute\_\_} token in
some \cpp{} code? This is used to annotate the code with hints or
directives about some code. Unfortunately it is a compiler extension,
so it was not usable in portable code.

Starting with \cpp11, we now have a standard way to specify attributes
with the \code{[[some\_attribute]]} syntax. Their semantic is still
implementation-defined but at least it is syntaxically portable.

One of the only two well defined attributes in the \cpp11
standard is \code{[[noreturn]]}, which tells that a function never
returns to the caller:

\begin{lstlisting}
@\emcode{[[noreturn]]}@ void fatal_error(const char* m)
{
  printf("%s\n", m);
  exit(1);
}
\end{lstlisting}

The second one is \code{[[carries\_dependency]]} and I won't talk
about it.
