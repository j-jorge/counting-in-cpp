%-------------------------------------------------------------------------------
\chapter{Preface}

%-------------------------------------------------------------------------------
\section{About This Book}
The goal of this document is to list most, if not all, features
introduced in the \cpp language since the first well-known deep update
in the language, also known as \cpp11.

These features are presented following a systematic format where the
old way is reminded to the reader, with a short explanation of why it
may have been problematic or inefficient, then the new way of doing
things is presented.

Some parts of the language will probably be silenced, mostly the parts
for which I don't know much. Some parts will be presented with harsh
critics, maybe even just based on my experience. Am I an authority on
the subject? Probably not. Am I experimented? I tend to think I am,
but you are the judge. Anyway, it is ok if you dismiss these critics,
just know that there are pros and cons for anything.

This book won't go into the details and subtleties of any features,
nor into compiler specific stuff. The reason being mostly time (I have
a limited time to dedicate to this book) and space (it is already
large enough). The reader is invited to satisfy his curiosity and
complete is knowledge by reading other material. For example, the
website \url{https://en.cppreference.com} has everything you need to
know about anything from the language.

Remember the basis of learning: read to acquire the experience of
other, practice to build your own experience.

%-------------------------------------------------------------------------------
\section{About The Author}

I'm just some guy who write stuff not to think about the
meaninglessness of existence. My intent is not to compete nor to show
of, but maybe are you wondering if I am relevant on the topics from
this book? See by yourself.

I write code as a hobby since 1994, and professionally since 2005. I
have at least 600000 lines of \cpp behind me, just counting alive
lines of code from past projects I could find. I did not go into the
history of each project, so this does not include lines that have been
overwritten.

I also have coded on Java projects, some Pascal and Delphi ones,
Visual Basic too, BASIC a long time ago, on a Commodore 128 and later
under DOS. I did also a bit of Objective-C.

Actually, a good share of this code was crap.

Some code did end up well, though. One project I am proud of is a
mobile game written in \cpp, which was played by X people every day
during more than Y years. Apart from that, I also took part on
projects that were struggling to start and brought them into a viable
product. So I guess I made stuff that do not suck.

When I code I tend to think about long term and architecture. I try
not to take any shortcut and to answer the problem without expecting
to solve the future. I code in small boxes, many, with the intent that
they can be broken, removed, replaced, without changing everything.
