\subsection{Binary Suffix}

\problemtitle

In \cpp11 and before, we used to declare bit patterns as hexadecimal
values, or shifts. For example, if one wanted to access the flags
stored in the packed fields from the image descriptor of a GIF 87a
file \cite{gif87a}, the code could have been similar to:

\begin{lstlisting}
void parse_image_descriptor_fields(std::uint8_t packed_fields)
{
  // Bitmask with shifts.
  const bool use_local_color_table = packed_fields & (1 << 7);

  // Bitmasks with hexadecimal values.
  const bool is_interlaced = packed_fields & 0x40;
  const std::int8_t bits_for_color_palette_size_minus_one =
    packed_fields & 0x07;
}
\end{lstlisting}

While not really obscure, this syntax requires some effort from the
reader, and a good understanding of the binary representation of
numbers. What if we could directly write the binary?

\solutiontitle

That's possible with \cpp14. Just like the 0x prefix introduces an
hexadecimal value, 0b introduces a binary value:

\begin{lstlisting}
void parse_image_descriptor_fields(std::uint8_t packed_fields)
{
  // Bitmask with binary literal.
  const bool use_local_color_table = packed_fields & @\emcode{0b10000000}@;

  // Note that we can use the number separator @\aref{number-separator}@ to split the mask.
  const bool is_interlaced = packed_fields & @\emcode{0b0100'0000}@;
  const std::int8_t bits_for_color_palette_size_minus_one =
    packed_fields & @\emcode{0b0111}@;
}
\end{lstlisting}
