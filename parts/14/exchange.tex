\subsection{\code{std::exchange}}%
%
\problemtitle%
%
\marginheader{<utility>}%
%
If you are reading this, you certainly had to write a move constructor
at some point. When doing so, the programmer has to ``steal'' the
resources from an instance, and leave the aforementioned instance in a
valid state. In the example from section \ref{sec:move-constructor},
copied below for convenience, the resource from \code{that} is
transferred to \code{this}, then the field from \code{that} is set to
\code{nullptr}.

\begin{lstlisting}
struct foo
{
  foo(foo&& that)
    : m_resource(that.m_resource)
  {
    that.m_resource = nullptr;
  }
};
\end{lstlisting}

As a move constructor grows with the number of fields, the distance
between the transfer of the resource in the constructor's initializer
list and the reset in the constructor's body increases. This blurs the
relationship between the two statements, which actually form a single
semantic operation. This is also error-prone, as one may easily forget
one of the two instructions.

\solutiontitle

Thankfully \cpp14 introduces a nice small utility with
\code{std::exchange} to solve this issue.

\begin{lstlisting}
struct foo
{
  foo(foo&& that)
    // Assign that.m_resource to m_resource and set that.m_resource to
    // nullptr, in a single statement.
    : m_resource(@\emcode{std::exchange}@(that.m_resource, nullptr))
  {}
};
\end{lstlisting}

The \code{std::exchange} function assigns its second parameter to its
first parameter, then returns the previous value of its first
parameter. Yes, there is no actual exchange here… As with
\code{std::move}, the name does not carry the meaning very well, but
at least it solves a problem.
