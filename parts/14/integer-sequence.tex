\subsection{Compile-time Integer Sequences}%
\label{sec:integer-sequence}

\problemtitle
%
\marginheader{<utility>}%
%
If you start playing with parameter packs \aref{parameter-pack} and
tuples \aref{sec:tuple}, you will soon encounter a case where you will
need to so something ``for each item in the tuple''.

A typical use case consists in calling a function with its arguments
taken from a tuple, something we already tried in the previous pages
\aref{example:argument-binding}, in an argument-bindings example. Go
check, and see how much code we wrote to store an sequence of integers
in a type.

\solutiontitle

Starting with \cpp14, this code can be replaced by
\code{std::integer\_sequence<typename T, T... Ints>} and
\code{std::make\_integer\_sequence<typename T, T N>}. Writing a
function that runs a function with its arguments taken from a tuple is
approximately as simple as this:

\begin{lstlisting}
// The call_helper will help us to get the elements of a tuple, because we
// cannot get them by type due to possibly duplicate types.
template<typename IntegerSequence>
struct call_helper;

template<unsigned... I>
struct call_helper<@\emcode{std::integer\_sequence}@<std::size_t, I...>>
{
  template<typename F, typename... Args>
  static void call(F&& function, std::tuple<Args...>&& arguments)
  {
    // Here we unpack I... to get the arguments from the tuple.
    // See @\aref{variadic-template}@.
    //
    // Note that it does not work with member functions, as they require another
    // call syntax. This is still left as an exercise for the reader.
    function(std::get<I>(arguments)...);
  }
};

template<typename F, typename... Args>
decltype(auto) apply(F&& function, std::tuple<Args...>&& arguments)
{
  return call_helper
    <
      @\emcode{std::make\_integer\_sequence}@<std::size_t, sizeof...(Args)>
    >::call(std::forward<F>(function), std::move(arguments));
}
\end{lstlisting}

Isn't it clean? Hold on, we'll soon be able to remove this code too
\aref{sec:apply}\footnote{And it will be for the best, as this example
code misses many features, like calling member functions, accepting
a const tuple as argument, and probably more.}!
