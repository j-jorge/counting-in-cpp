\subsection{\code{[[deprecated]]} Attribute}

As code evolves, some parts naturally become obsolete. Most of the time,
obsolete code can be simply removed, but when it is published as a
public API, great care must be taken in order not to break client
code.

In such situation, the publisher would typically provide both the old
and the new API for a period of time long enough for the clients to
migrate. Which they may do only if they know which parts are obsolete!

So the issue here is actually to pass the information to the
developers that the code is going to be deleted. How can we do that?
Maybe by adding a paragraph in the release notes? I've heard that some
people read them. Or maybe a warning in the header? Here we are going
to rely on compiler-specific features, and the warning will be
displayed as soon as the file is included, even if the deprecated code
is not called. What about printing the deprecation message at run
time, when the code is executed\footnote{Please don't.}?

\bigskip

Starting with \cpp14, the \code{[[deprecated]]} attribute can be used
to target a specific symbol for deprecation, with an explicit message
that will be displayed to the user, if and only if the symbol is used.

\begin{lstlisting}
[[@\emcode{deprecated}@("Use the button class instead.")]]
class toggle_button
{
  /* … */
};

class button
{
public:
  [[@\emcode{deprecated}@("Use set_font(const font&) instead.")]]
  void set_font(const std::string& font_name);
};

void create_buttons()
{
  // This line will trigger a deprecation warning.
  toggle_button toggle;

  // This one doesn't.
  button b;

  // But calling the deprecated method will trigger the warning.
  b.set_font("sans");
}
\end{lstlisting}

