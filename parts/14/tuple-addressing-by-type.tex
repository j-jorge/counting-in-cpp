\subsection{Tuple Addressing By Type}

\Cpp11 introduced tuples \aref{sec:tuple} which are quite handy if
you are too lazy to write a struct, or if you come from toy languages
like Python or JavaScript.

In order to access an element in a tuple, one must know the index of
the element to pass to \code{std::get}. This is a bit problematic if
we add an entry in the tuple; then we have to update all
accesses to match the new indices\footnote{Something that would have
not happen if only we used a struct. Let that sink in!}.

In a metaprogramming context this can also be very limiting. Check for
example this map allowing to store values of different types with a
key (this is just an aggregate of maps whose key and value types are
passed as template arguments):

\begin{lstlisting}
#include <unordered_map>

template<typename K, typename... V>
class heterogeneous_map
{
public:
  template<typename T>
  void set(const K& k, const T& v)
  {
    // What should I use for the index?
    std::get</* index of T in V... */>(m_maps)[k] = v;
  }

private:
  std::tuple<std::unordered_map<K, V>...> m_maps;
};
\end{lstlisting}

In \cpp11 one would have to write a helper to find the index of a type
in a parameter pack, which is not very difficult but still a lot of
work compared with doing nothing.

\bigskip

In \cpp11, The \code{std::get} function has been extended and now
accepts a type as its template arguments, allowing to address a tuple
element by its type. This only works if the tuple declaration has a
single entry of the given type, which is good enough:

\begin{lstlisting}
#include <unordered_map>
#include <tuple>

template<typename K, typename... V>
class heterogeneous_map
{
public:
  template<typename T>
  void set(const K& k, const T& v)
  {
    // No index needed!
    std::get<@\emcode{std::unordered\_map<K, T>}@>(m_maps)[k] = v;
  }

private:
  std::tuple<std::unordered_map<K, V>...> m_maps;
};
\end{lstlisting}
