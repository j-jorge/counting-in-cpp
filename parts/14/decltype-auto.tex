\subsection{\code{decltype(auto)}}

\problemtitle

We saw in \aref{sec:return-type-deduction} that we can now omit the
return type of a function. So can you tell why the assert won't pass
in the code below?

\begin{lstlisting}
#include <type_traits>
#include <utility>

template<typename F, typename... Args>
auto invoke(F&& f, Args&&... args)
{
  return f(std::forward<Args>(args)...);
}

int& at(int* v, int i)
{
  return v[i];
}

// Note that no call occurs here since decltype(expression) produces
// the the type at compile-time, _as-if_ the expression was
// evaluated. So it is safe to use nullptr in the arguments.
static_assert
  (std::is_same
   <
     decltype(invoke(&at, nullptr, 42)),
     decltype(at(nullptr, 42))
   >::value,
   "");
\end{lstlisting}

If you ever encounter one of the few use cases where \code{auto} it is
actually useful, you may be surprised that the deduced type does not
match your expectation. In the above code one could read the return
type of \code{invoke} as ``whatever is returned by
\code{f}''. Unfortunately, \code{auto} does not work like that.

The \code{auto} keyword does not take into account the reference nor
the constness of the expression. That means that the type of
\code{auto\_value} in the code below is actually \code{int}, not
\code{const int\&} like \code{reference}.

\begin{lstlisting}
#include <type_traits>

int i;
const int& reference = i;
auto auto_value = reference;
static_assert(std::is_same<int, decltype(auto_value)>::value, "");
\end{lstlisting}

\solutiontitle

Most of the time, \code{auto} can be replaced by a more explicit type,
but in the case of the \code{invoke} function the explicit version
requires to repeat the whole body (see
\aref{sec:return-type-deduction}). As a workaround, \cpp14 introduces
the \code{decltype(auto)} specifier, which keeps all the properties of
the type of the statement from which the type is deduced:

\begin{lstlisting}
#include <type_traits>
#include <utility>

template<typename F, typename... Args>
@\emcode{decltype(auto)}@ invoke(F&& f, Args&&... args)
{
  return f(std::forward<Args>(args)...);
}

int& at(int* v, int i)
{
  return v[i];
}

// It works!
static_assert
  (std::is_same
   <
     decltype(invoke(&at, nullptr, 42)),
     decltype(at(nullptr, 42))
   >::value,
   "");
\end{lstlisting}

As usual, \code{decltype(auto)} can be used anywhere a type is
required, as long as there is a way for the compiler to deduce the
type.

As usual, there is no good reason to prefer that when a more explicit
type can be used.
