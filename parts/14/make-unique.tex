\subsection{\code{make\_unique}}

\problemtitle

%
\marginheader{<memory>}%
%
\Cpp11 introduced smart pointers in the standard library, with
\code{std::unique\_ptr} and \code{std::shared\_ptr}. For the latter
the idiomatic creation code would use \code{std::make\_shared}, which
would have the benefit to create storage for the control block and the
actual object into a single allocation. On the other hand,
\code{std::unique\_ptr} has no control block, so it made sense at the
time not to have specialized creation functions for it.

There is another use case though, where such function can help:

\begin{lstlisting}
create_form
  (std::unique_ptr<widget>(new label("Everything is fine.")),
   std::unique_ptr<widget>(new icon("resources/alert.png")));
\end{lstlisting}

See, since there is no guarantee on the order of evaluation of
function arguments, the compiler is free to sequence them as follows:

\begin{enumerate}
\item \code{new label(…)}
\item \code{new icon(…)}
\item \code{std::unique\_ptr(…)}
\item \code{std::unique\_ptr(…)}
\end{enumerate}

Now, if the constructor of \code{icon} throws an exception, then the
\code{label} won't ever be destroyed, nor its memory released during
the program execution.

\solutiontitle

This can be solved by wrapping the allocation in a function, a
function provided by the standard library for example.

\begin{lstlisting}
create_form
  (std::make_unique<label>("Everything is fine."),
   std::make_unique<icon>("resources/alert.png"));
\end{lstlisting}

With this code, the order of evaluation does not matter. If the first
argument is built before the second, then its memory already managed
by the smart pointer, so even if the construction of the second
argument throws, the object will be destroyed and the memory will be
released. The reasoning is the same if the second argument is built
before the first.

As a bonus, this is also less verbose.
