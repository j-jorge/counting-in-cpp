\subsection{Return Type Deduction}
\label{sec:return-type-deduction}

\problemtitle

The \code{decltype} specifier was added in \cpp11 to identify the type
of an expression \aref{decltype}. One of its typical use cases is the
declaration of the return type in a template function whose result
type must be deduced from the template arguments.

\begin{lstlisting}
// This function is a proxy to call a given function with the given
// arguments. Its return type depends on the provided function.
template<typename F, typename... Args>
auto invoke(F&& f, Args&&... args) -> decltype(f(std::forward<Args>(args)...))
{
  return f(std::forward<Args>(args)...);
}
\end{lstlisting}

Note how the expression passed to \code{decltype} is exactly the body
of the function.

\solutiontitle

To avoid this verbosity, \cpp14 allows to omit the trailing return
type to let the compiler deduce the type from the function's body.

\begin{lstlisting}
template<typename F, typename... Args>
auto invoke(F&& f, Args&&... args)
{
  return f(std::forward<Args>(args)...);
}
\end{lstlisting}

An excellent use case for this feature is a function returning a
lambda \aref{sec:lambda}. Have you ever tried to write one?

\begin{lstlisting}
auto make_comparator(int v) -> /* what should I write here? */
{
  return [v](int c) -> bool
  {
    return v == c;
  };
}
\end{lstlisting}

As far as I know there is no way to write such a function in
\cpp11. In \cpp14, though, we can simply drop the return type and let
the compiler deduce it from the body.

\begin{guideline}
One could be tempted to always omit the return type under the false
assumption that less code is more readable. Don't do that.

Whenever we omit the return type, the reader has to parse the whole
body to find the actual type. This is needlessly complex and makes
painful code reviews. What should be more readable actually requires
to process more code.

Implied information are bad in programming. Don't follow the ``modern
\cpp{}'' trend which pushes for using every new feature everywhere;
use them only when you need them.
\end{guideline}
