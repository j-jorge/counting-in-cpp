\subsection{Generic Lambdas}

The introduction of lambdas \ref{sec:lambda} in \cpp11 unlocked many
possibilities for the programmer. Despite that, some use cases are
still a bit difficult to tackle. Take for example a unique predicate
that should be applied to two collections of different types, as in
the example below:

\begin{lstlisting}
#include <vector>
#include <algorithm>

bool same_count_by_sign
(const std::vector<int>& ints, const std::vector<float>& floats)
{
  if (ints.size() != floats.size())
    return false;

  const auto non_positive_integer =
    [](int v) -> bool
    {
      return v <= 0;
    };
  const auto non_positive_float =
    [](float v) -> bool
    {
      // This is the same body as above. Do I really need two lambdas?
      return v <= 0;
    };

  const std::size_t count_ints =
    std::count_if(ints.begin(), ints.end(), non_positive_integer);
  const std::size_t count_floats =
    std::count_if(floats.begin(), floats.end(), non_positive_float);

  return count_ints == count_floats;
}
\end{lstlisting}

It is a bit disappointing to have two lambdas with the same
body. Intuitively, one would want to templatize it. Unfortunately
template lambdas do not exist, so we have to switch back to the
pre-\cpp11 way with free functions or function objects.

\begin{lstlisting}
#include <vector>
#include <algorithm>

namespace
{
  // There we go, a simple predicate function declared far from its use.
  template<typename T>
  bool non_positive(T v)
  {
    return v <= 0;
  }
}

bool same_count_by_sign
(const std::vector<int>& ints, const std::vector<float>& floats)
{
  if (ints.size() != floats.size())
    return false;

  const std::size_t count_ints =
    std::count_if(ints.begin(), ints.end(), &non_positive<int>);
  const std::size_t count_floats =
    std::count_if(floats.begin(), floats.end(), &non_positive<float>);

  return count_ints == count_floats;
}
\end{lstlisting}

It works, for sure, but it is not what we expect. Moreover, as soon as
a variable has to be captured, we are back to the extremely verbose
functor objects.

\bigskip

Enters \cpp14 and the generic lambdas. By declaring the lambda's
arguments as \code{auto}, we can declare what is effectively the
equivalent of a template lambda.

\begin{lstlisting}
#include <vector>
#include <algorithm>

bool same_count_by_sign
(const std::vector<int>& ints, const std::vector<float>& floats)
{
  if (ints.size() != floats.size())
    return false;

  const auto non_positive =
    [](@\emcode{auto}@ v) -> bool
    {
      return v <= 0;
    };

  // It's the same predicate in both calls.
  const std::size_t count_ints =
    std::count_if(ints.begin(), ints.end(), non_positive);
  const std::size_t count_floats =
    std::count_if(floats.begin(), floats.end(), non_positive);

  return count_ints == count_floats;
}
\end{lstlisting}

We saw in \ref{sec:lambda} what was the equivalent function object of
a lambda. Here the corresponding transformation, given a lambda
declaration like

\begin{lstlisting}
[](auto a1, …, auto an ) -> /* return type */ { /* statements */ }
\end{lstlisting}

would be something like

\begin{lstlisting}
struct something
{
  template<typename T1, …, typename Tn>
  /* return type */ operator()(T1 a1, …, Tn an) const { /* statements */ }
};
\end{lstlisting}

Note that the type of the arguments are not related. Each use of
\code{auto} here corresponds to a single template parameter.
