\subsection{Relaxed \code{constexpr}}

The introduction of \code{constexpr} in \cpp11 \aref{sec:constexpr}
greatly simplified some compile-time complex computations, even though
they were some limitations. The most frustrating of which was
certainly the constraint that \code{constexpr} functions should
contain a single statement (a \code{return} statement). In practice
this means no variable, no branch or control flow other than the
ternary \code{?:} operator and using recursion for any loop-based
algorithm. See for example this implementation of \code{popcount()} we
wrote in section \aref{sec:constexpr}:

\lstinputlisting{examples/constexpr/constexpr-11.cpp}

In \cpp14, these restrictions are lifted and we can implement this
function in a less convoluted way\footnote{Not that recursion is
inherently convoluted. Some algorithms work very well when
implemented in a recursive way, both regarding the readability and the
performance, but even though any programmer worth is money must be
know the practice I would argue that most code, i.e. the code we are
familiar with, is non-recursive.}:

\lstinputlisting{examples/constexpr/constexpr-14.cpp}

A subtle change in the transition from \cpp11 to \cpp14 is that
\code{constexpr} used on a member function does not implies
\code{const} anymore. For example, this code would not compile in
\cpp11 but does in \cpp14:

\begin{lstlisting}
struct s
{
  int value;

  constexpr int increment(int v)
  {
    // Can't do that in C++11 as the function signature is actually
    //
    //   constexpr int increment(int) const.
    //
    // Okay in C++14, where the function signature is as declared.
    return value += v;
  }
};
\end{lstlisting}

This can come as a surprise when switching to a newer standard since
calling a \code{constexpr}-only member function on a \code{const}
instance will suddenly trigger errors like ``passing ‘const x’ as
‘this’ argument discards qualifiers''.
