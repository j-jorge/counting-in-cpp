\documentclass{book}

\usepackage[english]{babel}
\usepackage[normalem]{ulem}
\usepackage{cite}
\usepackage{enumitem}
\usepackage[dvipsnames]{xcolor}
\usepackage{fontawesome}
\usepackage{times}
\usepackage[marginparwidth=2.5cm]{geometry}
\usepackage{graphicx}
\usepackage[colorlinks=true,allcolors=Blue]{hyperref}
\usepackage{bookmark}
\usepackage[T1]{fontenc}
\usepackage{marginnote}
\usepackage{mdframed}
\usepackage{sectsty}
\usepackage[thicklines]{cancel}
\usepackage[nottoc,numbib]{tocbibind}
\usepackage[parfill]{parskip}

% Write C++XX using the format from cppreference.com: capital C, no
% dash nor space.
\newcommand{\cpp}[1]{C++{#1}}
\newcommand{\Cpp}[1]{C++{#1}}

\newcommand{\cppheader}[1]{\textless #1\textgreater}

\newcommand{\aref}[1]{[\ref{#1}]}

\title{Counting from 98 to \cancel{20} 14 \\ in \Cpp{}}
\author{Julien Jorge}
\newcommand{\version}{@CMAKE_PROJECT_VERSION@}

\usepackage{couriers}
\usepackage{listings}
\usepackage{xcolor}

\definecolor{codehighlight}{rgb}{0.827,0.322,0.07}

\newcommand{\lstinputlistinghl}[3][]{%
  \lstinputlisting[%
    belowskip=0pt,
    lastline=\number\numexpr#2-1\relax,
    showlines=true,
    #1
  ]{#3}%
  \lstinputlisting[%
    aboveskip=0pt,
    belowskip=0pt,
    backgroundcolor=\color{codehighlight!15},
    firstline=#2,
    lastline=#2
  ]{#3}
  \lstinputlisting[%
    aboveskip=0pt,
    firstline=\number\numexpr#2+1\relax
  ]{#3}
}

\newcommand{\emcode}[1]{\bfseries\color{codehighlight}{#1}}

\lstset{
basicstyle=\ttfamily\small,
emphstyle=\bfseries\color{codehighlight},
escapechar=@,
extendedchars=\true,
inputencoding=utf8x,
language=c++,
literate={~} {$\sim$}{1},
rulecolor=\color{black!20},
showstringspaces=\false,
xleftmargin=5mm,
xrightmargin=5mm,
commentstyle=\itshape\color{OliveGreen},
stringstyle=\color{RawSienna}
}


\newcommand{\code}[1]{{\tt #1}}

\newcommand{\marginheader}[1]{%
  \marginnote{%
    \begin{center}%
      \includegraphics[width=0.5\linewidth]{assets/header-icon.pdf}

      \footnotesize #1
    \end{center}%
  }[-2.2em]%
}

\newenvironment{guideline}
               {
                 \begin{mdframed}[
                     topline=false,
                     rightline=false,
                     bottomline=false,
                     linewidth=1pt,
                     frametitle={Guideline}
                     ]
               }
               {
                 \end{mdframed}
               }

\newenvironment{pitfall}
               {
                 \begin{mdframed}[
                     topline=false,
                     rightline=false,
                     bottomline=false,
                     linewidth=1pt,
                     frametitle={It's a trap}
                     ]
               }
               {
                 \end{mdframed}
               }

\begin{document}
\allsectionsfont{\sffamily}

\frontmatter
\begin{titlepage}
  \pdfbookmark{Cover}{titlepage}

  \newgeometry{margin=3cm, noheadfoot, nomarginpar}
  \pagestyle{empty}
  \centering

  \makeatletter

  \vspace*{\fill}
  \vspace{-9cm}

  {\Huge\bfseries \@title}

  \vspace{1cm}
  Version \version

  \vfill

  \begin{minipage}{0.85\textwidth}
    \vfill

    \@author

    \href{mailto:julien.jorge@gmx.fr}{julien.jorge@gmx.fr}
  \end{minipage}%
  \hfill%
  \begin{minipage}{0.15\textwidth}%
    \raggedright
    \includegraphics[width=\linewidth]{assets/by-sa.pdf}
  \end{minipage}

  \makeatother
\end{titlepage}


\cleardoublepage
\pdfbookmark{\contentsname}{toc}
\tableofcontents

%-------------------------------------------------------------------------------
\chapter{Preface}
\renewcommand*\thesection{\arabic{section}}

%-------------------------------------------------------------------------------
\section{About this Book}

Once upon a time, as I was working on a large project with others
\cpp{} programmers, I was asked to set up a series of talks about the
language and especially about what has changed since the arrival of
\cpp11. It was in 2020.

So I started to write some slides with what seemed to be the key
features from \cpp11, and quite soon I had to face the truth: it is a
lot of content, and there was three additional major updates in the
language that should be covered too.

In the end I did not do the talks. However, I kept working on the
slides, until eventually I decided to switch the format. It is
probably too much material for a talk, but what about a small book?

Hence this document.

\bigskip

The goal is to list many, if not all, essential features introduced in
the \cpp{} language since its first well-known deep update, known as
\cpp11, up to the most recent version of the standard, which is \cpp20
by the time I am writing this.

These features are for the most part presented following a format
where the pre-\cpp11 way is reminded to the reader, with a short
explanation of why it may have been problematic or inefficient, then
the new way of doing things is presented.

Some parts of the language are silenced, mostly the ones for which I
don't know much, other parts are more thoroughly presented. In any
case, this book won't go into the details and subtleties of any
feature, nor into compiler-specific stuff. The reason being mostly
time (as far as I can tell I have a limited amount of that in my life)
and space (the book is already large enough). The reader is invited to
satisfy his curiosity and complete is knowledge by reading other
material. For example, the website \url{https://en.cppreference.com}
has everything we need to know about any feature of the language.

Be advised that some critics may suddenly appear in these pages, about
the language or the programmers. Keep in mind that those are personal
opinions and may change suddenly!

Finally, a basic knowledge of the language is preferable for the
reader to enjoy this book, as some notions will be used without being
explained.

\bigskip

While this book is about \cpp, one should remember that \cpp{} itself
is just a tool in the programmer's toolbox. If you are focusing on
learning \cpp{} to become a programmer, a good programmer, I would
suggest to rethink your plan and learn programming on a larger scale:
computer architecture, algorithms, data structures, project
management, packaging, dependency management, coding style, reviews,
testing…  There are many aspects to be familiar with in the daily life
of a programmer, keep some place for them.

As a good starting point, every programmer should read Code Complete
\cite{code-complete}. This book goes in detail in all aspects of
software development, backed up by data coming from over decades of
real-life projects, so if you read it you will also gain part of the
knowledge from these people who tried, failed, and succeeded before
you.

Clean Code \cite{clean-code} is a good second book to read, even
though I would not approve all suggestions. For example, it pushes for
intensive factorization and the use of object-oriented programming
everywhere, which are rather things I have painfully learnt to use
parsimoniously. Still, the book is a reference in software
development, so you should read it at least to make yourself an
opinion and to know what is going on in the business.

\bigskip

Finally, remember that if reading is acquiring the experience of
others, practicing is building our own experience. So I strongly
recommend to find or start a side project, maybe even a rewrite of
existing tools, just to try and get a grasp of the potential underlying
complexity.

%-------------------------------------------------------------------------------
\section{About the Author}

Should one take this book's content at face? Is the author legit? It
is normal to question the legitimacy of who pretend to give advice.
So in order to help you gauging the credibility of this book, I think
it is important to tell a bit about me.

\bigskip

First of all, I read a lot of code. I read code almost everyday, on
GitHub, on blogs, on StackOverflow, Reddit, and on other forums. I
read code written by me or others, from my personal projects, from my
employer, or from random projects I find on the Internet. All this
code is displayed either on my laptop, or in a terminal connected to a
remote server, or on my phone. Reading is undoubtedly the main part of
my programming activities.

On the productive side, I write code as a hobby since 1994, and
professionally since 2005. I have at least half a million of \cpp{}
behind me, just counting the lines of code that survived in past
projects I could find.

I also have coded a lot of Bash, a good share of Java, a bit of HTML
and a bit of JavaScript. Additionally, in a more anecdotal way, I
coded some C\# programs, some ActionScript, some Pascal and Delphi
ones, a small compiler in Eiffel, some pet projects in Visual Basic
too, BASIC a long time ago, on a Commodore 128 and later under DOS. I
also did a bit of Objective-C.

Let's face it though, a good share of this code was crap.

\bigskip

Some code did end up well nonetheless. One project I am proud of is a
mobile game written in \cpp, which was played by more than 500'000
people every day during more than three years. Aside from that, I also
took part in projects that were struggling to start and brought them
into a viable product. So I guess I made stuff that does not suck.

When I code I tend to think about long term and architecture. I try
not to take any shortcut and to answer the problem without attempting
to solve the future. I code in small boxes, many, with the intent that
they can be broken, removed, replaced, without changing everything. It
wasn't always like that but that's how I work today.

Finally, I am certainly not the type to rush for the new thing. I like
tools and practices that have been well tested, so you probably won't
hear me telling you to use this new thing from \cpp{42} because it's
new and it will show that you are modern and blah blah blah.

\bigskip

Convinced? Anyway, I hope you will find something useful in this book :)

%-------------------------------------------------------------------------------
\section{Where to find this book}

This book is available as a PDF, always synchronized with the latest
changes, at
\url{https://github.com/j-jorge/counting-in-cpp/releases/download/continuous/counting-in-c++-wip.pdf}.

It is also provided as a website at
\url{https://julien.jorge.st/counting-in-cpp/}, in good old HTML format.

Finally, the \LaTeX{} source files are stored in a Git repository at
\url{https://github.com/j-jorge/counting-in-cpp}.

%-------------------------------------------------------------------------------
\section{License}
This work of art is licensed under a Creative Commons
Attribution-ShareAlike 4.0 International License. See
Appendix~\ref{license} for the full license text.

\marginheader{This one.}%
%
The header file icon displayed in the margins, like the one on the
side of this paragraph, is based on the New Document icon from the
Tango Icon Theme \cite{tango-icon-theme}. The original icon is in the
public domain and in order to respect the intent of the source the
variation made for this book, as in the SVG file available in the
repository containing the sources of this book, is also released in
the public domain.

\renewcommand*\thesection{\arabic{chapter}.\arabic{section}}



\mainmatter
%-------------------------------------------------------------------------------
\chapter{The \Cpp{} Language and its Community}

\Cpp{} is a quite old programming language: it was created by Bjarne
Stroustrup in 1982. It was initially thought as an improved C.

The first official specifications were \emph{The C++ Programming
  Language} \cite{the-cpp-programming-language-1st}. This book was
then updated multiple times \cite{the-cpp-programming-language-2nd},
\cite{the-cpp-programming-language-3rd},
\cite{the-cpp-programming-language-se}, and
\cite{the-cpp-programming-language-4th}.

Starting from 1998, the official specifications are described in
\emph{The Standard}.

%-------------------------------------------------------------------------------
\section{The Standard}

The standard from 1998 set the ground for a new era of compilers by
defining how \cpp{} programs should behave, and by describing the
content of the standard library as well as the constraints on its
implementation.

The standard is defined by {\em The ISO C++ Committee}, i.e. people
from the industry: Google, HP, Intel, Oracle, and many more.

It is worth noting that no code is provided by the committee. The
standard just describes the language, then independent developers
(mostly compiler vendors) provide the implementation. Theoretically,
developers can switch easily from one compiler to the other. Clients
are not tied to the vendor's compiler anymore.

As I write there are six revisions for this document, informally
identified by the year they came out:

\begin{itemize}
\item {\bf \Cpp98} defined the core features: the syntax, the memory
  model, templates, namespaces… And the Standard Template Library
  (STL): \code{std::vector}, \code{std::string}, \code{std::map}…
\item {\bf \Cpp03} fixed some wording and inconsistencies.
\item {\bf \Cpp11} the beginning of what is called \emph{modern
  \cpp}. Initially expected during the 00's, the committee had to kiss
  goodbye to some awaited features. Better done than perfect.
\item {\bf \Cpp14} mostly bug fixes but also nice features:
  e.g. variable templates.
\item {\bf \Cpp17} nice new features: fold expressions, \code{if
  constexpr}, copy elision, \code{std::optional}, \code{
  string\_view}…
\item {\bf \Cpp20} \code{std::span}, concepts, modules… Also: your
  compiler still doesn't fully support \cpp17.
\end{itemize}

%-------------------------------------------------------------------------------
\section{The Community}

The \cpp{} community is made of humans, so it is naturally composed of
a lot of good things and many problems. Sometimes simultaneously. And
since the language allows several paradigms and provides multiple
tools, there are approximately as many programming styles than there
are \cpp{} programmers.

Two typical behaviours appeared very common to me in the recent
years. The first one is about the tools provided by the standard
library. First we hear people complaining: ``The standard does not
even contain {\em feature}. One has to use {\em libfeature} for it.''
Cue to the release of the aforementioned feature, and suddenly: ``The
specs for {\em feature} in the standard prevent efficient
implementations. One has to use {\em libfeature} for it.''

From where I stand, it looks like people's demands are ignored, and
they complain about it. Then they receive what they asked for, and
they obviously still complain about it. To be fair, it is probably not
the same people who complain in each situation. So I hope. Are they?

The second behaviour is about build times. \Cpp{} is well known for
being the language of quite long to compile programs, especially when
the program contains templates or other metaprogramming
techniques. This is a topic that regularly comes out as a major pain;
it was even listed as the second most frustrating thing in the 2021
Annual C++ Developer Survey
\cite{2021-annual-cpp-developer-survey}.

Despite that, once we are on the field we meet developers being like
``Hey, here's a header-only library for {\em feature} relying heavily
on templates and metaprogramming stuff.'' Simultaneously, we hear
complaints like ``Compilation times are too long! The committee should
do something about that!''. And still, more developers continue asking
for more header-only libraries because it is easy to integrate in a
project\footnote{Dependency management in \cpp{} is the main
  frustrating thing according to the aforementioned survey.}.

This is a problem of resource management. People have a limited
computing power, and they use every drop of it to compile and
recompile the same heavily template-based stuff again and again, until
it becomes unbearable. At this point, instead of reviewing their work,
they ask others (the committee, the standard, compiler vendors) to
solve their problem. Eventually they may also even buy more computing
power… only to push until using every drop of it. Doesn't it look
suspiciously like some other real-life important resource management
problems?

Ah, humans…

\bigskip

Aside from these little problems, that are actually as much inherent
to the language as they are the effect of human behaviour, there is a
very large and diverse ecosystem surrounding \cpp. Many nice libraries
and tools are published, often in a free software way, and every where
we can see that people want to do their best. This is very
stimulating.

%-------------------------------------------------------------------------------
\section{The Cost of Modernity}

With all these nice updates coming every three years, as \cpp{}
developers we naturally wonder if we can use the most recent features
or if we should stick with an older version.

There is a trend amongst us to push toward using the most recent
features, visible in online discussions or by the amount of libraries
whose description is along the line of ``A \cpp{20} library to do
something or something else''. This trend is called \emph{``modern''
  \cpp{} programming}, where ``modern'' unfortunately changes almost
every day.

It is tempting to go for the most recent features for several reasons:
to benefit from the increasing performance and support from compilers,
to write more straightforward code, to make a difference with the old
conservative folks… And obviously because it is fun. As developers we
are subject to a common disease: the refactoring, also known as ``just
burn everything and rewrite from scratch''.

On the other hand, using the bleeding-edge features has some
downsides: their implementation did not go through as much testing
than the old methods, and they may even not be available everywhere
yet. Moreover, if we write libraries, we must take into account that
not all projects can afford to use the most recent
features\footnote{Specialized tooling, weird architectures, politics…
  All of them may be valid explanations for inertia.}. Consequently,
when we choose a recent standard, we explicitly exclude a bunch of
project from using it.

On this topic, I would like to share a short experience I had.

\bigskip

Near 2014 I had been working on mobile games developed in \cpp{} and
available on iOS and Android (and unofficially on Linux and OSX for
the developers). \cpp11 was already available for a long time when the
first project was started so we naturally used it. Using
\code{std::unordered\_map} and the likes was quite common in the
projects.

At some point the Android version of a game started to take forever to
launch. It happened that the filling of a large
\code{std::unordered\_set} was the issue, and after digging a bit I
found why it was problematic only on Android. It turned out that the
Android Native Development Kit (NDK) was using GCC 4.7, for which
there was a bug that caused poor performance of the insertions in
unordered maps and
sets\footnote{\url{https://gcc.gnu.org/bugzilla/show_bug.cgi?id=54075}}.

Actually we probably had poor performance everywhere these
containers were used, but only one pathological case made it
visible. At the time we managed to work around the issue by using the
equivalent containers from Boost but the taste was bitter. We were
stuck with a broken compiler for Android and the solution we chose was
to tighten an already questioned and large dependency.

\bigskip

Could we have solved this problem by switching to a different or more
recent compiler? Probably later, but not at the moment. There was
indeed some experimental support for Clang in the NDK even though the
official way to go native was via GCC, so we did try that, and it
required a lot of work to compile the small shared common part of the
project. On the other hand, we were already using Boost so switching
to their container was simple, plus it guaranteed equivalent
performance on every platform.

Regarding the standard in use, in 2014 \cpp14 was barely ready and
certainly not available in the NDK. Moreover, to put things in
perspective, consider that even at the dawn of 2021, \cpp17 was still
not fully supported by the NDK\footnote{The revision r21e released in
  January lacks support for \code{std::filesystem}. This has been
  added in revision r22b, released in March the same year. Note that
  the former is a long term support (LTS) release, which means that it
  is expected to not have full \cpp17 for a long time.}. So, for this
platform at least, which I have heard is quite popular, pushing for
the trending features is counter productive as it can actually prevent
our tools to be used.

So should we stay with the oldest tools for maximum support? Look for
example at Curl or zlib to name a few, they are here since forever and
work perfectly, should we come back to the \emph{good old} C? Well,
probably not, but I guess there is a compromise to make between
cutting-edge and widely accepted.

\begin{guideline}
  There is an old saying going by ``use the right tool for the
  job''. Even though it is nowadays used by programmers to dismiss the choices
  other have made, there is some wisdom in it.

  When facing the question of whether or not to use a feature from a
  somewhat recent version of the language, take the time to consider
  what it deprives you and your users it terms of usability; then
  weight the benefit you gain from its usage.

  Know that apart from some details, code written for all older
  versions of \cpp{} are still valid in newer versions. On the
  contrary, the more recent are the features you use, the less your
  code can be imported into other projects.

  So, if you chose \cpp14 just to be able to write \code{auto foo() ->
    int \{\}} instead of \code{int foo() \{\}}, or if you chose \cpp20
  just to be able to use a pair of concepts, please reconsider, as
  these features have no benefit for the final product.
\end{guideline}

\chapter{Nice Things from C++11}

Ah, \cpp11, the long awaited update of the language. There was so much
hope in it! Just imagine, back in the day the language had no
automatic dynamic memory management, and not even a hash table in the
standard library… in the year 2000!

At that time we were trying to compensate for the lack of features
with Boost, eagerly waiting for the ever postponed upcoming \cpp 0x
that would change everything. We were going to have modules! And
concepts too!

Looking back into these years, I am quite happy that the committee
dropped the most complex features to finally publish a new version,
because most of the parts that were ready are quite awesome.

\section{At the Language Level}
%-------------------------------------------------------------------------------
\subsection{Range-based For Loops}
\label{range-based-for-loops}

\problemtitle

Before \cpp11, if someone wanted to iterate over a container, he had
to construct an iterator, of the correct type with respect to the
container, and write the loop with the classical three steps:
initialization (get an iterator on the begining of the container),
stopping condition (the iterator has not reached the end), and the
loop increment.

\begin{lstlisting}
void multiply(std::vector<int>& v, int c)
{
  // I'm going to iterate over v, so I need an iterator.
  typedef std::vector<int>::iterator it_t;
  const it_t end(v.end());

  for (it_t it(v.begin()); it != end; ++it)
    *it *= c;
}
\end{lstlisting}

\solutiontitle

All of this was quite verbose and repetitive when using standard
containers. Starting from \cpp11, all this administrative stuff can be
avoided by using a ranged-based for loop.

\begin{lstlisting}
void multiply(std::vector<int>& v, int c)
{
  // Just get all entries from v.
  for (int& vi : v)
    vi *= c;
}
\end{lstlisting}

This format uniformizes iteration over anything having a begining and
an end. Moreover, it handles the typical subtleties of for loops for
you: the end of the container is guaranteed to be computed only once
and the increment use the preincrement operator.

\subsection{\code{std::nullptr\_t} and \code{nullptr}}

\problemtitle

The traditional way to set a pointer to zero before \cpp11 was via the
\code{NULL} macro. So what would happen if we tried to compile the
following program?

\lstinputlisting{examples/nullptr/nullptr-98.cpp}

Did you expect the program to compile well and \code{foo(int*)} to be
called? Too bad, we are in a good old ambiguous call situation:

\begin{lstlisting}[language=bash]
error: call of overloaded ‘foo(NULL)’ is ambiguous
\end{lstlisting}

Since \code{NULL} is often defined as the integral value zero, the
compiler cannot distinguish it from an integer. Thus the error.

\solutiontitle

\marginheader{<cstddef>}%
%
As an answer to this problem, \cpp11 introduces the \code{nullptr}
keyword, which exactly represents a zero pointer. Its type is
\code{std::nullptr\_t} and it is implicitly convertible to any
pointer. Consequently, the code below compiles as expected.

\lstinputlisting[emph=nullptr]{examples/nullptr/nullptr-11.cpp}

Note that \code{nullptr} is a keyword available without prior
definition. Its type, on the other hand, is defined in header
\code{cstddef}.

\subsection{Scoped Enumerations}

\problemtitle

An enumeration in pre-\cpp11 code is just like a C enumeration: a list
of constants of type \code{int} in the scope containing the
enumeration. For example, note how \code{appliance::fan} can be
accessed directly and assigned to an integer in the code below:

\begin{lstlisting}
enum appliance
{
  fan,
  oven
};

int main()
{
  int v = fan;
  return 0;
}
\end{lstlisting}

While this is nice when we actually want the values of the
enumeration to be an alias for integer constants, for example to store
bit field flags, it can quickly become a mess when the entries of
different enumerations share the same identifier in the same scope.

\begin{lstlisting}
enum appliance
{
  fan,
  oven
};

enum follower
{
  fan,
  admirer
};
\end{lstlisting}

The above code will fail to compile with a message along the lines of
``error: ‘fan’ conflicts with a previous declaration.'' Now the
typical solution to that was to add a unique prefix to the enumerated
values, but in the times of namespaces and so, is it really a solution?

\solutiontitle

A scoped enumeration as introduced in \cpp11 is declared by adding the
\code{class} or \code{struct} keyword between \code{enum} and the
identifier. Additionally the storage type of the values can be defined
too:

\begin{lstlisting}
enum @\emcode{class}@ appliance
{
  fan,
  oven
};

enum @\emcode{class}@ follower : @\emcode{char}@
{
  fan,
  admirer
};
\end{lstlisting}

With these declarations the values are scoped in their respective
enumerations and do not conflict with each other; so if we want to
access a value we must now prefix it with the name of the enumeration,
like in \code{follower::fan}. Moreover, they are also not implicitly
convertible to \code{int} anymore, which may or may not always be a
good thing.

The fact that we can also define the type of their values allows for
smaller memory consumption when needed, but also add the possibility
to use longer-than-\code{int} values. Nevertheless, I usually don't
specify this type unless necessary since it has to be repeated when
the enumeration is forward declared. This repetition adds coupling and
complicates any change in the type, for something that look like
implementation details to me.


\subsection{\code{constexpr}}
\label{sec:constexpr}

Let's say we have some complex computation, like for example counting
the number of bits set to one in an integer:

\begin{lstlisting}
int popcount(unsigned n)
{
  return (n == 0) ? 0 : ((n & 1) + popcount(n >> 1));
}
\end{lstlisting}

What happens when we want to be able to call this function both with
run-time values and compile-time constant as arguments? In the code
below we would want the size of the array to be a constant, but as it
is written its size will be computed at run-time, which makes it a non
constant-sized array, which is not standard compliant.

\begin{lstlisting}
int main(int argc, char**)
{
  int array[popcount(45)];
  printf("%d\n", popcount(argc));

  return 0;
}
\end{lstlisting}

A typical solution for this problem is to implement the computation
via template classes and meta-programming:

\lstinputlistinghl{28}{examples/constexpr/constexpr-98.cpp}

This implementation works but has two major problems: first it is
incredibly verbose, second it forces us to implement the same
algorithm twice, respectively for run time and compile time
computations, doubling the risk of bugs and errors.

\bigskip

The \code{constexpr} keyword introduced in \cpp11 allows us to use the
same implementation for both compile-time and run-time computations.

\lstinputlistinghl[emph=constexpr]{10}{examples/constexpr/constexpr-11.cpp}

This keyword can be applied to a variable or a function to explicitly
tell the compiler that it can and should be computed at compile-time
when it appears in constant expressions. It is for example totally
possible to call the \code{constexpr popcount()} function as a
template argument, like in \code{popcount<popcount<42>{}>()}.

\subsection{Uniform initialization}
\label{uniform-initialization}

There are so many ways to initialize a variable in \cpp{} that it
became a running gag, so it's natural that a new initialization syntax
was added into \cpp11.

Anyway, let's jump to the problem. The following code is a recurring
issue in \cpp{}, so much that even experienced programmers stumble
upon it once in a while.

\begin{lstlisting}
struct foo {};

struct bar
{
  bar(foo f);

  int i;
};

int main()
{
  bar foobar(foo());
  printf("%d\n", foobar.i);

  return 0;
}
\end{lstlisting}

The question is ``what is \code{foobar} in the above code?'' Most
\cpp{} programmers will say it is a \code{bar} constructed with a
default-constructed \code{foo}. Unfortunately it is actually the
declaration of a function returning \code{bar} and taking a function
returning a \code{foo} as its single argument. This problem is known
as \emph{the most vexing parse}.

A workaround for this issue is to declare a temporary of type
\code{foo} and pass it to the constructor of \code{bar}:

\begin{lstlisting}
int main()
{
  foo f;
  bar foobar(f);
  printf("%d\n", foobar.i);
}
\end{lstlisting}

This is not very convenient, and suddenly the scope is polluted by a
useless variable. Moreover it can quickly become hard to implement
when a variable number of arguments are passed to \code{foobar}
(e.g. by forwarding the arguments of a variadic macro, or a variadic
template \ref{variadic-template}).

\bigskip

Enters the uniform initialization from \cpp11. By replacing the
parentheses with brackets in the construction of the argument, the
ambiguity is lifted.

\begin{lstlisting}
int main()
{
  bar foobar(foo{});
  printf("%d\n", foobar.i);
}
\end{lstlisting}

Used like that, the brackets mean something like ``a default-created
\code{foo}''. It can also be used to zero-initialize any variable,
which is especially nice in a template context:

\begin{lstlisting}
template<typename T>
void many_tees()
{
  // If T is a class, calls the default constructor.
  // If T is a fundamental type (e.g. int), its value is
  // whatever is in memory at &t1.
  T t1;

  // Declares a function t2 with no argument and returning
  // a T.
  T t2();

  // Seems to work. Does it?
  T t3 = T();

  // If T is a class, calls the default constructor.
  // If T is a fundamental type (e.g. int), its value is
  // the zero of this type.
  T t4{};
}
\end{lstlisting}

\subsubsection{When to Use the Uniform Initialization Syntax}

Consider the code below:

\lstinputlistinghl{43}{examples/uniform-initialization/struct.cpp}

What would be the output of this program? The highlighted line
constructs a variant of \code{foo} with what looks like the aggregate
initialization syntax, or is it uniform initialization? Maybe is it a
call to a constructor? Which one?

Here is the output of this program:

\begin{lstlisting}[language=bash]
$ a.out
.a=24, .b=42
.a=42, .b=24
.a=24, .b=24
\end{lstlisting}

So the first call is without surprise an aggregate initialization.

The second one is a call to the constructor; since there is one
defined for \code{foo\_constructor} then the aggregate initialization
is disabled. Note that before \cpp11 the compiler would have reported
an error, saying that the constructor should be used. Here it calls
the constructor silently even though it looks like an aggregate
initialization.

The last one is the worst. It creates an \code{std::initializer\_list}
\footnote{See section~\ref{initializer-list} for details about that.}
with the values, then pass it to the corresponding constructor.

This is painfully ambiguous. Unfortunately, the so-called ``modern''
\cpp{} programming trend is pushing for using bracket initialization,
maintaining ambiguities everywhere. I prefer to use it
parsimoniously. In particular, uses of \code{std::initializer\_list}
should certainly be avoided as they lure the programmer into thinking
that the assignment is an efficient aggregate initialization while it
is actually copying stuff around.

\begin{guideline}
  Use bracket initialization if:
  \begin{itemize}
  \item you want to initialize an aggregate,
  \item or you want to zero-initialize something in a context where
    you don't know for sure what the type of the thing is.
  \end{itemize}

  \emph{Do not} use bracket initialization if you want to call a
  non-default constructor.
\end{guideline}

As a side note, the bracket initialization can be used without
specifying the type, for example to assign a variable or for the
return statement of a function:

\begin{lstlisting}
struct interval
{
  interval();
  interval(int a, int b);
};

interval build_interval(bool f)
{
  interval b;

  if (f)
    b = {23, 23};

  return {42, 32};
}
\end{lstlisting}

This is terrible.

\begin{guideline}
  Mind the next reader, write what you mean.

  While the compiler can effortlessly find the type of a variable, or
  the type returned by a function, a human will either have to scan
  back toward the declarations or be helped by a tool, if they have
  one.

  Constructing a value without saying the type is ambiguous for a
  human. This is a lot of effort to put on the reader to save some
  characters on the writer's side.

  Don't build complex types without explicitly tell what you want to
  build.
\end{guideline}

\subsection{Compile-Time Assertions}

\problemtitle

Compile-time assertions, declared with a \code{static\_assert}, are a
way to verify a property at compile time.

For example, the code below declares a type with an array whose size,
with respect to the type's name, must be even:

\begin{lstlisting}
template<typename T, unsigned N>
struct even_sized_array
{
  typedef T type[N];
};
\end{lstlisting}

How can we ensure that \code{N} is always even? Before \cpp11 a
solution to that would have been to enter some template dance such
that passing an odd value would instantiate an incomplete type.

\lstinputlisting{examples/static-assert/even_sized_array-98.cpp}

Not only this is verbose, but these kind of template instantiation
errors are well known for producing kilometer-long unbearable error
messages.

\solutiontitle

Enters \code{static\_assert} in \cpp11. Just like the good old
\code{assert(condition)} would check that the given condition is true
at run time, \code{static\_assert(condition, message)} will check the
condition during the compilation.

\lstinputlisting{examples/static-assert/even_sized_array-11.cpp}

This is very less verbose than previously and it actually carries the
intent. The error message is also clearer, as it is just something
like ``static assertion failed: the message''.


\subsection{Lambdas}
\label{sec:lambda}

\problemtitle

How would we pass a custom comparator to \code{std::max\_element}
before \cpp11, say to compare strings by increasing size? Certainly by
writing a free or static function taking two strings as arguments and
returning the result of the comparison.

\begin{lstlisting}
// Comparator for strings by increasing size.
// Declared as a free function.
static bool string_size_less_equal
(const std::string& lhs, const std::string& rhs)
{
  return lhs.size() < rhs.size();
}

std::size_t
largest_string_size(const std::vector<std::string>& strings)
{
  // The comparator is outside the scope of this function.
  return
    std::max_element
      (strings.begin(), strings.end(), &string_size_less_equal)
    ->size();
}
\end{lstlisting}

Now what if we needed a parameterized comparator, for example to call
\code{std::find\_if} to search for a string of a specific size? The
free function would not allow to store the size, so we would
certainly write a functor object, i.e. a \code{struct} storing the
size parameter and defining an \code{operator()} receiving a string
and returning the result of the comparison.

\begin{lstlisting}
// Need a function object if the comparator has parameters.
struct string_size_equals
{
  std::size_t size;

  bool operator()(const std::string& string) const
  {
    return string.size() == size;
  }
};

bool has_string_of_size
(const std::vector<std::string>& strings, std::size_t s)
{
  string_size_equals comparator = {s};

  return
    std::find_if(strings.begin(), strings.end(), comparator)
      != strings.end();
}
\end{lstlisting}

\solutiontitle

Having the comparators as independent objects or functions is nice if
they are used in many places, but for a single use it is undoubtedly
too verbose. And confusing too. Wouldn't it be clearer if the
single-use comparator was declared next to where it is used?
Thankfully this is something we can do with lambdas, starting with
\cpp11:

\begin{lstlisting}
bool has_string_of_size
(const std::vector<std::string>& strings, std::size_t s)
{
  // Only three lines for the equivalent of the type
  // declaration, definition and the instantiation.
  // The third argument is a lambda.
  return std::find_if
    (strings.begin(), strings.end(),
     [=](const std::string& string) -> bool
     {
       return string.size() == s;
     })
    != strings.end();
}
\end{lstlisting}

\begin{guideline}
  Mind the next reader: keep your lambdas small.
\end{guideline}

%-------------------------------------------------------------------------------
\subsubsection{The Internals of Lambdas}
\label{lambdas-internals}

A declaration like

\begin{lstlisting}
[a, b, c]( /* arguments */ ) -> T { /* statements */ }
\end{lstlisting}

is equivalent to

\begin{lstlisting}
struct something
{
  T operator()( /* arguments */ ) const { /* statements */ }

  /* deduced type */ a;
  /* deduced type */ b;
  /* deduced type */ c;
};
\end{lstlisting}

Actually the compiler will create a unique type like this struct for
every lambda we write.

More conceptually, the parts of a lambda are the following:

\bigskip

\begin{center}
[{\it capture}]({\it arguments}) {\it specifier} -\textgreater
        {\it return\_type} \{ {\it body} \}
\end{center}

\bigskip

Where:

\begin{itemize}
\item {\it capture} is a list of variables from the parent scope that
  must be accessible inside the lambda. Use \code{[=]} to tell the
  compiler to automatically copy any variable used by the lambda, or
  \code{[\&]} to keep a reference to the corresponding variables from
  the parent scope. Variables can also be captured in a fine-grained
  way, e.g. \code{[=a, \&b]} to copy the value of \code{a} but store a
  reference to \code{b}.
\item {\it arguments} are the arguments of the function.
\item By default the variables captured by value cannot be assigned to
  in the body, unless we put the mutable keyword in {\it
    specifier}. In effect, it removes the \code{const} from
  \code{operator()}.
\item {\it return\_type} is the type of the value returned by this
  function, if any, or void otherwise. Contrary to any other
  function, the return type appear after the argument list instead of
  before.
\item {\it body} is the body of the function.
\end{itemize}


\subsection{Type Deduction with \code{decltype}}
\label{decltype}

\problemtitle

During the pre-\cpp11 era every variable, member, argument, etc. has
to be explicitly typed. For example, if we were writing a template
function operating on a range, how could we declare a variable of the
type of its items?

\begin{lstlisting}
template
<
  typename InputIt,
  typename OutputIt,
  typename Predicate,
  typename Transform,
>
void transform_if
(InputIt first, InputIt last, OutputIt out,
 Predicate& predicate, Transform& transform)
{
  for (; first != last; ++first)
  {
    /* some_type */ v = *first;
    if (predicate(v))
    {
      *out = transform(v);
      ++out;
    }
  }
}
\end{lstlisting}

\solutiontitle
Getting the correct type to put in place of \code{some\_type} was not
obvious, and required a fair share of template metaprogramming. Now,
with the \code{decltype} specifier introduced in \cpp11, the
programmer can tell the compiler to use ``the type of this
expression''.


\begin{lstlisting}
template
<
  typename InputIt,
  typename OutputIt,
  typename Predicate,
  typename Transform,
>
void transform_if
(InputIt first, InputIt last, OutputIt out,
 Predicate& predicate, Transform& transform)
{
  for (; first != last; ++first)
  {
    @\emcode{decltype(*first)}@& v = *first;
    if (predicate(v))
    {
      *out = transform(v);
      ++out;
    }
  }
}
\end{lstlisting}

In the example above, \code{decltype(*first)} is the type of the
result of dereferencing \code{first}.

\input{parts/11/auto}
\subsection{Rvalue References}

\problemtitle

In the function \code{build\_widget()} below, a temporary string is
created by the code \code{label + "def"}, then a reference to this
string is passed to the constructor of \code{widget}, where it is then
copied in \code{widget::m\_label}. Finally, the temporary is deleted.

\begin{lstlisting}
struct widget
{
  widget(const std::string& label)
    : m_label(label)
  {}

  std::string m_label;
};

void build_widget()
{
  std::string label("abc");
  widget w(label + "def");
}
\end{lstlisting}

Temporaries have the interesting property that they have no use but to
be consumed by an expression building another value. In other words,
in the example above, the purpose of the result of \code{label +
  "def"} is to end up in \code{widget::m\_label}. So why do we need a
copy? It would be more efficient to just pass the instance created by
the concatenation directly to the final variable.

\solutiontitle

To solve this problem, \cpp11 introduces the concept of rvalue
references, which are, to put it simply, references to
temporaries. These references are denoted with a double ampersand
\code{\&\&}. See the updated example, below:

\begin{lstlisting}
struct widget
{
  widget(@\emcode{std::string\&\&}@ label)
    : m_label(label)
  {}

  std::string m_label;
};

void build_widget()
{
  std::string label("abc");
  widget f(label + "def");
}
\end{lstlisting}

The signature of the constructor indicates that it accept temporaries
as arguments. However, this code is not enough to avoid copies of the
string yet, as the instruction \code{m\_label(label)} still copies the
string. The next step is to replace this instruction with
\code{m\_label(std::move(label))}, and then we will have no copy. This
is the topic of section \ref{move}.

Note that if the constructor's argument was \code{widget}, like in
\code{widget(widget\&\& that)}, then this would declare what is called
a move-constructor. This is a distinct constructor than the
copy-constructor and it is explicitly allowed to steal the internals
of its argument, as long as the instance behind the argument stays in
a valid state.



\section{Move Semantics}
%-------------------------------------------------------------------------------
\label{move}

\problemtitle

Copies! Copies everywhere! And dynamic allocations too!

So was the world before \cpp11. In these old days, if we had to pass
a large object to a function that needed its own instance of it, then
we would have certainly created a copy of it.

\begin{lstlisting}
struct catalog
{
  catalog(const std::vector<item>& entries)
    // Here we have one allocaction, for the storage,
    // plus copies of the elements from the vector.
    : m_entries(entries)
  {}

private:
  std::vector<item> m_entries;
};

void create_catalog(int n)
{
  // One allocation of n items, plus their initialization.
  std::vector<item> entries(n);

  // ...

  catalog c(entries);
  // I don't need the entries anymore, but they are still here.
}
\end{lstlisting}

\solutiontitle

\marginheader{<utility>}%
%
\Cpp11 introduces the move semantics and the \code{std::move()}
function. In practice it means far fewer copies, as the instances'
data is transfered from one place to another.

\begin{lstlisting}
struct catalog
{
  // Note the pass-by-value for the argument.
  catalog(std::vector<item> entries)
    // Explicit transfer into m_entries, no allocation.
    : m_entries(@\emcode{std::move}@(entries))
  {}

private:
  std::vector<item> m_entries;
};

void create_catalog(int n)
{
  // The single allocation in this program.
  std::vector<item> entries(n);

  // ...

  // I don't need the entries anymore, so I transfer them.
  catalog c(@\emcode{std::move}@(entries));
}
\end{lstlisting}

In the example above, there is only one allocation, for the vector of
entries in \code{create\_catalog()}, and zero copies. The ownership of
the allocated space is actually transfered from one vector to another
every time \code{std::move()} is used, until it ends up
in \code{m\_entries}.

%-------------------------------------------------------------------------------
\subsection{The Move Constructor and Assignment Operator}
\label{sec:move-constructor}

In order to make it work, a new type of reference was added to
the language; and classes can use this reference in constructors and
assignment operators to implement the actual transfer of data from one
instance to another. So to mark an argument as moveable, use \&\&:

\begin{lstlisting}
struct foo
{
  foo(foo@\emcode{\&\&}@ that)
    : m_resource(that.m_resource)
  {
    that.m_resource = nullptr;
  }

  foo& operator=(foo@\emcode{\&\&}@ that)
  {
    std::swap(m_resource, that.m_resource);
  }
};

// All of these call the constructor defined above.
foo f(foo());
foo f(construct_a_foo());
foo f(std::move(g));

// All of these call the assignment operator defined above.
f = foo();
f = construct_a_foo();
f = std::move(g);
\end{lstlisting}

The parameter of such functions is considered a temporary on the call
site. They are actually not restricted to constructors or assignment
operators, and can be used in any function.

The semantics of such arguments must be interpreted as ``I {\em may}
steal your data''. The {\em may} is important here, as there is no
guarantee on the call site that the instance going through an
\code{std::move} is actually moved.

%-------------------------------------------------------------------------------
\subsection{\code{std::move}, the Subtleties}

Naming quality: \faStar\faStarO\faStarO\faStarO\faStarO. Would have
put zero stars if I could.

As a great example of ``naming things is hard'', \code{std::move} does
not move anything. It actually just marks the value as transferable,
by casting it to an rvalue-reference. See this actual implementation
exctracted from GCC 9:

\begin{lstlisting}
template<typename _Tp>
constexpr typename std::remove_reference<_Tp>::type&&
move(_Tp&& __t) noexcept
{
  return static_cast
  <
    typename std::remove_reference<_Tp>::type&&
  >(__t);
}
\end{lstlisting}

\begin{guideline}
  As a rule of thumb of its usage, consider two use cases:

  \begin{enumerate}
  \item If callee \underline{always} takes ownership of the value:
    prefer arguments passed by value.
  \item If callee \underline{may} take ownership of the value: use an
    rvalue reference.
  \end{enumerate}
\end{guideline}

\begin{lstlisting}
// "I need my own bar."
void foo_copy(bar b) { /* ... */ }

// "I may take ownership of b...
//  ...Unless I don't."
void foo_rvalue(bar&& b) { /* ... */ }

void baz()
{
  bar b1;
  // Valid, b1 is unchanged in baz.
  foo_copy(b1);

  // Valid, we can forget about b1 now.
  foo_copy(std::move(b1));

  bar b2;
  // Invalid, b2 is not an rvalue
  // foo_rvalue(b2);

  // Valid, but is b2 actually moved?
  foo_rvalue(std::move(b2));
}
\end{lstlisting}

%-------------------------------------------------------------------------------
\subsection{\code{std::forward} and Universal References}

When \code{\&\&} is used in association with a type to be deduced,
like in the function below:

\begin{lstlisting}
template<typename T>
void foo(T&& arg)
{
  /* … */
}
\end{lstlisting}

Then it refers neither to an rvalue-reference nor a reference. In this
context, it is called a \emph{universal reference}. Depending on how
the function is called then \code{T\&\&} will be either deduced to an
rvalue-reference, if the argument is an rvalue, or else the references
are collapsed and \code{T\&\&} becomes an lvalue-reference, just like
\code{T\&}.

\begin{lstlisting}
void bar()
{
  std::string s;

  // calls foo(std::string&)
  foo(s);

  // calls foo(std::string&&)
  foo(s + "abc");
}
\end{lstlisting}

Now how one should pass to another function an argument received as a
universal reference? If it is deduced as an rvalue-reference, then
\code{std::move()} should be used, otherwise it should be passed as
is.

\marginheader{<utility>}%
%
Fortunately \cpp11 provides \code{std::forward()} to do the check for
us:

\begin{lstlisting}
template<typename T>
void foo(T&& arg)
{
  // Pass as an rvalue-reference or by address, whichever fits.
  foobar(std::forward<T>(arg));
}
\end{lstlisting}


\section{Template-Related Features}
%-------------------------------------------------------------------------------
\subsection{Template Aliases}

\problemtitle

Before \cpp11, we could simply not declare a typedef with a
parameterized type. For example, the following did not work:

\begin{lstlisting}
// We cannot do that.
template<typename T>
typedef std::vector<T> collection;

// We cannot do that either.
template<typename V>
typedef std::map<std::string, V> string_map;
\end{lstlisting}

The alternative was to use a struct to receive the type, then declare
the typedef inside the struct:

\begin{lstlisting}
// Workaround: nest the type.
template<typename V>
struct string_map
{
  typedef std::map<std::string, V> type;
};

// Easy to use, isn't it?
string_map<int>::type string_to_int;
\end{lstlisting}

\solutiontitle

Starting with \cpp11, the \code{using} keywork allows to declare
templated type aliases:

\begin{lstlisting}
// We can do that.
template<typename T>
@\emcode{using}@ collection = std::vector<T>;

// We can also do that.
template<typename V>
@\emcode{using}@ string_map = std::map<std::string, V>;

// Looks like a first-class type.
string_map<int> string_to_int;
\end{lstlisting}

The using keyword is also a replacement for typedef in general.

\subsection{External Templates}

\problemtitle

External templates are one of my favorite features from \cpp11.

When we were writing template code before \cpp11, for example a
template function, then every time the code was included in a
compilation unit the compiler would instantiate all used
symbols. Check for example the header below:

\lstinputlisting{%
  examples/extern-template/factorial-98.hpp%
}

If this header was included in two .cpp files, and its function
actually called, then its compiled code would be present in the object
file of each .cpp; something we can check with \code{nm}.

Following the example, let's compile the two files below:

\lstinputlisting{%
  examples/extern-template/foo-98.cpp%
}

\lstinputlisting{%
  examples/extern-template/bar-98.cpp%
}

Then \code{nm} would print:

\begin{lstlisting}[language=bash]
$ nm bar-98.cpp.o foo-98.cpp.o

bar-98.cpp.o:
000000000000001b T foo(int)
@\emcode{0000000000000036 W int factorial<int>(int)}@

foo-98.cpp.o:
000000000000001b T bar(int)
@\emcode{0000000000000036 W int factorial<int>(int)}@
\end{lstlisting}

Not only does this consume space (54 bytes per file here) but it also
costs a lot of work for the compiler and the linker. All of this adds
up, even for medium projects. Imagine that for each template function
or class the compiler parses the code, then it compiles it, then
writes all these bytes on disk; then all these bytes are read again by
the linker, who sorts all these duplicate symbols to keep only one.

At the end of this process, literally all the work done by the
compiler has been useless. Wouldn't it have been better not to do it
in the first place? The linker then spent more time removing the crap
rather than actually linking, and the programmer was wondering if he
could get a more powerful computer. Again.

\solutiontitle

Thankfully the \code{extern template} from \cpp11 allows us to skip
all this useless work. First we have to tell the compiler not to
instantiate the template for a subset of valid types by adding a
single line to our header:

\lstinputlistinghl{15}{%
  examples/extern-template/factorial-11.hpp%
}

This line tells the compiler not to instantiate
\code{factorial()} when \code{T = int}. Then we add a .cpp file
where we explicitly ask for the instantiation:

\lstinputlisting{%
  examples/extern-template/factorial-11.cpp%
}

That's it. Let's check with \code{nm}:

\begin{lstlisting}[language=bash]
$ nm bar-11.cpp.o foo-11.cpp.o factorial.cpp

bar-11.cpp.o:
000000000000001b T bar(int)

foo-11.cpp.o:
000000000000001b T foo(int)

factorial-11.cpp.o:
0000000000000036 W int factorial<int>(int)
\end{lstlisting}

The template function is indeed compiled in a single file, and absent
from the others\footnote{We can push even further in this case, by
having the whole body of \code{factorial()} in the .cpp file and only
its signature in the header. It would not only avoid the parsing of
the code but, more importantly, allow to remove from the header all
include directives related to implementation details. Note that in
this case the extern keyword can even be omitted.}.

\begin{guideline}
If you are writing template code for which you know some or all
instantiations, then add an \code{extern template} line for them in
the header, and explicitly instantiate them in an implementation file.
This is not always feasible, but when it can be done it should be done
\end{guideline}

\subsection{Variadic Templates}
\label{variadic-template}

Let's design a some kind of message dispatcher using features
available before \cpp11. The use case is presented by the code below.

\lstinputlisting[firstline=3]{examples/variadic-template/main-98.cpp}

The idea is to have a \code{dispatcher} whose role is to store
functions to be called with arguments provided later. It is
parameterized by the type of the arguments. If we want to accept any
kind of function, we must accept from zero to $n$ arguments.

A typical solution for that would have been to declare the type with
multiple template parameters and to default these parameters to a type
representing the state of not being used. Then the actual
implementation would have been selected by template specialization of
a super type. In order to keep this example short we are going to
limit our implementation to functions with up to two arguments.

Here is the dispatcher with the defaulted parameters.

\lstinputlisting[firstline=7]{examples/variadic-template/dispatcher-98.hpp}

Let's have a look at the dispatcher implementation with two arguments.

\lstinputlisting[firstline=7]{examples/variadic-template/dispatcher_impl_2.hpp}

Nothing special here. Let's look at the implementation with one
argument.

\lstinputlisting[firstline=7]{examples/variadic-template/dispatcher_impl_1.hpp}

This is very similar to the previous implementation. We could
factorize some parts (\code{queue()} and \code{m\_scheduled}) by
moving them in a parent class parameterized with
\code{function\_type}, but for now let's just keep it as is.

The \code{dispatch()} function cannot be factorized though, as its
signature and the \code{m\_scheduled[i](/*args*/)} line is specific to
each version.

The implementation with no arguments is equivalently similar, so I
won't include it here. I think you got the point: this is a lot of
code, redundant code, for a feature limited to only three use cases
amongst many. Wouldn't it be better if we could put all of that in a
single implementation that would handle any number of arguments?

\bigskip

The variadic template syntax introduced in \Cpp11 provides a solution
to this problem.

\lstinputlistinghl[firstline=5]{6}{examples/variadic-template/dispatcher-11.hpp}

Here we have a single implementation accepting any number of
arguments, and thus covering all use cases from the previous
implementation and more.

\bigskip

\label{parameter-pack}
The syntax of \code{typename... Args} defines a \emph{parameter
  pack}. This is like a template parameter but with an ellipsis.

A template with a parameter pack is called a \emph{variadic
  template}. Note that it is allowed to mix a parameter pack with
non-pack template arguments, as in \code{template<typename Head,
  typename... Tail>}.

When an expression containing a parameter pack ends with an ellipsis,
it is replaced by a comma-separated repetition of the same expression
applied to each type from the pack. For example:

\begin{lstlisting}
template<typename F, typename... Args)
void invoke(F&& f, Args&&... args)
{
  // this is equivalent to
  //   f(std::forward<Args0>(args_0),
  //     std::forward<Args1>(args_1),
  //     etc);
  f(std::forward<Args>(args)...);
}
\end{lstlisting}

Or, for another example, here is a function creating an array of the
names of the types passed as template parameters:

\begin{lstlisting}
template<typename... T>
void collect_type_names()
{
  const char* names[] = {typeid(T).name()...};
  // …
}
\end{lstlisting}


\section{Objects and Classes}
%-------------------------------------------------------------------------------
\subsection{Delegated Constructors}

\problemtitle

Before \cpp11, constructors could not call other
constructors. Consider the example below:

\begin{lstlisting}
struct foo
{
  foo(bar* b, float f, int i)
    : m_bar(b),
      m_f(f),
      m_i(i)
  {}

  foo(bar* b, int i)
    // can't I call foo(b, 0, i) directly?
    : m_bar(b),
      m_f(0),
      m_i(i)
    {}

private:
  bar* const m_bar;
  const float m_f
  const int m_i;
};
\end{lstlisting}

If we want to share initialization code between multiple constructors,
then we have to put it in some separate member function called by the
constructor, like this:

\begin{lstlisting}
struct foo
{
  foo(bar* b, float f, int i)
  {
    @\emcode{init}@(b, f, i);
  }

  foo(bar* b, int i)
  {
    @\emcode{init}@(b, 0, i);
  }

private:
  void @\emcode{init}@(bar* b, float f, int i)
  {
    m_bar = b;
    m_f = f;
    m_i = i;
  }

  // We cannot make any of these members const anymore.
  bar* m_bar;
  float m_f;
  int m_i;
};
\end{lstlisting}

This approach was kind of error-prone. Stuff may happen before and
after the call to the \code{init()} function, and actually nothing
prevents it to be called at any point in the life of the
instance. Finally, this is incompatible with \code{const} members.

\solutiontitle

Starting from \cpp11, a constructor can call another constructor:

\begin{lstlisting}
struct foo
{
  foo(bar* b, float f, int i)
    : m_bar(b),
      m_f(f),
      m_i(i)
  {}

  foo(bar* b, int i)
    : @\emcode{foo}@(b, 0, i)
  {}

private:
  bar* const m_bar;
  float m_f;
  int m_i;
};
\end{lstlisting}

This solves all problems.

%-------------------------------------------------------------------------------
\subsection{Deleted Constructors}

\problemtitle

Delegating constructors is a nice feature, but wat about totally
removing a constructor?

Consider the class below:

\begin{lstlisting}
struct scoped_listener
{
  scoped_listener(dispatcher& d, callback c)
    : m_dispatcher(&d)
  {
    m_id = m_dispatcher->connect(c);
  }

  ~scoped_listener()
  {
    m_dispatcher->disconnect(m_id)
  }

private:
  dispatcher* m_dispatcher;
  int m_id;
};
\end{lstlisting}

Creating copies of \code{scoped\_listener} does not make any sense, as
all copies would share the same \code{m\_id} and will thus trigger the
same call to \code{disconnect(int)} when destructed. See for example
its usage below:

\begin{lstlisting}
void foo()
{
  /* ... */
  scoped_listener listener(d, c1);
  scoped_listener copy(listener);
  scoped_listener other(d, c2);

  {
    // This assignment does not call disconnect(c2).
    other = listener;
    // disconnect(c1) is called here,
    // when other goes out of scope.
  }
  // disconnect(c1) is called twice here: in the
  // destruction of listener and copy.
}
\end{lstlisting}

One would typically want to forbid copies of \code{scoped\_listener} by
disabling its copy constructor.

Before \cpp11, one solution we would find here and there was to
declare the copy constructor and assignment operator as private. The
problem was that it was still available for the class and its
friends. So the programmer would then either implement an always
failing body for this constructor, emitting an error at run time, or
would just not implement the constructor, thus triggering an error at
link time.

These solutions were kind of weak, in the sense that the error, if
any, was presented quite late for the programmer, and with a not
obvious explanation.

\solutiontitle

Now, starting with \cpp11, the constructor and operators can be
explicitly deleted:

\begin{lstlisting}
struct scoped_listener
{
  scoped_listener(const scoped_listener&)@\emcode{ = delete}@;
  scoped_listener& operator=(const scoped_listener&)@\emcode{ = delete}@;

  scoped_listener(dispatcher& d, callback c)
    : m_dispatcher(&d)
  {
    m_id = m_dispatcher->connect(c);
  }

  ~scoped_listener()
  {
    m_dispatcher->disconnect(m_id)
  }

private:
  dispatcher* m_dispatcher;
  int m_id;
};
\end{lstlisting}

Using a deleted function will trigger a clear error from the compiler
when the call is encountered.

%-------------------------------------------------------------------------------
\subsection{Defaulted Constructors}

\problemtitle

Let's continue with constructors. Before \cpp11, each class would have
implicit constructors unless stated otherwise. One example of a
situation where an implicit constructor would not have been created
was the explicit declaration of a custom constructor by the
programmer. For example:

\begin{lstlisting}
struct foo
{
  foo(int) {}
};

int main()
{
  foo f1;     // fail
  foo f2(24); // ok
  foo f3(f2); // ok

  return 0;
}
\end{lstlisting}

In the above example, \code{foo} has no default constructor (but has
an implicit copy constructor). In order to have the default
constructor, the programmer had to implement one. The main problem
becomes maintenance: when new fields are added in the class, we have
to remember to update the constructor to initialize them.

\solutiontitle

Starting with \cpp11, the programmer can tell the compiler to
implement the constructor with what would have been the default
implementation if it was not deleted.

\begin{lstlisting}
struct foo
{
  foo()@\emcode{ = default}@;
  foo(int) {}
};

int main()
{
  foo f1;     @\emcode{// ok}@
  foo f2(24); // ok
  foo f3(f2); // ok

  return 0;
}
\end{lstlisting}

\subsection{Override and Final}

\problemtitle

So we have a class in our \cpp-98 code base that looks like that:

\begin{lstlisting}
struct abstract_concrete_base
{
  virtual ~abstract_concrete_base();
  virtual void consume();
};
\end{lstlisting}

Down the hierarchy we have some class that overrides the
\code{consume()} function\footnote{You may also find versions of this
pattern in the wild where the \code{virtual} keyword is repeated in
the derived classes, so the reader can suppose that it is an
overridden function without looking at the declaration of the parent
class.}:

\begin{lstlisting}
struct serious_business : abstract_concrete_base
{
  void consume();
};
\end{lstlisting}

Suddenly, the base class has some work to do in \code{consume()}
regardless of the details of the derived classes. Also, it is well
known that one should prefer non-virtual public functions calling
virtual private functions, so let's update our code:

\begin{lstlisting}
struct abstract_concrete_base
{
  virtual ~abstract_concrete_base();

  void consume()
  {
    stuff_from_base_class();
    do_consume();
  }

private:
   void stuff_from_base_class();
   virtual void do_consume();
};
\end{lstlisting}

Nice. Everything compiles well but all derived classes are broken. It
turns out that since \code{do\_consume()} is not a pure virtual
function, there is no problem to instantiate the class. All derived
will use the default \code{do\_consume()} and their old
\code{consume()} is now just another function of theirs.

Wouldn't it be nice if there was a way to detect the problem?

\solutiontitle

That's why the \code{override} keyword has been introduced in
\cpp11. By modifying the initial implementation of
\code{serious\_business} as follows:

\begin{lstlisting}
struct serious_business : abstract_concrete_base
{
  void consume() @\emcode{override}@;
};
\end{lstlisting}

The programmer explicitly tells the compiler that this function is
expected to override the same function from a parent class. So when
\code{abstract\_concrete\_base} is updated and its \code{consume()}
function becomes non-virtual, the compiler will immediately complain
that the function from \code{serious\_business} does not override
anything. Good!

\bigskip

Another keyword related to inheritance and function overriding has
been introduced in \cpp11: the \code{final} keyword. When used in
place of \code{override}, it tells the compiler that the function
should not be overridden in the derived classes. Previously a virtual
function could be overridden anywhere from the top to the bottom of
the hierarchy. Now if one declaration is marked as final, the derived
classes are not allowed to redefine the function anymore.

Additionally, one can place the \code{final} keyword after the name of
a class to indicate that this class cannot be derived at all:

\begin{lstlisting}
struct serious_business @\emcode{final}@ : abstract_concrete_base
{
  void consume() override;
};
\end{lstlisting}

\subsection{Using parent functions}

\problemtitle

Here is an actual problem I encountered when developing mobile games
in \cpp{}. When targeting Android devices, the application has to do
some calls from the \cpp{} part to the Java part. These calls are done
via the Java Native Interface (JNI), which is accessed via an object
of type \code{JNIEnv}.

Calling a Java method from Java requires the identifier of the class,
of type \code{jclass}, and the identifier of the method, of type
\code{jmethodID}. When all of them are known, we can call for example
a static method returning an integer as follows:

\begin{lstlisting}
JNIenv* env = /* … */;
jclass the_class = /* … */;
jmethodID the_method = /* … */;

const jint result =
  env->CallStaticIntMethod(the_class, the_method, /* arguments */);
\end{lstlisting}

I certainly did not want to write this stuff everywhere and was hoping
for something more like:

\begin{lstlisting}
// This variable represents the function to call.
static_method<jint> method = /* … */;

// Then we call it like a function.
const jint result = method(/* arguments */);
\end{lstlisting}

Since all method calls need the same parameters (i.e. the JNI
environment, the class, and the method), I put them in a base class
from which specialization for the returned type would be
created\footnote{This is simplified for the example. Check
  \url{https://github.com/IsCoolEntertainment/iscool-core/} for the
  actual code.}:

\begin{lstlisting}
class static_method_base
{
public:
  static_method_base
  ( JNIEnv* env, jclass class_id, jmethodID method_id );

protected:
  JNIEnv* m_env;
  jclass m_class;
  jmethodID m_method;
};
\end{lstlisting}

Then I would inherit from this class to create a function object for
each supported return type. Here for a static method returning an
integer:

\begin{lstlisting}
template< typename R >
class static_method;

template<>
class static_method<jint>:
  public static_method_base
{
public:
  // This constructor seems useless, doesn't it?
  static_method_base
  (JNIEnv* env, jclass class_id, jmethodID method_id)
    : static_method_base(env, class_id, method_id)
  {}

  template<typename... Arg>
  jint operator()(Arg&&... args) const;
};
\end{lstlisting}

At this point we have something weird: the only reason we have defined
a constructor in \code{static\_method\textless{}jint\textgreater} is
to be pass its argument to the parent class. Wouldn't it be better if
we could just say ``reuse the constructor from the parent''?

\solutiontitle

This is possible thanks to the \code{using} keyword from \cpp11:

\begin{lstlisting}
template< typename R >
class static_method;

template<>
class static_method<jint>:
  public static_method_base
{
public:
  // This is concise and precise.
  @\emcode{using static\_method\_base::static\_method\_base;}@

  template<typename... Arg>
  jint operator()(Arg&&... args) const;
};
\end{lstlisting}

When used like this, the \code{using} keyword imports the function
into the current class. It is also a way to solve the problem of
parent functions hidden by overloads in the child class:

\begin{lstlisting}
struct base
{
  void foo();
};

struct derived: base
{
  // This declaration hides base::foo().
  void foo(int);
};

void bar()
{
  derived d;
  // This fails, derived::foo requires an int argument.
  d.foo();
}
\end{lstlisting}

With the \code{using} keyword:

\begin{lstlisting}
struct base
{
  void foo();
};

struct derived: base
{
  @\emcode{using base::foo;}@
  void foo(int);
};

void bar()
{
  derived d;
  // This is ok, foo is a member function in derived.
  d.foo();
}
\end{lstlisting}


\section{In the Standard Library}
\subsection{Beginning and End of Sequence}

\problemtitle

All containers from the STL have a \code{begin()} and \code{end()}
member function to get an iterator on the first element in the
sequence or, respectively, just after the last element. Unfortunately,
there is no such function for the most basic sequences, i.e. C-like
arrays, so code like that was sure to fail before \cpp11:

\begin{lstlisting}
template<typename Sequence, typename T>
void replace_existing
(Sequence& s, const T& old_value, const T& new_value)
{
  *std::find(s.begin(), s.end(), old_value) = new_value;
}

void foo()
{
  int a[] = { 1, 4, 3 };
  replace_existing(a, 4, 2);
}
\end{lstlisting}

\solutiontitle

\marginheader{\cppheader{iterator}}%
%
Fortunately, \cpp11 introduces the
\code{std::begin()} and \code{std::end()} free functions which accepts
a C-like array\footnote{as long as we include \cppheader{array}}. Now,
this code will work in all cases:

\begin{lstlisting}
template<typename Sequence, typename T>
void replace_existing
(Sequence& s, const T& old_value, const T& new_value)
{
  *std::find(@\emcode{std::begin(s)}@, @\emcode{std::end(s)}@, old_value) = new_value;
}

void foo()
{
  int a[] = { 1, 4, 3 };
  replace_existing(a, 4, 2);
}
\end{lstlisting}

\subsection{Iterator Successors and Predecessors}

\problemtitle

There are many kind of iterators: forward iterators, that can be
incremented one step at a time with \code{operator++}, bidirectional
iterators, that can additionally be decremented with
\code{operator--}, and random access iterators, that can be
incremented or decremented by any amount at once.

When writing a function taking an iterator whose type is templated,
incrementing an iterator by more than one unit cannot be done
directly, as it would fail for non-random iterators. Prior to \cpp11,
this was done with \code{std::advance()}.

\begin{lstlisting}
template<typename Iterator, typename F>
void every_n_items
(Iterator it, std::size_t count, std::size_t step, F f)
{
  for (; count > step; count -= step)
  {
    f(*it);
    std::advance(it, step);
  }
}
\end{lstlisting}

\code{std::advance()} accepts a negative distance, in which case it
will advance… hum… backwards. Note that the function modifies its
argument, so using it to get another iterator from a given one is
cumbersome:

\begin{lstlisting}
template<typename Iterator>
void inplace_adjacent_sums(Iterator first, const Iterator& last)
{
  if (first == last)
    return;

  // Two steps to get an iterator on the second element. Ideally we
  // would have wanted to limit its scope to the loop too.
  Iterator second = first;
  std::advance(second, 1);

  for (; second != last; )
  {
    *first += *second;
    first = second;
    std::advance(second, 1);
  }
}
\end{lstlisting}

\solutiontitle

\marginheader{<iterator>}%
%
\Cpp11 introduces the replacement functions \code{std::prev()} and
\code{std::next()}, which are mostly here to make things clear and
more convenient. The former is to be used to move the iterator
backwards, while the latter is used to move it forward. If no distance
is passed, then it defaults to one.

Note that they return a copy of the new iterator, instead of modifying
its argument.

\begin{lstlisting}
template<typename Iterator>
void inplace_adjacent_sums(Iterator first, const Iterator& last)
{
  if (first == last)
    return;

  // Here we can initialize the second iterator in the loop, reducing
  // its scope.
  for (Iterator second = @\emcode{std::next}@(first); second != last; )
  {
    *first += *second;
    first = second;
    std::advance(second, 1);
  }
}
\end{lstlisting}

\subsection{Arrays}

\problemtitle

How do we usually pass a raw array to a function in \cpp? By passing
both a pointer to the first element and the number of elements, that
is how.

And when we need to copy such array, or to return such array from a
function, it becomes quite tricky.

\begin{lstlisting}
void replace_zeros_copy(int* out, int* in, std::size_t size, int r)
{
  // Can I trust the caller that the target array is large enough to
  // store the result? Is it a valid use case to pass null pointers
  // here? So many questions :(

  std::transform
    (in, in + size, out,
     [r](int v) -> int
     {
       return (v == 0) ? r : v;
     });
}

// We certainly cannot properly return a raw array unless we use
// some dynamic allocation or other techniques with their own
// problems.
int* replace_zeros_copy(int* array, std::size_t size, int r) { /* … */ }
\end{lstlisting}

\solutiontitle

\marginheader{<array>}%
%
With \cpp11 came \code{std::array} a template type to be used as an
alternative to raw arrays, with two parameters: the type of the
elements and their count.

\begin{lstlisting}
template<std::size_t N>
void replace_zeros_copy
(@\emcode{std::array<int, N>}@& out, const @\emcode{std::array<int, N>}@& in, int r)
{
  // Forget about the null pointers problem and the size issues,
  // everything fits perfectly here.

  std::transform
    (in.begin(), in.end(), out.begin(),
     [r](int v) -> int
     {
       return (v == 0) ? r : v;
     });
}

// We can also directly return the array.
template<std::size_t N>
@\emcode{std::array<int, N>}@ replace_zeros_copy(const @\emcode{std::array<int, N>}@& array, int r)
{
  std::array<int, N> result;
  replace_zeros_copy(result, array, r);
  return result;
}
\end{lstlisting}

An \code{std::array} is as cheap as a raw array. It has no fancy
constructor or any subtleties, and I can think of only two downsides
to using it:
\begin{enumerate}
\item the cost of including an extra header \cite{compile-health}
  \cite{stl-header-heft}, not negligible in this case as
  \cppheader{array} pulls \cppheader{utility} and more,
\item the need to carry the size as a template parameter.
\end{enumerate}

\begin{guideline}
  Because of the downsides associated with the type and its header,
  using \code{std::array} should always be preceded by two
  considerations:

  \begin{enumerate}
  \item Will the header propagate to many files?
  \item Will I have to templatize everything?
  \item Should I implement iterator-based algorithms instead?
  \end{enumerate}
\end{guideline}

Also, note that \code{std::array} is not a fixed-capacity vector. All
its entries are default initialized as soon as the array is
constructed. So if it contains complex types, it means that the
default constructor of this type is called for each entry.

\subsection{Random}

\problemtitle

Getting random numbers in \cpp{} was historically done by using
functions from the C library. First we initialized the a global
generator via a call to \code{srand()} then the values were generated
via \code{rand()}.

Unfortunately this generator was known for being a quite poor one.

\solutiontitle

\marginheader{<random>}%
%
This has been greatly improved in \cpp11. For the initialization, we
now have \code{std::random\_device} to access a hardware-based
generator\footnote{If there is one, otherwise it would be
  software-based.}, then there are many pseudo-random number
generators, \code{std::mt19937} being a quite popular one.

\begin{lstlisting}
int main()
{
  // The random device will get a random value from the system. It is
  // certainly the best random source we can have here.
  std::random_device seed;

  // This is a pseudo-random number generator, here initialized by
  // a value obtained from the random device.
  std::mt19937 generator(seed());

  // Finally this distribution will project the random values in the
  // [1, 24] range with equiprobability for each value.
  std::uniform_int_distribution<int> range(1, 10);

  for (int i = 0; i != 1000; ++i)
    printf("%d\n", range(generator));

  return 0;
}
\end{lstlisting}

Unfortunately the generators available in the standard library are
known to be both inefficient and not very good at randomness. As an
alternative, we can check for a permuted congruential generator (PCG)
\cite{pcg}, at \url{https://www.pcg-random.org/}.

\subsection{Smart Pointers}

Memory allocation in \cpp{} is a tough subject, dynamic allocation
being the hardest part.

Before \cpp11, the programmer had to be very careful with the life
span of dynamically allocated memory in order to, first, be sure that
it is released at some point and, second, that no access is made to it
once it has been released, not even another release.

See how many problems could occur with this short snippet:

\begin{lstlisting}
struct foo
{
  foo(int* p)
    : m_p(p)
  {}

  ~foo()
  {
    delete m_p;
  }

private:
  int* m_p;
};

void bar()
{
  foo f1(new int);
  foo f2(f1);

  int* p1(new int);
}
\end{lstlisting}

There are two problems with this code. First, \code{foo} does not
define nor disable its copy constructor, so, by default, its
\code{m\_p} pointer will be copied to the new instance. Then, when the
original instance and its copy will be destroyed, both will call
\code{delete} on the same pointer. This is what will happen with
\code{f1} and its copy \code{f2}. In the best scenario the program
would crash here.

The second problem is \code{p1}. This pointer points to a dynamically
allocated \code{int} for which no \code{delete} is written. When the
pointer will go out of scope then there will be no way to release the
allocated memory.

%-------------------------------------------------------------------------------
\subsubsection{Self-Deleting Non-Shared Pointer}
\label{sec:unique_ptr}

\marginheader{<memory>}%
%
\Cpp11 introduces a pointer wrapper named \code{std::unique\_ptr},
which has the merit of automatically calling \code{delete} on the
pointer upon destruction. Applied to the previous example, it would
solve one problem and force us to find a solution for the other:

\begin{lstlisting}
struct foo
{
  foo(std::unique_ptr<int> p)
    : m_p(std::move(p))
  {}

  // The default destructor does the job.

private:
  std::unique_ptr<int> m_p;
};

void bar()
{
  std::unique_ptr<int> p1(new int);

  foo f1(std::unique_ptr<int>(new int));
  foo f2(std::move(f1));
}
\end{lstlisting}

This program is undoubtedly safer than the previous one. The copy
constructor of \code{foo} is still not defined, but it is for sure
deleted since \code{std::unique\_ptr} has no copy constructor
neither. So, by default, we cannot share the resource between two
instances, which is great. The only solution here is to either
allocate a new int for \code{f2} or steal the one from {f1}. The
latter is implemented here.

Then, for the release of \code{p1}, it is automatically done when the
variable leaves the scope, so no memory is leaked.

\bigskip

One of the best features of \code{std::unique\_ptr} is the possibility
to use a custom deleter to release the pointer. This makes this smart
pointer a tool of choice when using C-like resources.

Check for example the use case of libavcodec's
\code{AVFormatContext}. The format context is obtained via a call to
\code{avformat\_open\_input(AVFormatContext**, const char*,
  AVInputFormat*, AVDictionary**)} and must be released by a call to
\code{avformat\_close\_input(AVFormatContext**)}. With the help of
\code{std::unique\_ptr} this could be done as follows:

\begin{lstlisting}
namespace detail
{
  static void close_format_context(AVFormatContext* context);
}

void foo(const char* path)
{
  AVFormatContext* raw_context_pointer(nullptr);

  const int open_result =
    avformat_open_input
      (&raw_context_pointer, path, nullptr, nullptr);

  std::unique_ptr
  <
    AVFormatContext,
    decltype(&detail::close_format_context)
  >
  context_pointer
  (raw_context_pointer, &detail::close_format_context);

  // …
}
\end{lstlisting}

With this approach, the format context will be released via a call to
\code{detail::close\_format\_context} as soon as
\code{context\_pointer} leaves the scope. Do we know if the memory
pointed by \code{raw\_context\_pointer} is dynamically allocated? We
do not, and it does not matter. What we have here is a simple robust
way to attach a release function to an acquired resource.

%-------------------------------------------------------------------------------
\subsubsection{Self-Deleting Shared Pointer}

\marginheader{<memory>}%
%
I have a hard time trying to find a use case where
\code{std::shared\_ptr} is the best solution, so let's just focus on a
good-enough solution.

This smart pointer is the answer to the problem of having a
dynamically allocated resource that may outlive its creator, and that
may also be accessed from two independent owners. In this case, the
ownership of the resource is unclear, as it is \emph{shared}, so the
idea is to keep the resource alive until all its owner release it.

For example, let's say we have a function whose role is to dispatch a
message to multiple listeners, whom will not process it right now. For
some reason the message cannot be copied, so we cannot send a copy of
it to everyone:

\begin{lstlisting}
void dispatch_message
(const std::vector<listener*>& listeners,
 const std::string& raw)
{
  message m = parse_message(raw);
  std::shared_ptr<message> p
    (std::make_shared<message>(std::move(m)));

  for(listener* l : listeners)
    listener->add_to_queue(m);
}
\end{lstlisting}

With this implementation the message outlives the scope of
\code{dispatch\_message()} and will remain in memory until all
listeners have released it.

Is it the best solution for this problem? Honestly I doubt
that. Actually, most uses of \code{std::shared\_ptr} I have seen,
including mine, looked a bit like a lack of reasoning about resource
management.

Wouldn't it have been better if the dispatching and the listeners were
scheduled in a loop and if the dispatcher would own the messages for
some iterations, or until they would be marked as processed by the
listeners? Do we really have to pay for dynamic allocations here?

\begin{guideline}
When you find yourself thinking that a shared resource with no clear
life span — a \code{std::shared\_ptr} — is a good answer to your
problem, please double check your solution, and ask for a second
opinion.
\end{guideline}

\begin{pitfall}
Many documentations declare, rightfully, that \code{std::shared\_ptr}
is thread-safe, and I have seen people using it to \emph{safely}
access the allocated resource in a concurrent way.

\bigskip

It must be said that the only thread-safe part in a
\code{std::shared\_ptr} is its reference counter.

\bigskip

Nothing is done — nothing can be done — to provide an automatic
thread-safe access to the allocated resource. Actually, even having
two threads doing respectively a copy (i.e.\ \code{std::shared\_ptr<T>
  p(sp)}) and a deletion (i.e.\ \code{sp.release()}), for the same
shared pointer \code{sp}, has no defined outcome without additional
synchronization. The only guarantee is that the increment of the
counter in the copy won't be done between the decrement and the
deletion from the release.

\end{pitfall}

%-------------------------------------------------------------------------------
\subsubsection{Shared Pointer Observer}

\marginheader{<memory>}%
%
What happens when the instance pointed by a \code{std::shared\_ptr}
owns a \code{std::shared\_ptr} pointing to the owner of the former?
Then we have a cycle and none of the instances will be released.

\begin{lstlisting}
struct foo;

struct bar
{
  std::shared_ptr<foo> m_foo;
};

struct foo
{
  std::shared_ptr<bar> m_bar;
};

void foobar()
{
  std::shared_ptr<foo> f(std::make_shared<foo>());
  std::shared_ptr<bar> b(std::make_shared<bar>());
  b->m_foo = f;
  f->m_bar = b;

  // The instances pointed by foo and foo->bar won't be deleted.
}
\end{lstlisting}

In order to break this kind of dependency loop, one pointer should be
declared as an \code{std::weak\_ptr}. Just like
\code{std::shared\_ptr}, this smart pointer is here to point to a
shared resource, except that it does not count as an owner of the
resource. Additionally, it provides a way to test if the resource is
still available via the \code{std::weak\_ptr::lock()} function, which
returns a shared pointer on the resource if it is available, or
\code{nullptr} otherwise.

\begin{lstlisting}
struct foo;

struct bar
{
  @\emcode{std::weak\_ptr}@<foo> m_foo;
};

struct foo
{
  std::shared_ptr<bar> m_bar;
};

void foobar()
{
  std::shared_ptr<foo> f(std::make_shared<foo>());
  std::shared_ptr<bar> b(std::make_shared<bar>());
  b->m_foo = f;
  f->m_bar = b;

  {
    std::shared_ptr<foo> resource(b->m_foo.lock());
    if (resource)
      printf("This message is printed.\n");
  }

  f.reset();

  std::shared_ptr<foo> resource(b->m_foo.lock());

  if (resource)
    printf("This message is not.\n");
}
\end{lstlisting}

\subsection{Initializer list}
\label{initializer-list}

\problemtitle

Initializing a container like \code{std::vector} with a predefined
sequence of values was quite verbose before \cpp11, as we had to
insert the values one by one:

\begin{lstlisting}
void transform
(std::vector<int>& values, const std::vector<int>& multipliers);

void up_down_transform(std::vector<int>& values)
{
  // Pfff it is not even const.
  std::vector<int> pattern(4);
  pattern[0] = 0;
  pattern[1] = 1;
  pattern[2] = 0;
  pattern[3] = -1;

  // Can't we have simple initialization, like we have for arrays?
  // const int pattern[4] = {0, 1, 0, -1};

  transform(values, pattern);
}
\end{lstlisting}

\solutiontitle

\marginheader{<initializer\_list>}%
%
However, thanks to the the introduction of
\code{std::initializer\_list} in \cpp11, we can now initialize our
containers wich bracket enclosed values:

\begin{lstlisting}
void transform
(std::vector<int>& values, const std::vector<int>& multipliers);

void up_down_transform(std::vector<int>& values)
{
  const std::vector<int> pattern = {0, 1, 0, -1};
  transform(values, pattern);

  // This one also works:
  // transform(values, {0, 1, 0, -1});
}
\end{lstlisting}

It's just like uniform initialization \aref{uniform-initialization}!
Or is it?

\subsubsection{The Problem with \code{std::initializer\_list}}

Instances of \code{std::initializer\_list} are created by the compiler
when it encounters a list of values between brackets and if the target
to which these values are assigned is, or can be constructed from, an
\code{std::initializer\_list}.

In the example above, we can create a vector from the values because
\code{std::vector} defines a constructor taking an
\code{std::initializer\_list} as it sole argument. This constructor
then copies the values from the list into the vector\footnote{Check
  \cite{beware-of-copies-initializer-list} for an illustration of the
  problem.}.

\bigskip

I think it is important to emphasize that: the constructor
\emph{copies} the values from the list. There is no way it can move
them, let alone wrap them.

\bigskip

So we have something that looks surreptitiously like the good old
simple and efficient aggregate initialization, that is consequently
identical in its syntax to uniform initialization
\aref{uniform-initialization}, and that is actually a quite
inefficient way to initialize something as soon as the element type is
non trivial.

The worse part is that bracket initialization is pushed by the
so-called ``modern'' \cpp{} trend, propagating this inefficiency
everywhere.

\begin{guideline}
  Avoid \code{std::initializer\_list}.

  If you really, really, want the target container to be const, then
  use an immediately invoked lambda to construct it:

  \begin{lstlisting}
    const std::vector<int> pattern =
      []() -> std::vector<int>
      {
        std::vector<int> result(4);
        result[0] = 0;
        result[1] = 1;
        result[2] = 0;
        result[3] = -1;

        return result;
      }(); // Watch out for the parentheses here,
           // it is a call to the lambda.
  \end{lstlisting}
\end{guideline}


\subsection{Threads}

\problemtitle

Before \cpp11 there was just nothing in the standard library to
execute a thread, so we had to either go native with pthreads, Windows
threads, or whatever fit our need, or else we could use an independent
library that would do the system abstraction for us, like Boost.Thread.

\solutiontitle

\marginheader{<thread>}%
%
Thankfully, \cpp11 came with \code{std::thread}, which does the system
abstraction for us and allows us to write portable threaded
code\footnote{When the targeted platform supports it. I'm looking at
  you WebAssembly!}. Launching a thread is as simple as this:

\begin{lstlisting}
auto expensive_computation =
  []() -> void
  {
    // This infinite loop takes forever to complete! It slows down
    // our awesome app, better put it in a thread.
    while (true);
  };

@\emcode{std::thread t(expensive\_computation);}@
// Here we go, the thread is running.
\end{lstlisting}

\marginheader{<mutex>,\\
<condition\_variable>}%
%
With the threads come the mutexes, condition variables, and more, that
will help us write synchronization points. Here is a more complete
example with two threads accessing shared data:

\lstinputlisting{examples/thread/flood.cpp}

\subsection{Tuple}
\label{sec:tuple}

The type \code{std::pair} is a nice utility class available in the STL
since forever. It is a struct accepting two template parameters and
defining two fields of these types, named ``first'' and
``second''. Approximately like:

\begin{lstlisting}
template<typename T1, typename T2>
struct pair
{
  T1 first;
  T2 second;
};
\end{lstlisting}

It is for example the value type of associative containers like
\code{std::map}, and now we are stuck with map entries with fields
named ``first'' and ``second'' while something like ``key'' and
``value'' would have carried the semantics better. Meh. Well, it may
not be the most expressive but at least we are reusing code via this
hyper generic reusable type, hooray!

Sorry, I inadvertently switched the rant-mode button on\footnote{That
  being said, I actually love the fact that even for small types like
  that effort is made by \cpp{} developers to build more efficient
  implementations. Search for \emph{compressed pair} or \emph{tight
    pair} for example.}.

\bigskip

Generalizing this structure to any number of fields/types was quite
complex, as before the arrival of variadic templates
\aref{variadic-template} in \cpp11 we had to go through some type lists
and other metaprogramming dances.

\marginheader{<tuple>}%
%
Thanks to their introduction, generalizing \code{std::pair} to any
number of elements is now somewhat simpler, and it is already done in
the STL via \code{std::tuple}. Additionally, \code{std::make\_tuple()}
is a utility function that can create a tuple from its values,
deducing the actual tuple type from its arguments. Finally, the main
companion function to \code{std::tuple} is \code{std::get()}, which
allows to access a tuple element by its index.

\begin{lstlisting}
// Creating a tuple explicitly.
std::tuple<int, float, bool> t(24, 42.f, true);

// Creating a tuple from its values.
t = std::make_tuple(42, 24.f, false);

// Accessing the elements in a tuple.
printf("%f\n", std::get<1>(t));
std::get<2>(t) = false;
\end{lstlisting}

Additionally, \code{std::tie()} is a helper function which creates a
tuple whose elements are references to the arguments passed to the
function.

\begin{lstlisting}
// Swap the first mismatching elements from two vectors.
void swap_first_mismatch(std::vector<int>& v1, std::vector<int>& v2)
{
  using iterator = std::vector<int>::iterator;

  iterator it_v1;
  iterator it_v2;

  // std::mismatch() returns a pair of iterators. We assign this pair
  // to a tuple whose elements are references to the two variables
  // declared above, meaning that we actually assign the variables.
  std::tie(it_v1, it_v2) = std::mismatch(v1.begin(), v1.end(), v2.begin());

  std::swap(*it_v1, *it_v2);
}
\end{lstlisting}

\bigskip

Declaring a tuple in a \cpp{} program typically happen for few reasons:
\begin{itemize}
\item By laziness where a struct could have been used\footnote{Don't
  do that.},
\item To return multiple values from a function:
  \code{std::tuple<int, int> get\_dimensions()}\footnote{Don't do
    that either, just write a meaningful type.}.
\item To lure Python developers into writing \cpp{}\footnote{But… why
  should we lure them if it's not their kind of stuff?}.
\item To group types, to carry type lists, in metaprogramming context.
\end{itemize}

\begin{guideline}
If you ever end up in a situation where \code{std::tuple} seems to be
the best type for an aggregate value, please reconsider.

Naming things is hard, but having a named struct will be better to
carry the meaning to the next readers than presenting them a bunch of
data thrown in a bag.
\end{guideline}

It is thus quite difficult to find a good short example for
\code{std::tuple}. One good example could be argument binding, but
it's a quite long example that needs many extra features. Another
example is found in one constructor of \code{std::pair}, which allows
to directly pass the arguments to use to construct its fields:

\begin{lstlisting}
template<typename T, typename U>
struct pair
{
  // Simplified for the example.
  template<typename... FirstArgs, typename... SecondArgs>
  pair
  (std::tuple<FirstArgs...> first_args, std::tuple<SecondArgs...> second_args)
    : first(/* here we pass the content of first_args */),
      second(/* here we pass the content of second_args */)
  {}
\end{lstlisting}

Yet another example is a metaprogramming use case where multiple
parameter packs must be passed to a type or function:

\begin{lstlisting}
// This won't work since the compiler cannot tell where to split the
// As and Bs
template<typename... As, typename... Bs>
struct failing_multi_pack
{};

// The code below will work though.

// This one is just the base template declaration, not defined.
template<typename As, typename Bs>
struct working_multi_pack;

// And we can specialize it for tuples. Now the compiler can split As
// and Bs.
template<typename... As, typename... Bs>
struct working_multi_pack<std::tuple<As>, std::tuple<Bs>>
{};
\end{lstlisting}

Well, I can see that you are a bit disappointed. Let's have a look at
the binding stuff. Hold on, it is not simple (still incomplete however):

\label{example:argument-binding}
\lstinputlisting{examples/tuple/binding.cpp}

\subsection{Argument Bindings}

The bindings example from \ref{sec:tuple} was quite complex, and it
does not even handle member functions. Since it is already using
\cpp11 features, it is left to the reader to consider how to implement
something equivalent with pre-\cpp11 features.

\marginheader{<utility>}%
%
However, a better implementation of a binding is available in \cpp11,
via \code{std::bind()}. This function creates a function object, of an
unspecified type, that will forward its arguments to the bound
function. This is something I would typically use in event handling.

\begin{lstlisting}
struct message_queue { /* … */ };

struct message_counter
{
  void count_message();
};

void connect_handlers(message_queue& queue)
{
  message_counter counter;

  // Here std::bind() creates a function object that calls
  // counter.process_message() when invoked.
  //
  // Note that the arguments of the function object are deduced from
  // the signature of message_dispatcher::count_message.
  queue.on_message(@\emcode{std::bind}@(&message_counter::process_message, &counter));
}
\end{lstlisting}

Interestingly, it is also possible to either force the value of an
argument, or to redirect an argument from the caller to a specific
argument of the called.

\begin{lstlisting}
struct message_queue { /* … */ };

void log_error(int queue_id, error e);

void connect_error_handler(message_queue& queue, int queue_id)
{
  // This one creates a function object accepting a single argument,
  // that calls log_error(queue_id, e), for a given argument e.
  //
  // Note that we must explicitly tell what to do with the argument
  // given by the caller here, via the placeholder.
  queue.on_message(@\emcode{std::bind}@(&log_error, queue_id, std::placeholders::_1));
}
\end{lstlisting}

In the above code, \code{std::placeholders::\_1} tells
\code{std::bind} to pass whatever is received as the first argument
(because the \_1) to the second argument of \code{log\_error} (second
because it is the second following the function when the binding is
created).

Note that the standard does not define how many placeholders must be
defined.

\subsection{Reference Wrapper}

\problemtitle

Be it \cpp11 or before, template argument deduction never pick a
reference. So, for example, if we wanted to create an
\code{std::pair<int, int\&>}, we could not use \code{std::make\_pair}:

\begin{lstlisting}
int a;
int b;

// Ok in C++11, not before. The members of the pair are references to
// a and b.
std::pair<int, int&> pair_1(a, b);

// Not ok, std::make_pair returns an std::pair<int, int>.
std::pair<int, int&> pair_2 = std::make_pair(a, b);

// Not ok in C++11 and before, for different reasons.
std::pair<int, int&> pair_3 = std::make_pair<int, int&>(a, b);
\end{lstlisting}

\solutiontitle

\marginheader{<utility>}%
%
The \code{std::ref()} function (as well as \code{std::cref()})
introduced in \cpp11 will help us with this problem. They both create
a reference wrapper for their argument (non const or const,
respectively), that is implicitly convertible to a raw reference.

\begin{lstlisting}
int a;
int b;

// Ok, std::make_pair returns an
// std::pair<int, std::reference_wrapper<int>>, which is itself
// convertible to std::pair<int, int&>.
std::pair<int, int&> pair_make = std::make_pair(a, @\emcode{std::ref(b)}@);
\end{lstlisting}

This is especially useful when binding variables that we don't want
to copy.

\lstinputlisting{examples/reference-wrapper/capacity_tracker.cpp}

\subsection{Functions}

Storing functions in a variable in \cpp{} used to be a pain. For
example, how would we implement \code{callable} such that the code
below works?

\begin{lstlisting}
void foo();
void bar(int arg);

struct some_object
{
  void some_method();
};

void test();
{
  std::vector<callable> scheduled;

  scheduled.push_back(&foo);
  // bar with arg = 5.
  scheduled.push_back(std::bind(&bar, 5));

  some_object c;
  scheduled.push_back(std::bind(&some_object::some_method, &c));

  // All stored functions can be called as if they were void().
  scheduled[0]();
  scheduled[1]();
  scheduled[2]();
}
\end{lstlisting}

Having a working type for all these use cases is a huge task, and
before \cpp11 our only hope were Boost.Function, some other
libraries\footnote{Search for FastDelegate, the Impossibly Fast C++
  Delegates, More Fasterest Delegates, and Ultimate Fast C++ Delegates
  II'. Some of them may not exist.}, or and homemade solution.

A binding like presented in Section~\ref{tuple} is a partial answer to
that but first, it does not even handle member functions, and two, it
is already \cpp11.

\bigskip

\marginheader{<utility>}%
%
The \code{std::function} type introduced in \cpp11, in combination
with \code{std::bind} \aref{bind}, is exactly what we
need to fix our previous example. The former is an object that represents a
function, and that can be called to invoke this function. It can be a
function object or a free function, everything works.

\begin{lstlisting}
void foo();
void bar(int arg);

struct some_object
{
  void some_method();
};

void test();
{
  std::vector<@\emcode{std::function<void()>}@> scheduled;

  scheduled.push_back(&foo);
  // bar with arg = 5.
  scheduled.push_back(std::bind(&bar, 5));

  some_object c;
  scheduled.push_back(std::bind(&some_object::some_method, &c));

  // All stored functions can be called as if they were void().
  scheduled[0]();
  scheduled[1]();
  scheduled[2]();
}
\end{lstlisting}

As far as can tell there is only one downside to \code{std::function},
it is that every call begins with a test checking if a function is
set. If we care about performance, it can be an issue\footnote{This
  test is used to throw an \code{std::bad\_function\_call} if the
  invocation is done on an empty instance. If you wonder why
  \code{std::function} does not reference by default a function that
  throws the exception, such that no test is done and the exception is
  still thrown when an empty function is invoked, know that I wonder
  too. If you have insight about it, I would love to know.}.

\subsection{Hash Tables}

\problemtitle

There were two main associative containers in \cpp{} before \cpp11:
\code{std::vector} and the likes, to associate integer keys with
almost any value type, and \code{std::map}, to use any key type whose
values can be strictly ordered. Similar to the latter, \code{std::set}
is an ordered set of values. In practice, \code{std::map} is an
\code{std::set} where the value type is \code{std::pair}, with the key
as the first field (with a const modifier), and the value as the
second field.

These last two collections had several problems, the main one being
that they require a total order on their entries. However, it is not
exceptional to have a type for which \code{operator<()} is not defined
and where providing one would be weird.

\begin{lstlisting}
struct color
{
  uint8_t red;
  uint8_t green;
  uint8_t blue;

  // Does it make sense to say that a color is less than another color?
  bool operator<(const color& that) const
  {
    // Note that in @\cpp@11 we could have used std::tie() to create a
    // tuple from the fields @\aref{sec:tuple}@, then used
    // std::tuple::operator<() to compare them lexicographically.

    if (red != that.red)
      return red < that.red;

    if (green != that.green)
      return green < that.green;

    return blue < that.blue;
  }
};
\end{lstlisting}

A way to avoid the weird operator is to declare the comparison
operator outside the type, as an independent function object. It has
the effect of not pretending that there is a meaningful order on the
given type. The type of the function object is then passed to
\code{std::set} or \code{std::map}; it becomes a property of the
container (i.e. the way its entries are ordered) instead of a property
of the data.

\begin{lstlisting}
struct color
{
  uint8_t red;
  uint8_t green;
  uint8_t blue;
};

// This is one way to order, amongst many. The advantage here is that
// there is no inherent ordering attached to color.
struct color_lexicographical_order
{
  bool operator()(const color& left, const color& right) const
  {
    if (left.red != right.red)
      return left.red < right.red;

    if (left.green != right.green)
      return left.green < right.green;

    return left.blue < right.blue;
  }
};

std::set<color, color_lexicographical_order> used_colors;
\end{lstlisting}

Another problem with \code{std::set} and \code{std::map} is that they
are implemented as a balanced tree, typically a red-black tree, where
the nodes are dynamically allocated. This is not very
performance-friendly since the data ends up scattered in memory, with
an additional cost of three pointers per entry. Additionally, look-up
is logarithmic.

\solutiontitle

\marginheader{<unordered\_map>,\\
<unordered\_set>}%
%
Since \cpp11, these containers are advantageously replaced by
\code{std::unordered\_set} and \code{std::unordered\_map}, which
provide the same service but with an implementation based on hash
tables. Now, look-up is more often constant-time than anything
else. The difficulty being to find a good hash function for the stored
type.

\begin{lstlisting}
struct color
{
  // Both std::unordered_set and std::unordered_map need a way to tell
  // if two instances are equal, in case of a hash collision.
  bool operator==(const color& that) const

  uint8_t red;
  uint8_t green;
  uint8_t blue;
};

namespace std
{
  // We can specialize std::hash for our own types. It is one of the
  // few symbols from the STL that we are allowed to specialize.
  template<>
  struct hash<color>
  {
    std::size_t operator()(const color& c) const
    {
      return
        ((std::size_t)c.red << 16)
        | ((std::size_t)c.green << 8)
        | (std::size_t)c.blue;
    }
  };
}

// No need to define the hash since we defined it via
// std::hash. Alternatively, we could have set the second template
// parameter std::unordered_set<color, some_hash_type>.
std::unordered_set<color> used_colors;
\end{lstlisting}

Moreover, a new way to insert elements in a set or a map has been
introduced, via the \code{emplace()} method. This method will create
the item in-place, without copies, while the previous \code{insert()}
approach would copy its argument into the container. It is available
both for the ordered and unordered variants:

\begin{lstlisting}
struct color
{
  color(uint8_t r, uint8_t g, uint8_t b);

  bool operator==(const color& that) const

  uint8_t red;
  uint8_t green;
  uint8_t blue;
};

std::unordered_set<color> used_colors;

// The arguments are forwarded to the constructor of color.
used_colors.emplace(218, 13, 13);
\end{lstlisting}

For the map version, since the entries are instances of
\code{std::pair}, we cannot pass all parameters in a single argument
list, as the compiler would not know where the constructor arguments
for \code{std::pair::first} would end and where the ones for
\code{std::pair::second} would start. Consequently, we must use tuples
\aref{sec::tuple} and the piecewise constructor:

\begin{lstlisting}
std::unordered_map<color, unsigned> color_count;

// The std::piecewise_construct is here to select the corresponding
// constructor from std::pair. The triplet argument will be forwarded
// directly to the construction of the first member; The second
// singleton argument will be forwarded to the second member.
color_count.emplace
  (std::piecewise_construct,
   std::forward_as_tuple(218, 13, 13),
   std::forward_as_tuple(1));
\end{lstlisting}

The main problem with \code{std::unordered\_set} and
\code{std::unordered\_map} is that they perform quite
poorly. Unfortunately, due to the constraints imposed by the standard,
they cannot be implemented in another way.

\begin{guideline}
If you need an hash table in the internals of your application or
library, i.e. it won't be visible by anything outside, then you
probably want to switch to another implementation. Otherwise, if your
library or application exposes a hash table, you have to consider:

\begin{enumerate}
\item if it makes sense to pass the exposed table as-is to another
  third-party application, then use the standard containers,
\item if the exposed table is expected not to go anywhere, then use a
  better container.
\end{enumerate}

So far, I have been quite satisfied with the robin hood hashing from
Martin Ankerl, a.k.a martinus, at
\url{https://github.com/martinus/robin-hood-hashing}.
\end{guideline}

\subsection{Type Traits}

\problemtitle

Let's have a look at this small function:

\begin{lstlisting}
template<typename T>
T mix(T a, T b, T r)
{
  return r * a + (1 - r) * b;
}
\end{lstlisting}

This is a quite basic function that just combine two values with a
weighted sum. Its implementation tells us that $r$ should probably be
in $[0, 1]$, but most importantly, the computation makes sense only if
T is a float-like type. If we want to prevent the compiler or the
programmer to call this function with an incompatible type, we can add
some SFINAE\footnote{Substitution Failure Is Not An Error, is a
  principle in template instantiations according to which the compiler
  must not emit an error if it fails to instantiate a
  template. Instead it should try another template candidate. This
  feature is often used and abused to provide multiple implementation
  behind the same function signature.}, for example by introducing a
type deduced from T that would not be defined if T is not a float-like
type. In \cpp98 the implementation could have been similar to:

\lstinputlistinghl{21}{examples/type-traits/mix-98.hpp}

Aside from the fact that \code{long double} is not handled, does it
work as expected?

\begin{lstlisting}
void test()
{
  // float and double are handled as expected.
  printf("%f\n", mix(1.f, 3.f, 0.5f));
  printf("%f\n", mix(1.d, 3.d, 0.5d));

  // Using integers triggers an error, we can say that it fails successfully!
  // printf("%d\n", mix(1, 3, 2));
}
\end{lstlisting}

Typing custom type like \code{only\_if\_float\_like} for every use
case is repetitive, so we would certainly end up splitting it into two
types, that could be used like
\code{only\_if<is\_float\_like<T>::value, T>::type}.

\solutiontitle

\marginheader{<type\_traits>}%
%
Lucky us, \cpp11 greatly simplify this work by introducing the
required types, in the form of \code{std::enable\_if} and
\code{std::is\_floating\_point\_type}. Using these types the whole
implementation is reduced to the following:

\lstinputlistinghl{4}{examples/type-traits/mix-11.hpp}

There are many other predicates and operations added in
\cppheader{type\_traits} in \cpp11, that can greatly help for
metaprogramming. I can only suggest to have a look at them on a nice
online reference.


% chrono is missing

\section{Miscellaneous}
In the previous sections we saw many nice features from \cpp11. There
are actually many more! As I did not use all of them, here come a
short description for the missing ones.

\subsection{Regular Expressions}

\marginheader{\cppheader{regex}}%
%
Regular expressions are now first class citizen in the STL. We can
finally easily write an HTML parser in a few lines
of \cpp{}\footnote{Reference for the
joke: \url{https://stackoverflow.com/a/1732454/1171783}. If you don't
follow this link, just know that one cannot correctly parse random
HTML with a regular expression.}!

Interestingly, the format for the expressions can be of almost any
existing syntax: ECMAScript, grep, awk…

\subsection{Explicit Conversion Operators}

Conversion operators can now be declared as explicit, to avoid
unexpected conversions:

\begin{lstlisting}
// This struct wraps an 8 bits non-negative integer value.
struct custom_uint8 { /* … */ };

// This struct wraps a 64 bits signed integer value.
struct custom_int64 { /* … */ };

// This struct wraps a 32 bits integer value.
struct custom_int32
{
  // This operator is explicit, as information may be lost when
  // truncating to 8 bits. Thus we don't want it to happen silently.
  @\emcode{explicit}@ operator custom_uint8() const
  {
    return custom_uint8(m_value);
  }

  // This operator is implicit, as all signed 32 bits values can be
  // represented in a signed 64 bits.
  operator custom_int64() const
  {
    return custom_int64(m_value);
  }

private:
  int32_t m_value;
};

void foo()
{
  custom_uint8 i8;
  custom_int32 i32;
  custom_int64 i64;

  // This is ok thanks to the implicit conversion operator.
  i64 = i32;

  // This won't work since there is an implicit conversion.
  i8 = i32;

  // Here the conversion is explicit, so it will work.
  i8 = static_cast<custom_uint8>(i32);
}
\end{lstlisting}

\subsection{Unrestricted Unions}

Unions can now contain non-POD members. I do not know where it is
useful but it can be done.

\subsection{String Literals}

New string literals have been introduced to represent UTF-8, UTF-16,
UTF-32 strings, as well as raw strings:

\begin{lstlisting}
const char* utf_8 = @\emcode{u8}@"I'm UTF-8.";
const char* utf_16 = @\emcode{u}@"I'm UTF-16.";
const char* utf_32 = @\emcode{U}@"I'm UTF-32.";

const char* raw = @\emcode{R"\_(}@This text is stored as is,
line breaks included,
  spaces included.
It is thus a four-lines text.@\emcode{)\_}@";
\end{lstlisting}

%-------------------------------------------------------------------------------
\subsection{User Defined Literals}

Have you ever mixed some units like in the declartion of \code{m} below?

\begin{lstlisting}
struct mass
{
  mass(float kg)
    : kilograms(kg)
  {}

  float kilograms;
};

void foo()
{
  // A mass of 2000 grams.
  mass m(2000);

  // Oops, mass() takes the value in kilograms :(
}
\end{lstlisting}

If you have already made this mistake, or just if you use this kind of
code, you may be happy to learn that you can now use custom suffixes
for literals, which may help to explicit the unit.

\begin{lstlisting}
// We define the 'g' suffix for float numbers, to represent grams.
mass @\emcode{operator"" \_g}@(float v)
{
  return mass(v / 1000);
}

void foo()
{
  // A mass of 2000 grams.
  mass m = 2000@\emcode{\_g}@;

  // m.kilograms is equal to 2, everything is fine.
}
\end{lstlisting}

%-------------------------------------------------------------------------------
\subsection{Very Long Integers}

Sometimes an \code{int} is a bit short to store large values. Then we
switch to \code{long int}, but sometimes it is still not enough. Now
with \cpp11 we can also switch to \code{long long int}\footnote{Also
available as \code{long int long}, or \code{int long long},
or \code{signed long int long}, or \code {long int signed long}, etc.}.

This type is guaranteed to be made of at least 64 bits,
while \code{int} and \code{long int} are made of at least 16 and 32
bits respectively.

Actually, I doubt I will ever use it, because when I need to store a
value in at least $n$ bits, I usually choose \code{int$n$\_t} type
from \cppheader{cstdint}.

%-------------------------------------------------------------------------------
\subsection{Size of Members}

Computing the size of a struct member in the old days of \cpp{}
required to use an instance of the struct, like in:

\begin{lstlisting}
struct foo
{
  int m;
};

int main()
{
  // There is no way to apply sizeof directly to m, so we "create"
  // an instance just to get the member. Fortunately the expression
  // on which sizeof is applied is not evaluated.
  printf("%ld\n", sizeof(foo().m));

  return 0;
}
\end{lstlisting}

This is doable for simple structs but clearly becomes difficult if
the constructor needs arguments.

In \cpp11 there is now a syntax to get the size of a member:

\begin{lstlisting}
int main()
{
  printf("%ld\n", sizeof(@\emcode{foo::m}@));
  return 0;
}
\end{lstlisting}

%-------------------------------------------------------------------------------
\subsection{Type Alignment}

If you are the kind of person who write allocators or who use
placement new in a buffer, then you will be happy to learn
about \code{alignas(T)}, which sets a type alignment to the
value \code{T}, if it is an integer, or to match the alignment of the
type \code{T} otherwise. Similarly, \code{alignof(T)} is the
alignement of the type \code{T}.

\begin{lstlisting}
char buffer[1024];
int consumed = 0;

template<typename T>
T* allocate()
{
  const int shift = consumed % @\emcode{alignof(T)}@;
  const int offset = (shift == 0) ? 0 : (@\emcode{alignof(T)}@ - shift);

  T* const result = new (buffer + consumed + offset) T();
  consumed += offset + sizeof(T);

  return result;
}
\end{lstlisting}

%-------------------------------------------------------------------------------
\subsection{Attributes}

Maybe have you already seen the \code{\_\_attribute\_\_} token in
some \cpp{} code? This is used to annotate the code with hints or
directives about some code. Unfortunately it is a compiler extension,
so it was not usable in portable code.

Starting with \cpp11, we now have a standard way to specify attributes
with the \code{[[some\_attribute]]} syntax. Their semantic is still
implementation-defined but at least it is syntaxically portable.

One of the only two well defined attributes in the \cpp11
standard is \code{[[noreturn]]}, which tells that a function never
returns to the caller:

\begin{lstlisting}
@\emcode{[[noreturn]]}@ void fatal_error(const char* m)
{
  printf("%s\n", m);
  exit(1);
}
\end{lstlisting}

The second one is \code{[[carries\_dependency]]} and I won't talk
about it.


% member initialization
% inline namespaces

\chapter{Nice Things from C++14}

\Cpp14 is frequently qualified as a bugfix version of \cpp11, indeed
most features are improvements or extensions of things introduced in
the latter.

Nevertheless, it does not mean that these improvements are not worth
it. Let's see.

\section{At the Language Level}
\subsection{Number separator}
\label{number-separator}

\problemtitle

Long numbers are hard to read. Check how long you need to get the
numbers in this \cpp11 code:

\begin{lstlisting}
int wisconsin_population = 5822434;
int california_population = 39512223;
\end{lstlisting}

My guess is that you grouped the digits by three in your mind in order
to get the numbers right. Didn't you? Usually when we write large
numbers like that, we group the digits with a separator to make the
numbers easier to read, like in 1'234'567.

\solutiontitle

Since \cpp14 we can use this syntax in the code:

\begin{lstlisting}
int wisconsin_population = 5'822'434;
int california_population = 39'512'223;

// It works with float numbers too.
const float f = 1'111.222'222;

// And also with for non-decimal numbers. Funny thing, the quote can
// appear anywhere between two digits.

const unsigned mask = 0xfff'0'00'ea;
\end{lstlisting}

\subsection{Binary Suffix}

\problemtitle

In \cpp11 and before, we used to declare bit patterns as hexadecimal
values, or shifts. For example, if one wanted to access the flags
stored in the packed fields from the image descriptor of a GIF 87a
file \cite{gif87a}, the code could have been similar to:

\begin{lstlisting}
void parse_image_descriptor_fields(std::uint8_t packed_fields)
{
  // Bitmask with shifts.
  const bool use_local_color_table = packed_fields & (1 << 7);

  // Bitmasks with hexadecimal values.
  const bool is_interlaced = packed_fields & 0x40;
  const std::int8_t bits_for_color_palette_size_minus_one =
    packed_fields & 0x07;
}
\end{lstlisting}

While not really obscure, this syntax requires some effort from the
reader, and a good understanding of the binary representation of
numbers. What if we could directly write the binary?

\solutiontitle

That's possible with \cpp14. Just like the 0x prefix introduces an
hexadecimal value, 0b introduces a binary value:

\begin{lstlisting}
void parse_image_descriptor_fields(std::uint8_t packed_fields)
{
  // Bitmask with binary literal.
  const bool use_local_color_table = packed_fields & @\emcode{0b10000000}@;

  // Note that we can use the number separator @\aref{number-separator}@ to split the mask.
  const bool is_interlaced = packed_fields & @\emcode{0b0100'0000}@;
  const std::int8_t bits_for_color_palette_size_minus_one =
    packed_fields & @\emcode{0b0111}@;
}
\end{lstlisting}

\subsection{\code{[[deprecated]]} Attribute}

\problemtitle

As code evolves, some parts naturally become obsolete. Most of the time,
obsolete code can be simply removed, but when it is published as a
public API, great care must be taken in order not to break client
code.

In such situation, the publisher would typically provide both the old
and the new API for a period of time long enough for the clients to
migrate. Which they may do only if they know which parts are obsolete!

So the issue here is actually to pass the information to the
developers that the code is going to be deleted. How can we do that?
Maybe by adding a paragraph in the release notes? I've heard that some
people read them. Or maybe a warning in the header? Here we are going
to rely on compiler-specific features, and the warning will be
displayed as soon as the file is included, even if the deprecated code
is not called. What about printing the deprecation message at run
time, when the code is executed\footnote{Please don't.}?

\solutiontitle

Starting with \cpp14, the \code{[[deprecated]]} attribute can be used
to target a specific symbol for deprecation, with an explicit message
that will be displayed to the user, if and only if the symbol is used.

\begin{lstlisting}
[[@\emcode{deprecated}@("Use the button class instead.")]]
class toggle_button
{
  /* … */
};

class button
{
public:
  [[@\emcode{deprecated}@("Use set_font(const font&) instead.")]]
  void set_font(const std::string& font_name);
};

void create_buttons()
{
  // This line will trigger a deprecation warning.
  toggle_button toggle;

  // This one doesn't.
  button b;

  // But calling the deprecated method will trigger the warning.
  b.set_font("sans");
}
\end{lstlisting}


\subsection{Generic Lambdas}

The introduction of lambdas \ref{sec:lambda} in \cpp11 unlocked many
possibilities for the programmer. Despite that, some use cases are
still a bit difficult to tackle. Take for example a unique predicate
that should be applied to two collections of different types, as in
the example below:

\begin{lstlisting}
#include <vector>
#include <algorithm>

bool same_count_by_sign
(const std::vector<int>& ints, const std::vector<float>& floats)
{
  if (ints.size() != floats.size())
    return false;

  const auto non_positive_integer =
    [](int v) -> bool
    {
      return v <= 0;
    };
  const auto non_positive_float =
    [](float v) -> bool
    {
      // This is the same body as above. Do I really need two lambdas?
      return v <= 0;
    };

  const std::size_t count_ints =
    std::count_if(ints.begin(), ints.end(), non_positive_integer);
  const std::size_t count_floats =
    std::count_if(floats.begin(), floats.end(), non_positive_float);

  return count_ints == count_floats;
}
\end{lstlisting}

It is a bit disappointing to have two lambdas with the same
body. Intuitively, one would want to templatize it. Unfortunately
template lambdas do not exist, so we have to switch back to the
pre-\cpp11 way with free functions or function objects.

\begin{lstlisting}
#include <vector>
#include <algorithm>

namespace
{
  // There we go, a simple predicate function declared far from its use.
  template<typename T>
  bool non_positive(T v)
  {
    return v <= 0;
  }
}

bool same_count_by_sign
(const std::vector<int>& ints, const std::vector<float>& floats)
{
  if (ints.size() != floats.size())
    return false;

  const std::size_t count_ints =
    std::count_if(ints.begin(), ints.end(), &non_positive<int>);
  const std::size_t count_floats =
    std::count_if(floats.begin(), floats.end(), &non_positive<float>);

  return count_ints == count_floats;
}
\end{lstlisting}

It works, for sure, but it is not what we expect. Moreover, as soon as
a variable has to be captured, we are back to the extremely verbose
functor objects.

\bigskip

Enters \cpp14 and the generic lambdas. By declaring the lambda's
arguments as \code{auto}, we can declare what is effectively the
equivalent of a template lambda.

\begin{lstlisting}
#include <vector>
#include <algorithm>

bool same_count_by_sign
(const std::vector<int>& ints, const std::vector<float>& floats)
{
  if (ints.size() != floats.size())
    return false;

  const auto non_positive =
    [](@\emcode{auto}@ v) -> bool
    {
      return v <= 0;
    };

  // It's the same predicate in both calls.
  const std::size_t count_ints =
    std::count_if(ints.begin(), ints.end(), non_positive);
  const std::size_t count_floats =
    std::count_if(floats.begin(), floats.end(), non_positive);

  return count_ints == count_floats;
}
\end{lstlisting}

We saw in \ref{sec:lambda} what was the equivalent function object of
a lambda. Here the corresponding transformation, given a lambda
declaration like

\begin{lstlisting}
[](auto a1, …, auto an ) -> /* return type */ { /* statements */ }
\end{lstlisting}

would be something like

\begin{lstlisting}
struct something
{
  template<typename T1, …, typename Tn>
  /* return type */ operator()(T1 a1, …, Tn an) const { /* statements */ }
};
\end{lstlisting}

Note that the type of the arguments are not related. Each use of
\code{auto} here corresponds to a single template parameter.

\subsection{Lambda Capture Expressions}

When we are defining a lambda, we have the possibility to capture
variables of the enclosing scope such that they become available in
the body of the lambda \aref{lambdas-internals}. In practice, such
variables are either copied (like \code{var1} below, due to the
\code{=} syntax), or referenced (like \code{var2} below, with the
\code{\&} syntax).

\begin{lstlisting}
const auto foo = [=var1, &var2]() -> void {};
\end{lstlisting}

Unfortunately there is no other variant of the capture, which leads to
situations like this:

\begin{lstlisting}
void wait_for_new_players(const std::string& group)
{
  std::string display_label = group + ": ";

  // Copies are striking again. Here display_label is copied into a
  // field of the function object associated with this lambda; not
  // moved nor initialized on construction.
  auto add_player_to_list =
    [display_label](player_id id, const std::string& player alias)
    {
      m_player_list->add(id, display_label + alias);
    };

  m_team_service.wait_for_players(group, add_player_to_list);
}
\end{lstlisting}

In the above example, we would want to avoid the copy of the display
label into the equivalent member variable of the lambda. In \cpp11
there is only two ways to achieve that: either by creating the display
label in the callback's body, thus creating a new string on each
call\footnote{We would also have a copy of the captured group variable
anyway.}, or by using a custom function object, \cpp98-style.

Additionally, we can note that this behavior prevent any use of a
move-only type for a captured variable, such as
\code{std::unique\_ptr}. Indeed, there is no way to move the variable
from the outer scope into the lambda.

\bigskip

In \cpp14, capture lists have been extended to allow the initialization
of a variable by an expression. We can use that to initialize the
display label with the final string directly:

\begin{lstlisting}
void wait_for_new_players(std::string group)
{
  // display_label is initialized directly in the construction of the
  // lambda, thus saving a copy.
  auto add_player_to_list =
    [@\emcode{display\_label = group + ": "}@]
    (player_id id, const std::string& alias)
    {
      m_player_list.add(id, display_label + alias);
    };

  m_team_service.wait_for_players(group, add_player_to_list);
}
\end{lstlisting}

This also solve the move-only type situation since we can not move the
variable:

\begin{lstlisting}
void apply_filter(std::vector<int>& values, std::unique_ptr<filter> filter)
{
  std::for_each
    (values.begin(), values.end(),
     [@\emcode{f = std::move(filter)}@](int& v) -> void
     {
       v = f->transform(v);
     });
}
\end{lstlisting}


% omit the return type
% decltype(auto)
% better constexpr
% variable templates
% [[deprecated]]
% new/delete elision

\section{In the Standard Library}
% make_unique
% integer_sequence
% exchange
% tuple addressing by type
% standard user defined literals
% heterogeneous lookup in associative containers
% enable_if_t (type aliases)
% quoted


\bibliographystyle{alpha}
\bibliography{bibliography.bib}

\appendix
\input{parts/license.tex}

\end{document}
