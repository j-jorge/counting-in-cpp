\documentclass{book}

\usepackage[english]{babel}
\usepackage[normalem]{ulem}
\usepackage{cite}
\usepackage{enumitem}
\usepackage[dvipsnames]{xcolor}
\usepackage{fontawesome}
\usepackage{times}
\usepackage[marginparwidth=2.5cm]{geometry}
\usepackage{graphicx}
\usepackage[colorlinks=true,allcolors=Blue]{hyperref}
\usepackage{bookmark}
\usepackage[T1]{fontenc}
\usepackage{marginnote}
\usepackage{mdframed}
\usepackage{sectsty}
\usepackage[thicklines]{cancel}
\usepackage[nottoc,numbib]{tocbibind}
\usepackage[parfill]{parskip}

% Write C++XX using the format from cppreference.com: capital C, no
% dash nor space.
\newcommand{\cpp}[1]{C++{#1}}
\newcommand{\Cpp}[1]{C++{#1}}

\newcommand{\cppheader}[1]{\textless #1\textgreater}

\newcommand{\aref}[1]{[\ref{#1}]}

\title{Counting from 98 to \cancel{20} 14 \\ in \Cpp{}}
\author{Julien Jorge}
\newcommand{\version}{@CMAKE_PROJECT_VERSION@}

%\usepackage{inconsolata}
%\usepackage{courier}
\usepackage{listings}
\usepackage{xcolor}
%\usepackage{ucs}

%%\lstset{basicstyle=\ttfamily,
  %% commentstyle=\ttfamily\color{purple!40!black},
  %% stringstyle=\ttfamily\color{orange},
  %% keywordstyle=\bfseries\color{green!40!black},
  %% identifierstyle=\color{black},
  %% showstringspaces=false,
  %% numbers=left,
  %% tabsize=2,
  %% breaklines=true,
  %% framexleftmargin=0mm,
  %% captionpos=t,
  %% xleftmargin=1mm,
  %% extendedchars=\true,
  %% inputencoding=utf8x,
  %% emphstyle={\color{DarkRed}\bfseries}
%%}


\newcommand{\code}[1]{{\tt #1}}

\newcommand{\marginheader}[1]{%
  \marginnote{%
    \begin{center}%
      \includegraphics[width=0.5\linewidth]{assets/header-icon.pdf}

      \footnotesize #1
    \end{center}%
  }[-2.2em]%
}

\newenvironment{guideline}
               {
                 \begin{mdframed}[
                     topline=false,
                     rightline=false,
                     bottomline=false,
                     linewidth=1pt,
                     frametitle={Guideline}
                     ]
               }
               {
                 \end{mdframed}
               }

\newenvironment{pitfall}
               {
                 \begin{mdframed}[
                     topline=false,
                     rightline=false,
                     bottomline=false,
                     linewidth=1pt,
                     frametitle={It's a trap}
                     ]
               }
               {
                 \end{mdframed}
               }

\begin{document}
\allsectionsfont{\sffamily}

\frontmatter
\begin{titlepage}
  \pdfbookmark{Cover}{titlepage}

  \newgeometry{margin=3cm, noheadfoot, nomarginpar}
  \pagestyle{empty}
  \centering

  \makeatletter

  \vspace*{\fill}
  \vspace{-9cm}

  {\Huge\bfseries \@title}

  \vspace{1cm}
  Version \version

  \vfill

  \begin{minipage}{0.85\textwidth}
    \vfill

    \@author

    \href{mailto:julien.jorge@stuff-o-matic.com}{julien.jorge@stuff-o-matic.com}
  \end{minipage}%
  \hfill%
  \begin{minipage}{0.15\textwidth}%
    \raggedright
    \includegraphics[width=\linewidth]{assets/by-sa.pdf}
  \end{minipage}

  \makeatother
\end{titlepage}


\cleardoublepage
\pdfbookmark{\contentsname}{toc}
\tableofcontents

%-------------------------------------------------------------------------------
\chapter{Preface}
\renewcommand*\thesection{\arabic{section}}

%-------------------------------------------------------------------------------
\section{About this Book}

Once upon a time, as I was working on a large project with others
\cpp{} programmers, I was asked to set up a series of talks about the
language and especially about what has changed since the arrival of
\cpp11. It was in 2020.

So I started to write some slides with what seemed to be the key
features from \cpp11, and quite soon I had to face the truth: it is a
lot of content, and there was three additional major updates in the
language that should be covered too.

In the end I did not do the talks. However, I kept working on the
slides, until eventually I decided to switch the format. It is
probably too much material for a talk, but what about a small book?

Hence this document.

\bigskip

The goal is to list many, if not all, essential features introduced in
the \cpp{} language since its first well-known deep update, known as
\cpp11, up to the most recent version of the standard, which is \cpp20
by the time I am writing this.

These features are for the most part presented following a format
where the pre-\cpp11 way is reminded to the reader, with a short
explanation of why it may have been problematic or inefficient, then
the new way of doing things is presented.

Some parts of the language are silenced, mostly the ones for which I
don't know much, other parts are more thoroughly presented. In any
case, this book won't go into the details and subtleties of any
feature, nor into compiler-specific stuff. The reason being mostly
time (as far as I can tell I have a limited amount of that in my life)
and space (the book is already large enough). The reader is invited to
satisfy his curiosity and complete is knowledge by reading other
material. For example, the website \url{https://en.cppreference.com}
has everything we need to know about any feature of the language.

Be advised that some critics may suddenly appear in these pages, about
the language or the programmers. Keep in mind that those are personal
opinions and may change suddenly!

Finally, a basic knowledge of the language is preferable for the
reader to enjoy this book, as some notions will be used without being
explained.

\bigskip

While this book is about \cpp, one should remember that \cpp{} itself
is just a tool in the programmer's toolbox. If you are focusing on
learning \cpp{} to become a programmer, a good programmer, I would
suggest to rethink your plan and learn programming on a larger scale:
computer architecture, algorithms, data structures, project
management, packaging, dependency management, coding style, reviews,
testing…  There are many aspects to be familiar with in the daily life
of a programmer, keep some place for them.

As a good starting point, every programmer should read Code Complete
\cite{code-complete}. This book goes in detail in all aspects of
software development, backed up by data coming from over decades of
real-life projects, so if you read it you will also gain part of the
knowledge from these people who tried, failed, and succeeded before
you.

Clean Code \cite{clean-code} is a good second book to read, even
though I would not approve all suggestions. For example, it pushes for
intensive factorization and the use of object-oriented programming
everywhere, which are rather things I have painfully learnt use
parsimoniously. Still, the book is a reference in software
development, so you should read it at least to make yourself an
opinion and to know what is going on in the business.

\bigskip

Finally, remember that if reading is acquiring the experience of
others, practicing is building our own experience. So I strongly
recommend to find or start a side project, maybe even a rewrite of
existing tools, just to try and get a grasp of the potential underlying
complexity.

%-------------------------------------------------------------------------------
\section{About the Author}

Should one take this book's content at face? Is the author legit? It
is normal to question the legitimacy of who pretend to give advice.
So in order to help you gauging the credibility of this book, I think
it is important to tell a bit about me.

\bigskip

First of all, I read a lot of code. I read code almost everyday, on
GitHub, on blogs, on StackOverflow, Reddit, and on other forums. I
read code written by me or others, from my personal projects, from my
employer, or from random projects I find on the Internet. All this
code is displayed either on my laptop, or in a terminal connected to a
remote server, or on my phone. Reading is undoubtedly the main part of
my programming activities.

On the productive side, I write code as a hobby since 1994, and
professionally since 2005. I have at least half a million of \cpp{}
behind me, just counting the lines of code that survived in past
projects I could find.

I also have coded a lot of Bash, a good share of Java, a bit of HTML
and a bit of JavaScript. Additionally, in a more anecdotal way, I
coded some C\# programs, some ActionScript, some Pascal and Delphi
ones, a small compiler in Eiffel, some pet projects in Visual Basic
too, BASIC a long time ago, on a Commodore 128 and later under DOS. I
also did a bit of Objective-C.

Let's face it though, a good share of this code was crap.

\bigskip

Some code did end up well nonetheless. One project I am proud of is a
mobile game written in \cpp, which was played by more than 500'000
people every day during more than three years. Aside from that, I also
took part in projects that were struggling to start and brought them
into a viable product. So I guess I made stuff that does not suck.

When I code I tend to think about long term and architecture. I try
not to take any shortcut and to answer the problem without attempting
to solve the future. I code in small boxes, many, with the intent that
they can be broken, removed, replaced, without changing everything. It
wasn't always like that but that's how I work today.

Finally, I am certainly not the type to rush for the new thing. I like
tools and practices that have been well tested, so you probably won't
hear me telling you to use this new thing from \cpp{42} because it's
new and it will show that you are modern and blah blah blah.

\bigskip

Convinced? Anyway, I hope you will find something useful in this book :)

%-------------------------------------------------------------------------------
\section{License}
This work of art is licensed under a Creative Commons
Attribution-ShareAlike 4.0 International License. See
Appendix~\ref{license} for the full license text.

\marginheader{This one.}%
%
The header file icon displayed in the margins, like the one on the
side of this paragraph, is based on the New Document icon from the
Tango Icon Theme \cite{tango-icon-theme}. The original icon is in the
public domain and in order to respect the intent of the source the
variation made for this book, as in the SVG file available in the
repository containing the sources of this book, is also released in
the public domain.

\renewcommand*\thesection{\arabic{chapter}.\arabic{section}}



\mainmatter
%-------------------------------------------------------------------------------
\chapter{The \Cpp{} Language and its Community}

\Cpp is a quite old programming language: it was created by Bjarne Stroustrup
in 1982. It initially thought as an improved C.

The first official specifications were \emph{The C++ Programming
  Language} \cite{the-cpp-programming-language-1st}. This book was
then updated multiple times \cite{the-cpp-programming-language-2nd},
\cite{the-cpp-programming-language-3rd},
\cite{the-cpp-programming-language-se}, and
\cite{the-cpp-programming-language-4th}.

Starting from 1998, the official specifications are described in
\emph{The Standard}.

%-------------------------------------------------------------------------------
\section{The Standard}

The standard from 1998 set the ground for a new era of compilers by
defining how \cpp{} programs should behave, and by describing the
content of the standard library as well as the constraints on its
implementation.

The standard is defined by {\em The ISO C++ Committee}, i.e. people
from the industry: Google, HP, Oracle, Intel and many more.

It is worth noting that no code is provided by the committee. The
standard just describes the language, then independent developers
(mostly compiler vendors) provide the implementation. Theoretically,
developers can switch easily from one compiler to the other. Clients
are not tied to the vendor's compiler anymore.

As I write there are six revisions for this document, informally
identified by the year they came out:

\begin{itemize}
\item {\bf \Cpp98} defined the core features: the syntax, the memory
  model, templates, namespaces… And the Standard Template Library
  (STL): \code{std::vector}, \code{std::string}, \code{std::map}…
\item {\bf \Cpp03} fixed some wording and inconsistencies.
\item {\bf \Cpp11} the beginning of what is called \emph{modern
  \cpp}. Initially expected during the 00's, the committee had to kiss
  goodbye to some awaited feature. Better done than perfect.
\item {\bf \Cpp14} mostly bug fixes but also nice features:
  e.g. variable templates.
\item {\bf \Cpp17} nice new features: fold expressions, \code{if
  constexpr}, copy elision, \code{std::optional}, \code{
  string\_view}…
\item {\bf \Cpp20} \code{std::span}, concepts, modules… Also: your
  compiler still doesn't fully support \cpp17.
\end{itemize}

%-------------------------------------------------------------------------------
\section{The Community}

The \cpp{} community is made of humans, so it is naturally composed of
a lot of good things and many problems. Sometimes simultaneously. And
since the language allows several paradigms and provides multiple
tools, there are approximately as many programming styles than there
are \cpp{} programmers.

Two typical behaviours appeared very common to me in the recent
years. The first one is about the tools provided by the standard
library. First you hear people complaining: ``The standard does not
even contain {\em feature}. One has to use {\em libfeature} for it.''
Cue to the release of the aforementioned feature, and suddenly: ``The
specs for {\em feature} in the standard prevent efficient
implementations. One has to use {\em libfeature} for it.''

The second behaviour is about build times. \Cpp{} is well known for
being the language of quite long to compile programs, especially when
the program contains templates or other metaprogramming
techniques. This is a topic that regularly comes as a major pain; it
was even listed as the second most frustrating thing in the 2021
Annual C++ Developer Survey
\cite{2021-annual-cpp-developer-survey}. Nevertheless, once you are on
the field you will meet developers being like ``Here's a header-only
library for {\em feature} relying heavily on templates and
metaprogramming stuff.'' Then you will hear complaints like
``Compilation times are too long!  The committee should do something
about that!''. Finally, there will be even other developers asking for
more header-only libraries because it is easy to integrate in a
project\footnote{Dependency management in \cpp{} is the main
  frustrating thing according to the aforementioned survey.}.

\bigskip

In one case, people's demands are ignored and they complain about
it. Then they receive what they asked for, and they still complain
about it.

The second case is more about resource management. People have a
limited computing power, and they use every drop of it until it
becomes unbearable. Then they ask others to solve their problem. They
also buy even more computing power, and still use every drop of
it. Doesn't it look suspiciously like some other real-life important
resource management problems?

\bigskip

Humans…

\chapter{Nice Things from C++11}
%-------------------------------------------------------------------------------
\section{Range-based For Loops}

Before \cpp11, if someone wanted to iterate over a container, he had
to construct an iterator, of the correct type with respect to the
container, and write the loop with the classical three steps:
initialization (get an iterator on the begining of the container),
stopping condition (the iterator is not at the end), and the loop
increment.

\begin{lstlisting}[language=c++]
void foo(std::vector<int>& v, int c)
{
  // I'm going to iterate over v, so I need an iterator.
  typedef std::vector<int>::iterator it_t;
  const it_t end(v.end());

  for (it_t it(v.begin()); it != end; ++it)
    *it *= c;
}
\end{lstlisting}

All of this was quite verbose and repetitive when using standard
containers. Starting from \cpp11, all this administrative stuff can be
avoided by using a ranged-based for loop.

\begin{lstlisting}[language=c++]
void foo(std::vector<int>& v, int c)
{
  // Just get all entries from v.
  for (int& vi : v)
    *it *= c;
}
\end{lstlisting}

This format uniformizes iteration over anything having a begining and
an end. Moreover, it handles the typical subtelties of for loops for
you: the end of the container is guaranteed to be computed only once
and the increment use the preincrement operator.

%-------------------------------------------------------------------------------
\section{Template Aliases}

Before \cpp11, you could simply not declare a typedef with a
parameterized type. For example, the following did not work:

\begin{lstlisting}[language=c++]
// You can't do that.
template<typename T>
typedef std::vector<T> collection;

// You can't do that either.
template<typename V>
typedef std::map<std::string, V> string_map;
\end{lstlisting}

The alternative was to use a struct to receive the type, then declare
the typedef inside the struct:

\begin{lstlisting}[language=c++]
// Workaround: nest the type.
template<typename V>
struct string_map
{
  typedef std::map<std::string, V> type;
};

// Easy to use, isn't it?
string_map<int>::type string_to_int;
\end{lstlisting}

Starting with \cpp11, the using keywork allows to declare templated
type aliases:

\begin{lstlisting}[language=c++]
// You can do that.
template<typename T>
using collection = std::vector<T>;

// You can also do that.
template<typename V>
using string_map = std::map<std::string, V>;

// Looks like a first-class type.
string_map<int> string_to_int;
\end{lstlisting}

\subsection{External Templates}

External templates are one of my favorite features from \cpp11.

When you were writing template code before \cpp11, for example a
template function, then every time the code was included in a
compilation unit then the compiler would instantiate all used
symbols. Check for example the header below:

\lstinputlisting[%
  title=some\_template-98.hpp%
]{%
  examples/extern-template/some_template-98.hpp%
}

If this header was included in two .cpp files, and its function
actually called, then its compiled code would be present in the object
file of each .cpp; something we can check with \code{nm}.

Following the example, let's compile the two files below:

\lstinputlisting[%
  title=foo-98.cpp%
]{%
  examples/extern-template/foo-98.cpp%
}

\lstinputlisting[%
  title=bar-98.cpp%
]{%
  examples/extern-template/bar-98.cpp%
}

Then \code{nm} would print:

\begin{lstlisting}[language=bash]
$ nm bar-98.cpp.o foo-98.cpp.o

bar-98.cpp.o:
000000000000001b T foo(int)
@\emcode{0000000000000036 W int some\_template<int>(int)}@

foo-98.cpp.o:
000000000000001b T bar(int)
@\emcode{0000000000000036 W int some\_template<int>(int)}@
\end{lstlisting}

Not only does this consume space (54 bytes per file here) but it also
costs a lot of work for the compiler and the linker. All of this adds
up, even for medium projects. Imagine that for each template function
or class the compiler parses the code, then it compiles it, then
writes all these bytes on disk; then all these bytes are read again by
the linker, who sorts all these duplicate symbols to keep only one.

At the end of this process, literally all the work done by the
compiler has been useless. Wouldn't it have been better not to do it
in the first place? The linker then spent more time removing the crap
rather than actually linking, and the programmer was wondering if he
could get a more powerful computer. Again.

\bigskip

Thankfully the \code{extern template} from \cpp11 allows us to skip
all this useless work. First we have to tell the compiler not to
instantiate the template by adding a single line to our header:

\lstinputlistinghl[%
  title=some\_template-11.hpp%
]{14}{%
  examples/extern-template/some_template-11.hpp%
}

This line tells the compiler not to instantiate
\code{some\_template()} when \code{T = int}. Then we add a .cpp file
where we explicitly ask for the instantiation:

\lstinputlisting[%
  title=some\_template-11.cpp%
]{%
  examples/extern-template/some_template-11.cpp%
}

That's it. Let's check with \code{nm}:

\begin{lstlisting}[language=bash]
$ nm bar-11.cpp.o foo-11.cpp.o some_template.cpp

bar-11.cpp.o:
000000000000001b T bar(int)

foo-11.cpp.o:
000000000000001b T foo(int)

some_template-11.cpp.o:
0000000000000036 W int some_template<int>(int)
\end{lstlisting}

The template function is indeed compiled in a single file, and absent
from the others.

We can push even further in this case, by having the whole body of
\code{some\_template()} in the .cpp file and only its signature in the
header. It would not only avoid the parsing of the code but, more
importantly, allow to remove from the header all include directives
related to implementation details.

\begin{guideline}
If you are writing template code for which you know some or all
instantiations, then add an \code{extern template} line for them in
the header, and explicitly instantiate them in an implementation file.
This is not always feasible, but when it can be done it should be done
\end{guideline}

\section{\code{std::nullptr\_t} and \code{nullptr}}

The traditional way to set a pointer to zero before \cpp11 was via the
\code{NULL} macro. So what would happen if you tried to compile the
following program?

\lstinputlisting{examples/nullptr/nullptr-98.cpp}

Did you expect the program to compile well and \code{foo(int*)} to be
called? Too bad, we are in a good old ambiguous call situation:

\begin{lstlisting}[language=bash]
error: call of overloaded ‘foo(NULL)’ is ambiguous
\end{lstlisting}

Since \code{NULL} is often defined as the integral value zero, the
compiler cannot distinguish it from an integer. Thus the error.

\bigskip

As an answer to this problem, \cpp11 introduces the \code{nullptr}
keyword, which exactly represents a zero pointer. Its type is
\code{std::nullptr\_t} and it is implicitly convertible to any
pointer. Consequently, the previous code, if converted as below,
compiles as expected.

\lstinputlisting[emph=nullptr]{examples/nullptr/nullptr-11.cpp}

\subsection{Anonymous Namespaces}

In the pre \cpp11{} era we could tell the compiler that a given global
variable or free function was only used in the current translation
unit by preceding its declaration with the keyword \code{static}:

\begin{lstlisting}
static int global_int;
static void some_function() { /* … */ }
\end{lstlisting}

Thanks to this keyword the same symbol (variable or function) could
appear in multiple translation units without problems.

However, this keyword is not applicable to type declarations. So what
happened if we needed a class definition at file scope but happened to
use the same name in different files, like with \code{struct foo}
below?

\lstinputlisting[title=foo.cpp]{examples/anonymous-namespace/foo.cpp}
\lstinputlisting[title=main.cpp]{examples/anonymous-namespace/main.cpp}

If we try to compile the these files for example with \code{g++
  foo.cpp main.cpp} then run the program, we would expect to have the
following output:

\begin{lstlisting}{language=bash}
$ ./a.out
foo.cpp 1
main.cpp 1
\end{lstlisting}

Instead, we may have the following output:

\begin{lstlisting}{language=bash}
$ ./a.out
foo.cpp 1
foo.cpp 1
\end{lstlisting}

I say ``may'' because this program violates the one definition rule,
which states that a symbol cannot have more than one definition in the
program. Here, however, the \code{struct foo} has two definitions, one
in each cpp file. We are thus entering the territory of undefined
behavior.

\bigskip

\Cpp11{} introduces the anonymous namespaces, which can totally solve
this issue. Any symbol defined in an anonymous namespace has internal
linkage. That means that it cannot conflict with anything defined
outside the current compilation unit. Its usage is as we would expect:

\begin{lstlisting}
// No identifier following the keyword.
namespace
{
  // Everything declared here is available in the current
  // compilation unit.
}
\end{lstlisting}

Concretely, when the code above is modified as follows, the program
is well defined and the output matches the expectations.

\lstinputlisting[
  title=foo-anonymous.cpp,
  emph=namespace
]{%
  examples/anonymous-namespace/foo-anonymous.cpp%
}

\lstinputlisting[
  title=main-anonymous.cpp,
  emph=namespace
]{%
  examples/anonymous-namespace/main-anonymous.cpp%
}


%-------------------------------------------------------------------------------
\subsection{Delegated Constructors}

Before \cpp11, constructors cannot call other constructors, so if you
want to share initialization code between multiple constructors,
like in the example below:

\begin{lstlisting}
struct foo
{
  foo(bar* b, float f, int i)
    : m_bar(b),
      m_f(f),
      m_i(i)
  {}
  
  foo(bar* b, int i)
    // can't I call foo(b, 0, i) directly?
    : m_bar(b),
      m_f(0),
      m_i(i)
    {}

private:
  bar* const m_bar;
  const float m_f
  const int m_i;
};
\end{lstlisting}

Then you have to put it in some separate member function called by the
constructor, like this:

\begin{lstlisting}
struct foo
{
  foo(bar* b, float f, int i)
  {
    @\emcode{init}@(b, f, i);
  }
  
  foo(bar* b, int i)
  {
    @\emcode{init}@(b, 0, i);
  }

private:
  void @\emcode{init}@(bar* b, float f, int i)
  {
    m_bar = b;
    m_f = f;
    m_i = i;
  }

  // We cannot make any of these member const anymore.
  bar* m_bar;
  float m_f;  
  int m_i;
};
\end{lstlisting}

This approach was kind of error-prone. Stuff may happen before and
after the call to the \code{init()} function, and actually nothing
prevent it to be called at any point in the life of the
instance. Finally, this is incompatible with \code{const} members.

Starting from \cpp11, a constructor can call another constructor:

\begin{lstlisting}
struct foo
{
  foo(bar* b, float f, int i)
    : m_bar(b),
      m_f(f),
      m_i(i)
  {}
  
  foo(bar* b, int i)
    : @\emcode{foo}@(b, 0, i)
  {}

private:
  bar* const m_bar;
  float m_f;
  int m_i;
};
\end{lstlisting}

This solves all problems.

%-------------------------------------------------------------------------------
\subsection{Deleted Constructors}

\problemtitle

Delegating constructors is a nice feature, but wat about totally
removing a constructor?

Consider the class below:

\begin{lstlisting}
struct scoped_listener
{
  scoped_listener(dispatcher& d, callback c)
    : m_dispatcher(&d)
  {
    m_id = m_dispatcher->connect(c);
  }

  ~scoped_listener()
  {
    m_dispatcher->disconnect(m_id)
  }

private:
  dispatcher* m_dispatcher;
  int m_id;
};
\end{lstlisting}

Creating copies of \code{scoped\_listener} does not make any sense, as
all copies would share the same \code{m\_id} and will thus trigger the
same call to \code{disconnect(int)} when destructed. See for example
its usage below:

\begin{lstlisting}
void foo()
{
  /* ... */
  scoped_listener listener(d, c1);
  scoped_listener copy(listener);
  scoped_listener other(d, c2);

  {
    // This assignment does not call disconnect(c2).
    other = listener;
    // disconnect(c1) is called here,
    // when other goes out of scope.
  }
  // disconnect(c1) is called twice here: in the
  // destruction of listener and copy.
}
\end{lstlisting}

One would typically want to forbid copies of \code{scoped\_listener} by
disabling its copy constructor.

Before \cpp11, one solution we would find here and there was to
declare the copy constructor and assignment operator as private. The
problem was that it was still available for the class and its
friends. So the programmer would then either implement an always
failing body for this constructor, emitting an error at run time, or
would just not implement the constructor, thus triggering an error at
link time.

These solutions were kind of weak, in the sense that the error, if
any, was presented quite late for the programmer, and with a not
obvious explanation.

\solutiontitle

Now, starting with \cpp11, the constructor and operators can be
explicitly deleted:

\begin{lstlisting}
struct scoped_listener
{
  scoped_listener(const scoped_listener&)@\emcode{ = delete}@;
  scoped_listener& operator=(const scoped_listener&)@\emcode{ = delete}@;

  scoped_listener(dispatcher& d, callback c)
    : m_dispatcher(&d)
  {
    m_id = m_dispatcher->connect(c);
  }

  ~scoped_listener()
  {
    m_dispatcher->disconnect(m_id)
  }

private:
  dispatcher* m_dispatcher;
  int m_id;
};
\end{lstlisting}

Using a deleted function will trigger a clear error from the compiler
when the call is encountered.

%-------------------------------------------------------------------------------
\subsection{Defaulted Constructors}

\problemtitle

Let's continue with constructors. Before \cpp11, each class would have
implicit constructors unless stated otherwise. One example of a
situation where an implicit constructor would not have been created
was the explicit declaration of a custom constructor by the
programmer. For example:

\begin{lstlisting}
struct foo
{
  foo(int) {}
};

int main()
{
  foo f1;     // fail
  foo f2(24); // ok
  foo f3(f2); // ok

  return 0;
}
\end{lstlisting}

In the above example, \code{foo} has no default constructor (but has
an implicit copy constructor). In order to have the default
constructor, the programmer had to implement one. The main problem
becomes maintenance: when new fields are added in the class, we have
to remember to update the constructor to initialize them.

\solutiontitle

Starting with \cpp11, the programmer can tell the compiler to
implement the constructor with what would have been the default
implementation if it was not deleted.

\begin{lstlisting}
struct foo
{
  foo()@\emcode{ = default}@;
  foo(int) {}
};

int main()
{
  foo f1;     @\emcode{// ok}@
  foo f2(24); // ok
  foo f3(f2); // ok

  return 0;
}
\end{lstlisting}


\subsection{Beginning and End of Sequence}

All containers from the STL have a \code{begin()} and \code{end()}
member function to get an iterator on the first element in the
sequence or, respectively, just after the last element. Unfortunately,
there is no such function for the most basic sequences, i.e. C-like
arrays, so code like that was sure to fail before \cpp11:

\begin{lstlisting}
template<typename Sequence, typename T>
void replace_existing
(Sequence& s, const T& old_value, const T& new_value)
{
  *std::find(s.begin(), s.end(), old_value) = new_value;
}

void foo()
{
  int a[] = { 1, 4, 3 };
  replace_existing(a, 4, 2);
}
\end{lstlisting}

Fortunately, \cpp11 introduces the \code{std::begin()} and
\code{std::end()} free functions which accepts a C-like
array\footnote{as long as you include
  \code{\textless{}array\textgreater}}. Now, this code will work in
all cases:

\begin{lstlisting}
template<typename Sequence, typename T>
void replace_existing
(Sequence& s, const T& old_value, const T& new_value)
{
  *std::find(@\emcode{std::begin(s)}@, @\emcode{std::end(s)}@, old_value) = new_value;
}

void foo()
{
  int a[] = { 1, 4, 3 };
  replace_existing(a, 4, 2);
}
\end{lstlisting}

\subsection{Iterator Successors and Predecessors}

\problemtitle

There are many kind of iterators: forward iterators, that can be
incremented one step at a time with \code{operator++}, bidirectional
iterators, that can additionally be decremented with
\code{operator--}, and random access iterators, that can be
incremented or decremented by any amount at once.

When writing a function taking an iterator whose type is templated,
incrementing an iterator by more than one unit cannot be done
directly, as it would fail for non-random iterators. Prior to \cpp11,
this was done with \code{std::advance()}.

\begin{lstlisting}
template<typename Iterator, typename F>
void every_n_items
(Iterator it, std::size_t count, std::size_t step, F f)
{
  for (; count > step; count -= step)
  {
    f(*it);
    std::advance(it, step);
  }
}
\end{lstlisting}

\code{std::advance()} accepts a negative distance, in which case it
will advance… hum… backwards. Note that the function modifies its
argument, so using it to get another iterator from a given one is
cumbersome:

\begin{lstlisting}
template<typename Iterator>
void inplace_adjacent_sums(Iterator first, const Iterator& last)
{
  if (first == last)
    return;

  // Two steps to get an iterator on the second element. Ideally we
  // would have wanted to limit its scope to the loop too.
  Iterator second = first;
  std::advance(second, 1);

  for (; second != last; )
  {
    *first += *second;
    first = second;
    std::advance(second, 1);
  }
}
\end{lstlisting}

\solutiontitle

\marginheader{<iterator>}%
%
\Cpp11 introduces the replacement functions \code{std::prev()} and
\code{std::next()}, which are mostly here to make things clear and
more convenient. The former is to be used to move the iterator
backwards, while the latter is used to move it forward. If no distance
is passed, then it defaults to one.

Note that they return a copy of the new iterator, instead of modifying
its argument.

\begin{lstlisting}
template<typename Iterator>
void inplace_adjacent_sums(Iterator first, const Iterator& last)
{
  if (first == last)
    return;

  // Here we can initialize the second iterator in the loop, reducing
  // its scope.
  for (Iterator second = @\emcode{std::next}@(first); second != last; )
  {
    *first += *second;
    first = second;
    std::advance(second, 1);
  }
}
\end{lstlisting}


\subsection{Rvalue References}

\problemtitle

In the function \code{build\_widget()} below, a temporary string is
created by the code \code{label + "def"}, then a reference to this
string is passed to the constructor of \code{widget}, where it is then
copied in \code{widget::m\_label}. Finally, the temporary is deleted.

\begin{lstlisting}
struct widget
{
  widget(const std::string& label)
    : m_label(label)
  {}

  std::string m_label;
};

void build_widget()
{
  std::string label("abc");
  widget w(label + "def");
}
\end{lstlisting}

Temporaries have the interesting property that they have no use but to
be consumed by an expression building another value. In other words,
in the example above, the purpose of the result of \code{label +
  "def"} is to end up in \code{widget::m\_label}. So why do we need a
copy? It would be more efficient to just pass the instance created by
the concatenation directly to the final variable.

\solutiontitle

To solve this problem, \cpp11 introduces the concept of rvalue
references, which are, to put it simply, references to
temporaries. These references are denoted with a double ampersand
\code{\&\&}. See the updated example, below:

\begin{lstlisting}
struct widget
{
  widget(@\emcode{std::string\&\&}@ label)
    : m_label(label)
  {}

  std::string m_label;
};

void build_widget()
{
  std::string label("abc");
  widget f(label + "def");
}
\end{lstlisting}

The signature of the constructor indicates that it accept temporaries
as arguments. However, this code is not enough to avoid copies of the
string yet, as the instruction \code{m\_label(label)} still copies the
string. The next step is to replace this instruction with
\code{m\_label(std::move(label))}, and then we will have no copy. This
is the topic of section \ref{move}.

Note that if the constructor's argument was \code{widget}, like in
\code{widget(widget\&\& that)}, then this would declare what is called
a move-constructor. This is a distinct constructor than the
copy-constructor and it is explicitly allowed to steal the internals
of its argument, as long as the instance behind the argument stays in
a valid state.


%-------------------------------------------------------------------------------
\label{move}

Copies! Copies everywhere! And dynamic allocations too!

So was the world before \cpp11. In these old days, if we had to pass
a large object to a function that needed its own instance of it, then
we would have certainly created a copy of it.

\begin{lstlisting}
struct foo
{
  foo(const std::vector<bar>& bars)
    // Here we have one allocaction,
    // plus copies of the vector elements.
    : m_bars(bars)
  {}

private:
  std::vector<bar> m_bars;
};

void baz(int n)
{
  // One allocation of n bars, plus their initialization.
  std::vector<bar> bars(n);

  // ...

  foo f(bars);
  // I don't need the bars anymore, but they are still here.
}
\end{lstlisting}

\Cpp11 introduces the move semantics and the \code{std::move()}
function. In practice it means far fewer copies, as the instances data
are transfered from one place to another.

\begin{lstlisting}
struct foo
{
  // Note the pass-by-value
  foo(std::vector<bar> bars)
    // Explicit transfer into m_bars, no allocation.
    : m_bars(@\emcode{std::move}@(bars))
  {}

private:
  std::vector<bar> m_bars;
};

void baz(int n)
{
  // The single allocation in this program.
  std::vector<bar> bars(n);

  // ...

  // I don't need the bars anymore, so I transfer them.
  foo f(@\emcode{std::move}@(bars));
}
\end{lstlisting}

%-------------------------------------------------------------------------------
\subsection{The Move Constructor and Assignment Operator}

In order to have this working, a new type of reference was added to
the language, and classes can use this reference in constructors and
assignment operators to implement the actual transfer of data from one
instance to another.

\begin{lstlisting}
struct foo
{
  foo(foo@\emcode{\&\&}@ that)
    : m_resource(that.m_resource)
  {
    that.m_resource = nullptr;
  }

  foo& operator=(foo@\emcode{\&\&}@ that)
  {
    std::swap(m_resource, that.m_resource);
  }
};

// All of these call the constructor defined above.
foo f(foo());
foo f(construct_a_foo());
foo f(std::move(g));

// All of these call the assignment operator defined above.
f = foo();
f = construct_a_foo();
f = std::move(g);
\end{lstlisting}

The parameter of such functions is considered a temporary on the call
site. They are actually not restricted to constructors or assignment
operators and can be used in any function.

The semantics of such arguments must be interpreted as ``I {\em may}
steal your data''. The {\em may} is important here, as there is no
guarantee on the call site that the instance going through an
\code{std::move} is actually moved.

%-------------------------------------------------------------------------------
\subsection{\code{std::move}, the Subtleties}

Naming quality: \faStar\faStarO\faStarO\faStarO\faStarO. Would have
put zero stars if I could.

As a great example of ``naming things is hard'', \code{std::move} does
not move anything. It actually just marks the value as transferable,
by casting it to an rvalue-reference. See this actual implementation
exctracted from GCC 9:

\begin{lstlisting}
template<typename _Tp>
constexpr typename std::remove_reference<_Tp>::type&&
move(_Tp&& __t) noexcept
{
  return static_cast
  <
    typename std::remove_reference<_Tp>::type&&
  >(__t);
}
\end{lstlisting}

\begin{guideline}
  As a rule of thumb of its usage, consider two use cases:

  \begin{enumerate}
  \item If callee \underline{always} takes ownership of the value:
    prefer arguments passed by value.
  \item If callee \underline{may} take ownership of the value: use an
    rvalue reference.
  \end{enumerate}
\end{guideline}

\begin{lstlisting}
// "I need my own bar."
void foo_copy(bar b) { /* ... */ }

// "I may take ownership of b...
//  ...Unless I don't."
void foo_rvalue(bar&& b) { /* ... */ }

void baz()
{
  bar b1;
  // Valid, b1 is unchanged in baz.
  foo_copy(b1);

  // Valid, we can forget about b1 now.
  foo_copy(std::move(b1));

  bar b2;
  // Invalid, b2 is not an rvalue
  // foo_rvalue(b2);

  // Valid, but is b2 actually moved?
  foo_rvalue(std::move(b2));
}
\end{lstlisting}

%-------------------------------------------------------------------------------
\subsection{\code{std::forward} and Universal References}

When \code{\&\&} is used in association with a type to be deduced,
like in the function below:

\begin{lstlisting}
template<typename T>
void foo(T&& arg)
{
  /* … */
}
\end{lstlisting}

Then it refers neither to an rvalue-reference nor a reference. In this
context, it is called a \emph{universal reference}. Depending on how
the function is called then \code{T\&\&} will be either deduced to an
rvalue-reference, if the argument is an rvalue, or else the references
are collapsed and \code{T\&\&} becomes an lvalue-reference, just like
\code{T\&}.

\begin{lstlisting}
void bar()
{
  std::string s;

  // calls foo(std::string&)
  foo(s);

  // calls foo(std::string&&)
  foo(s + "abc");
}
\end{lstlisting}

Now how one should pass to another function an argument received as a
universal reference? If it is deduced as an rvalue-reference, then
\code{std::move()} should be used, otherwise it should be passed as
is.

Fortunately \cpp11 provides \code{std::forward()} to do the check for
us:

\begin{lstlisting}
template<typename T>
void foo(T&& arg)
{
  // Pass as an rvalue-reference or by address, whichever fits.
  foobar(std::forward<T>(arg));
}
\end{lstlisting}


\subsection{Lambdas}
\label{sec:lambda}

\problemtitle

How would we pass a custom comparator to \code{std::max\_element}
before \cpp11, say to compare strings by increasing size? Certainly by
writing a free or static function taking two strings as arguments and
returning the result of the comparison.

\begin{lstlisting}
// Comparator for strings by increasing size.
// Declared as a free function.
static bool string_size_less_equal
(const std::string& lhs, const std::string& rhs)
{
  return lhs.size() < rhs.size();
}

std::size_t
largest_string_size(const std::vector<std::string>& strings)
{
  // The comparator is outside the scope of this function.
  return
    std::max_element
      (strings.begin(), strings.end(), &string_size_less_equal)
    ->size();
}
\end{lstlisting}

Now what if we needed a parameterized comparator, for example to call
\code{std::find\_if} to search for a string of a specific size? The
free function would not allow to store the size, so we would
certainly write a functor object, i.e. a \code{struct} storing the
size parameter and defining an \code{operator()} receiving a string
and returning the result of the comparison.

\begin{lstlisting}
// Need a function object if the comparator has parameters.
struct string_size_equals
{
  std::size_t size;

  bool operator()(const std::string& string) const
  {
    return string.size() == size;
  }
};

bool has_string_of_size
(const std::vector<std::string>& strings, std::size_t s)
{
  string_size_equals comparator = {s};

  return
    std::find_if(strings.begin(), strings.end(), comparator)
      != strings.end();
}
\end{lstlisting}

\solutiontitle

Having the comparators as independent objects or functions is nice if
they are used in many places, but for a single use it is undoubtedly
too verbose. And confusing too. Wouldn't it be clearer if the
single-use comparator was declared next to where it is used?
Hopefully this is something we can use with lambdas, starting with
\cpp11:

\begin{lstlisting}
bool has_string_of_size
(const std::vector<std::string>& strings, std::size_t s)
{
  // Only three lines for the equivalent of the type
  // declaration, definition and the instantiation.
  // The third argument is a lambda.
  return std::find_if
    (strings.begin(), strings.end(),
     [=](const std::string& string) -> bool
     {
       return string.size() == s;
     })
    != strings.end();
}
\end{lstlisting}

\begin{guideline}
  Mind the next reader: keep your lambdas small.
\end{guideline}

%-------------------------------------------------------------------------------
\subsubsection{The Internals of Lambdas}
\label{lambdas-internals}

A declaration like

\begin{lstlisting}
[a, b, c]( /* arguments */ ) -> T { /* statements */ }
\end{lstlisting}

is equivalent to

\begin{lstlisting}
struct something
{
  T operator()( /* arguments */ ) const { /* statements */ }

  /* deduced type */ a;
  /* deduced type */ b;
  /* deduced type */ c;
};
\end{lstlisting}

Actually the compiler will create a unique type like this struct for
every lambda we write.

More conceptually, the parts of a lambda are the following:

\bigskip

\begin{center}
[{\it capture}]({\it arguments}) {\it specifier} -\textgreater
        {\it return\_type} \{ {\it body} \}
\end{center}

\bigskip

Where:

\begin{itemize}
\item {\it capture} is a list of variables from the parent scope that
  must be accessible inside the lambda. Use \code{[=]} to tell the
  compiler to automatically copy any variable used by the lambda, or
  \code{[\&]} to keep a reference to the corresponding variables from
  the parent scope. Variables can also be captured in a fine-grained
  way, e.g. \code{[=a, \&b]} to copy the value of \code{a} but store a
  reference to \code{b}.
\item {\it arguments} are the arguments of the function.
\item By default the variables captured by value cannot be assigned to
  in the body, unless we put the mutable keyword in {\it
    specifier}. In effect, it removes the \code{const} from
  \code{operator()}.
\item {\it return\_type} is the type of the value returned by this
  function, if any, or void otherwise. Contrary to any other
  function, the return type appear after the argument list instead of
  before.
\item {\it body} is the body of the function.
\end{itemize}


\subsection{Type Deduction with \code{decltype}}
\label{decltype}

\problemtitle

During the pre-\cpp11 era every variable, member, argument, etc. has
to be explicitly typed. For example, if we were writing a template
function operating on a range, how could we declare a variable of the
type of its items?

\begin{lstlisting}
template
<
  typename InputIt,
  typename OutputIt,
  typename Predicate,
  typename Transform,
>
void transform_if
(InputIt first, InputIt last, OutputIt out,
 Predicate& predicate, Transform& transform)
{
  for (; first != last; ++first)
  {
    /* some_type */ v = *first;
    if (predicate(v))
    {
      *out = transform(v);
      ++out;
    }
  }
}
\end{lstlisting}

\solutiontitle
Getting the correct type to put in place of \code{some\_type} was not
obvious, and required a fair share of template metaprogramming. Now,
with the \code{decltype} specifier introduced in \cpp11, the
programmer can tell the compiler to use ``the type of this
expression''.


\begin{lstlisting}
template
<
  typename InputIt,
  typename OutputIt,
  typename Predicate,
  typename Transform,
>
void transform_if
(InputIt first, InputIt last, OutputIt out,
 Predicate& predicate, Transform& transform)
{
  for (; first != last; ++first)
  {
    @\emcode{decltype(*first)}@& v = *first;
    if (predicate(v))
    {
      *out = transform(v);
      ++out;
    }
  }
}
\end{lstlisting}

In the example above, \code{decltype(*first)} is the type of the
result of dereferencing \code{first}.

\section{Auto}

During the pre-\cpp11 every variable, member, argument, etc. has to be
exactly typed. For example, if you were writing a template function
receiving another function as an argument, how could you store the
result of a call of this function in a local variable?

\begin{lstlisting}
template<typename F>
void foo(F& f)
{
  /* some_type */ r = f();
}
\end{lstlisting}

Getting the correct type to put in place of \code{some\_type} was not
obvious, and required a fair share of template metaprogramming. But at
least it was feasible.

Now what about storing a lambda in a local variable, what would be the
type of the variable? As seen in \ref{lambdas-internals}, the type of
the lambda is generated by the compiler, and thus out of reach for the
programmer. The solution is then found in the \code{auto} keyword,
introduced in \cpp11.

\begin{lstlisting}
bool has_string_of_size
(const std::vector<std::string>& strings, std::size_t s)
{
  @\emcode{auto}@ predicate =
     [=](const std::string& string) -> bool
     {
       return string.size() == s;
     };

  return
    std::find_if(strings.begin(), strings.end(), predicate)
    != strings.end();
}
\end{lstlisting}

%-------------------------------------------------------------------------------
\subsection{Auto As A Type Placeholder}

The \code{auto} keyword tells the compiler to deduce the actual type
of a variable from whatever is assigned to it. It can be used every
time a type should be typed, as long as there is an expression the
compiler can use to find the type. It can be augmented with
\code{const} or \code{\&}.

A typical use is for iterating over an associative container in a for
loop:

\begin{lstlisting}
template<typename F>
void for_each_entry(const std::map<int, int>& m, F& f)
{
  for (const auto& e : m)
    f(e.first, e.second);
}
\end{lstlisting}

In the spirit of \ref{range-based-for-loops}, the type of \code{e} is
deduced to \code{std::map\textless{}int,
  int\textgreater::value\_type}, to which are added the \code{const}
and the \code{\&}.

When used as a return type, the \code{auto} keyword allows to defer
the declaration of the actual return type after the argument list. For
example, if you don't know what is the type but you know it is exactly
the one of a given expression, you can combine this with
\code{decltype}:

\begin{lstlisting}
template<typename F>
auto foo(F&& f) -> decltype(f())
{
  return f();
}
\end{lstlisting}

%-------------------------------------------------------------------------------
\subsection{When Not To Use \code{auto}}

It is tempting to use the \code{auto} keyword everywhere, especially
for programmers coming from loosely typed languages such as Python or
JavaScript. Moreover, using \code{auto} gives the satisfying feeling
of writing seemingly ``generic'' code that can work with whatever type
is deduced.

In practice, the use of this keyword has lead to very painful to read
code, where nothing can be understood without going through every
expression assigned to an \code{auto} variable. This is a very high
load to pass to the next reader.

\begin{guideline}
Mind the next reader; write what you mean.
  
Declaring a variable or a function as \code{auto} is like writing an
innuendo, and innuendos make the communication more difficult; if you
see what I mean.

As a rule of thumb, use \code{auto} only if:
\begin{itemize}
\item there is no other way to write the type (e.g. assigning a
  lambda to a variable),
\item \underline{maybe} if the type is a well known idiom (e.g. \code{auto it =
  some\_container.begin()}, or for a loop variable in a range-based for
  loop over an associative container), but I would argue that writing
  the actual type would still be more explanatory for the reader.
\end{itemize}

Absolutely never use \code{auto} in place where you could have used
basic types like \code{int} or \code{bool}, or by laziness. It is not
because it is shorter that it improves the readability. On the
contrary, the readability is improved by being clear about what is
going on.
\end{guideline}


\subsection{Smart Pointers}

Memory allocation in \cpp{} is a tough subject, dynamic allocation
being the hardest part.

Before \cpp11, the programmer had to be very careful with the life
span of dynamically allocated memory in order to, first, be sure that
it is released at some point and, second, that no access is made to it
once it has been released, not even another release.

See how many problems could occur with this short snippet:

\begin{lstlisting}
struct foo
{
  foo(int* p)
    : m_p(p)
  {}

  ~foo()
  {
    delete m_p;
  }

private:
  int* m_p;
};

void bar()
{
  foo f1(new int);
  foo f2(f1);

  int* p1(new int);
}
\end{lstlisting}

There are two problems with this code. First, \code{foo} does not
define nor disable its copy constructor, so, by default, its
\code{m\_p} pointer will be copied to the new instance. Then, when the
original instance and its copy will be destroyed, both will call
\code{delete} on the same pointer. This is what will happen with
\code{f1} and its copy \code{f2}. In the best scenario the program
would crash here.

The second problem is \code{p1}. This pointer points to a dynamically
allocated \code{int} for which no \code{delete} is written. When the
pointer will go out of scope then there will be no way to release the
allocated memory.

%-------------------------------------------------------------------------------
\subsubsection{Self-Deleting Non-Shared Pointer}

\marginheader{<memory>}%
%
\Cpp11 introduces a pointer wrapper named \code{std::unique\_ptr},
which has the merit of automatically calling \code{delete} on the
pointer upon destruction. Applied to the previous example, it would
solve one problem and force us to find a solution for the other:

\begin{lstlisting}
struct foo
{
  foo(std::unique_ptr<int> p)
    : m_p(std::move(p))
  {}

  // The default destructor does the job.

private:
  std::unique_ptr<int> m_p;
};

void bar()
{
  std::unique_ptr<int> p1(new int);

  foo f1(std::unique_ptr<int>(new int));
  foo f2(std::move(f1));
}
\end{lstlisting}

This program is undoubtedly safer than the previous one. The copy
constructor of \code{foo} is still not defined, but it is for sure
deleted since \code{std::unique\_ptr} has no copy constructor
neither. So, by default, we cannot share the resource between two
instances, which is great. The only solution here is to either
allocate a new int for \code{f2} or steal the one from {f1}. The
latter is implemented here.

Then, for the release of \code{p1}, it is automatically done when the
variable leaves the scope, so no memory is leaked.

\bigskip

One of the best features of \code{std::unique\_ptr} is the possibility
to use a custom deleter to release the pointer. This makes this smart
pointer a tool of choice when using C-like resources.

Check for example the use case of libavcodec's
\code{AVFormatContext}. The format context is obtained via a call to
\code{avformat\_open\_input(AVFormatContext**, const char*,
  AVInputFormat*, AVDictionary**)} and must be released by a call to
\code{avformat\_close\_input(AVFormatContext**)}. With the help of
\code{std::unique\_ptr} this could be done as follows:

\begin{lstlisting}
namespace detail
{
  static void close_format_context(AVFormatContext* context);
}

void foo(const char* path)
{
  AVFormatContext* raw_context_pointer(nullptr);

  const int open_result =
    avformat_open_input
      (&raw_context_pointer, path, nullptr, nullptr);

  std::unique_ptr
  <
    AVFormatContext,
    decltype(&detail::close_format_context)
  >
  context_pointer
  (raw_context_pointer, &detail::close_format_context);

  // …
}
\end{lstlisting}

With this approach, the format context will be released via a call to
\code{detail::close\_format\_context} as soon as
\code{context\_pointer} leaves the scope. Do we know if the memory
pointed by \code{raw\_context\_pointer}? We do not, and it does not
matter. What we have here is a simple robust way to attach a release
function to an acquired resource.

%-------------------------------------------------------------------------------
\subsubsection{Self-Deleting Shared Pointer}

\marginheader{<memory>}%
%
I have a hard time trying to find a use case where
\code{std::shared\_ptr} is the best solution, so let's just focus on a
good-enough solution.

This smart pointer is the answer to the problem of having a
dynamically allocated resource that may outlive its creator, and that
may also be accessed from two independent owners. In this case, the
ownership of the resource is unclear, as it is \emph{shared}, so the
idea is to keep the resource alive until all its owner release it.

For example, let's say we have a function whose role is to dispatch a
message to multiple listeners, whom will not process it right now. For
some reason the message cannot be copied, so we cannot send a copy of
it to everyone:

\begin{lstlisting}
void dispatch_message
(const std::vector<listener*>& listeners,
 const std::string& raw)
{
  message m = parse_message(raw);
  std::shared_ptr<message> p
    (std::make_shared<message>(std::move(m)));

  for(listener* l : listeners)
    listener->add_to_queue(m);
}
\end{lstlisting}

With this implementation the message outlives the scope of
\code{dispatch\_message()} and will remain in memory until all
listeners have released it.

Is it the best solution for this problem? Honestly I doubt
that. Actually, most uses of \code{std::shared\_ptr} I have seen,
including mine, looked a bit like a lack of reasoning about resource
management.

Wouldn't it have been better if the dispatching and the listeners were
scheduled in a loop and if the dispatcher would own the messages for
some iterations, or until they would be marked as processed by the
listeners? Do we really have to pay for dynamic allocations here?

\begin{guideline}
When you find yourself thinking that a shared resource with no clear
life span — a \code{std::shared\_ptr} — is a good answer to your
problem, please double check your solution, and ask for a second
opinion.
\end{guideline}

\begin{pitfall}
Many documentations declare, rightfully, that \code{std::shared\_ptr}
is thread-safe, and I have seen people using it to \emph{safely}
access the allocated resource in a concurrent way.

\bigskip

It must be said that the only thread-safe part in a
\code{std::shared\_ptr} is its reference counter.

\bigskip

Nothing is done — nothing can be done — to provide an automatic
thread-safe access to the allocated resource. Actually, even having
two threads doing respectively a copy (i.e.\ \code{std::shared\_ptr<T>
  p(sp)}) and a deletion (i.e.\ \code{sp.release()}), for the same
shared pointer \code{sp}, has no defined outcome without additional
synchronization. The only guarantee is that the increment of the
counter in the copy won't be done between the decrement and the
deletion from the release.

\end{pitfall}

%-------------------------------------------------------------------------------
\subsubsection{Shared Pointer Observer}

\marginheader{<memory>}%
%
What happens when the instance pointed by a \code{std::shared\_ptr}
owns a \code{std::shared\_ptr} pointing to the owner of the former?
Then we have a cycle and none of the instances will be released.

\begin{lstlisting}
struct foo;

struct bar
{
  std::shared_ptr<foo> m_foo;
};

struct foo
{
  std::shared_ptr<bar> m_bar;
};

void foobar()
{
  std::shared_ptr<foo> f(std::make_shared<foo>());
  std::shared_ptr<bar> b(std::make_shared<bar>());
  b->m_foo = f;
  f->m_bar = b;

  // The instances pointed by foo and foo->bar won't be deleted.
}
\end{lstlisting}

In order to break this kind of dependency loop, one pointer should be
declared as an \code{std::weak\_ptr}. Just like
\code{std::shared\_ptr}, this smart pointer is here to point to a
shared resource, except that it does not count as an owner of the
resource. Additionally, it provides a way to test if the resource is
still available via the \code{std::weak\_ptr::lock()} function, which
returns a shared pointer on the resource if it is available, or
\code{nullptr} otherwise.

\begin{lstlisting}
struct foo;

struct bar
{
  @\emcode{std::weak\_ptr}@<foo> m_foo;
};

struct foo
{
  std::shared_ptr<bar> m_bar;
};

void foobar()
{
  std::shared_ptr<foo> f(std::make_shared<foo>());
  std::shared_ptr<bar> b(std::make_shared<bar>());
  b->m_foo = f;
  f->m_bar = b;

  {
    std::shared_ptr<foo> resource(b->m_foo.lock());
    if (resource)
      printf("This message is printed.\n");
  }

  f.reset();

  std::shared_ptr<foo> resource(b->m_foo.lock());

  if (resource)
    printf("This message is not.\n");
}
\end{lstlisting}


% "using" parent function.
% override
% final
% strongly typed enum (enum class)
% - variadic templates
% - constexpr
% initializer list
% uniform initialization
% static_assert

% - trailing return types
% explicit for operators
% unrestricted unions
% string literals
% user defined literals
% long long int
% sizeof on members
% alignof alignas
% attributes

% <thread>
% <tuple>
% <unordered_set>
% <unordered_map>
% <regex>
% <random>
% reference_wrapper
% std::function
% std::bind
% type_traits
% result_of


% RVO?

\chapter{Nice Things from C++14}

\Cpp14 is frequently qualified as a bugfix version of \cpp11, indeed
most features are improvements or extensions of things introduced in
the latter.

Nevertheless, it does not mean that these improvements are not worth
it. Let's see.

\section{At the Language Level}
\subsection{Number separator}
\label{number-separator}

\problemtitle

Long numbers are hard to read. Check how long you need to get the
numbers in this \cpp11 code:

\begin{lstlisting}
int wisconsin_population = 5822434;
int california_population = 39512223;
\end{lstlisting}

My guess is that you grouped the digits by three in your mind in order
to get the numbers right. Didn't you? Usually when we write large
numbers like that, we group the digits with a separator to make the
numbers easier to read, like in 1'234'567.

\solutiontitle

Since \cpp14 we can use this syntax in the code:

\begin{lstlisting}
int wisconsin_population = 5'822'434;
int california_population = 39'512'223;

// It works with float numbers too.
const float f = 1'111.222'222;

// And also with for non-decimal numbers. Funny thing, the quote can
// appear anywhere between two digits.

const unsigned mask = 0xfff'0'00'ea;
\end{lstlisting}


% binary suffix
% number separator
% deprecated
% lambda: [](auto arg)
% lambda capture expression (move)
% omit the return type
% decltype(auto)
% better constexpr
% variable templates
% [[deprecated]]
% new/delete elision

\section{In the Standard Library}
% make_unique
% integer_sequence
% exchange
% tuple addressing by type
% standard user defined literals
% heterogeneous lookup in associative containers
% enable_if_t (type aliases)
% quoted


\bibliographystyle{alpha}
\bibliography{bibliography.bib}

\appendix
\input{parts/license.tex}

\end{document}
